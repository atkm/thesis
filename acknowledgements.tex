I am fortunate to have two excellent advisors, Tom Wieting and Rao Potluri.
They were the instructors for the two core courses of my junior year, the year when I came to Reed as a transfer student.
As some of you may know, hopping onto a moving train is not an easy task.
My attempt to jump onto a train called Reed mathematics was not pain-free, but overall a delightful experience thanks to the two professors and their beautiful presentations of mathematics.
It was my great pleasure to have the opportunity to study under them again in my senior year.
They provided me with the right amount of guidance to make this year-long project a rewarding endeavor.
Whenever I walked into his office abruptly, Tom made himself available to stimulate my thoughts, even during his sabbatical.
I tried to imitate the style of his essay, 'Metric Spaces,' although a lot more needs to be done to achieve the level of sophistication.
A lot of thanks goes to Rao for patiently listening to and reading my mathematics.

I owe much to Nelia Mann for introducing me to chaos theory as well as providing me with one of the best academic experiences at Reed.
Ray Mayer for inspiring me to study the outer billiards, which I ended up being obsessed with for weeks.
%Colorado College and Virginia Tech for 
Polytopia, the breeding ground of this document, and the folks who happened to be there with me at strange hours.

\vspace{1cm}
{\small
お父さん、お母さん、アメリカへ、そしてリード大学へ行かせてくれてありがとう。
長いこと自分勝手を許してもらって感謝しています。
木下、卒論を書こうとするといつも、カオス、カオスとブツブツ言いながら新長田をうろついた日々を思い出してなかなか集中できませんでした。
まさか真剣にカオスの勉強をすることになるとは夢にも思っていなかったあの頃。
皆さんどーもありがとう。
}
