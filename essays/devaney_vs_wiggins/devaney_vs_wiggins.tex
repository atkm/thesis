\documentclass[11pt]{book}
\usepackage{../../thesis}

\begin{document}
\section{A comparison of Devaney's and Wiggins's definitions}
(Example 3.4 of martelli's article)
\begin{proposition}
  (A mapping that is chaotic in Wiggins's sense but not so in Devaney's sense)
  Consider a mapping $F: D \to D$, where $D = \set{x \in R^2: \norm{x} \leq 1}$, defined as
  \begin{equation*}
    F(\rho, \theta) = (4\rho(1 - \rho), \theta + 1).
  \end{equation*}
\end{proposition}
For no $n$, $\theta$ is coming back to the original angle. 
Hence there is no periodic orbit, and Devaney's definition does not apply.

It can be shown that $F$ is topologically transitive in $D$, and $F$ exhibits sensitive dependence on initial conditions.

Also, there exists $x_0$ (in fact most points satisfy) $L(x_0) = D$
because the logistic map visits every radius, and the angle also visits every angle.
Unstability of $O(x_0)$ follows from the fact that $F$ has sensitive dependence on initial conditions.
 

\bibliographystyle{pjgsm}
\bibliography{../../bibliography/thesis}

\end{document}

