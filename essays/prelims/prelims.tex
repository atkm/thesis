\documentclass[12pt,twoside,draft]{book}
\usepackage{../../thesis}

\makeindex
\begin{document}

\chapter{Preliminaries}
I will start this chapter with a discussion of metric space.
Then I will define dynamical systems, and introduce notions that will be used frequently.
The main purpose of this chapter is to introduce notations used in the subsequent chapters. 
Since each definition of chaos requires different ideas in analysis/dynamical systems, we introduce those ideas as required on the fly.
%The reader may prefer to come back to this chapter as he/she encounters unfamiliar notations.

%%% Metric
\section{Metric Spaces}
This section introduces notations used in the rest of the book.
Also, we prove the contraction fixed point theorem.

\begin{definition}
  (Metric space)
  A binary operation $\delta$ is called a \textit{metric} if it satisfies the following properties for each $x,y,z \in X$:
  \begin{enumerate}
    \item $\metric{x,y} = 0 \mbox{ iff } x = y$
    \item $\metric{x,y} = \metric{y,x}$
    \item $\metric{x,y} \leq \metric{x,z} + \metric{z,y}$.
  \end{enumerate}
  A \textit{metric space} $X$ is a set equipped with a metric.
  \index{metric}
\end{definition}
%%%
\begin{definition}
  (Neighborhood)
  Let $X$ be a metric space, and $\delta$ be a metric on $X$.
  For $\epsilon \in \R$, and $x \in X$, a (open) \textit{neighborhood} $\oball{\epsilon}{x}$ is defined as 
  \begin{equation*}
    \oball{\epsilon}{x} \ceq \setst{y}{\metric{y,x} < \epsilon}
  \end{equation*}
  I shall use the terms ``open ball'' and ``neighborhood'' interchangeably.
\end{definition}
%%%
%\begin{definition}
%  (Closed neighborhood)
%  Let $X$ be a metric space, and $\delta$ be a metric on $X$.
%  For $\epsilon \in \R$, and $x,y \in X$, a \textit{closed neighborhood} $\cball{\epsilon}{x}$ is defined as the closure of the corresponding open neighborhood, i.e.
%  \begin{equation*}
%    \cball{\epsilon}{x} \ceq \cl(\oball{\epsilon}{x}).
%  \end{equation*}
%\end{definition}

\begin{definition}
  (Diameter of an set)
  We denote the \textit{diameter} of $A \subseteq X$ by $\diam(A)$, and is defined as
  \begin{equation*}
    \sup\limits_{x,y \in A} \metric{x,y}.
  \end{equation*}

\end{definition}

\begin{definition}
  (Lipschitz constant; Contraction)
  Let $X, Y$ be metric spaces, and $\delta_X$ and $\delta_Y$ be corresponding metrics.
  A map $F: X \to Y$ is said to be a \textit{Lipschitz map} or to satisfy the \textit{Lipschitz condition} iff there exists a $C \in \R$ such that for each $x_1, x_2 \in X$,
  \begin{equation}
    \delta_Y(F(x_1),F(x_2)) \leq C \delta_X(x_1, x_2). 
    \label{eqn:lipcond}
  \end{equation}
  Clearly, a Lipschitz map is uniformly continuous.
  We define the \textit{Lipschitz constant}, $\Lip(F)$, of $F$ as the greatest lower bound of the set of all $C$, i.e.
  \begin{equation*}
    \Lip(F) \ceq \inf\setst{C}{C \mbox{ satisfies } \eqref{eqn:lipcond}}.
  \end{equation*}
  Furthermore, $F$ is called a \textit{contraction} if $\Lip(F) < 1$.
  \index{Lipschitz constant}
\end{definition}
%%%

\begin{theorem}
  (Contraction Fixed Point Theorem)
  Let $(X,\delta)$ be a complete metric space, and $K: X \to X$ a contraction with $\Lip(K) = C$.
  Then, $K$ has a unique fixed point.
  \label{thm:cfp}
  \index{contraction fixed point theorem}
  %
  \begin{proof}
    For any point $x_0 \in X$, define
    \begin{equation*}
      x_n \ceq \itr{K}{n}(x_0).
    \end{equation*}
    We have
    \begin{equation*}
      \delta(x_{n+1}, x_n) \leq C \delta(x_n, x_{n-1}),
    \end{equation*}
    which implies
    \begin{equation*}
      \delta(x_{n+1}, x_n) \leq C^n \delta(x_1, x_0).
    \end{equation*}
    For any $m > n$, we have
    \begin{align*}
      \delta(x_m, x_n) &\leq \sum\limits_n^{m-1} \delta(x_{i+1}, x_i) 
        \leq (C^n + C^{n+1} + \cdots + C^{m-1}) \cdot \delta(x_1, x_0) \\
        & = C^n (C^{0} + \cdots + C^{m-n-1} ) \cdot \delta(x_1, x_0) \\
        & = \frac{C^n(1 - C^{m-n})}{1-C} \cdot \delta(x_1, x_0)
        \leq \frac{C^n}{1-C} \cdot \delta(x_1, x_0).
    \end{align*}
    Therefore, $\set{x_n}$ is Cauchy.
    Since $X$ is complete, the sequence converges to a point in $X$, say $x$.
    Finally, 
    \begin{equation*}
      K(x) = \lim\limits_{n\to \infty} K(x_n) = \lim\limits_{n\to \infty} x_{n+1} = x.
    \end{equation*}
    Thus, $x$ is a fixed point for $F$.
  \end{proof}
\end{theorem}
%%%

\section{Topological Dynamical Systems}
Although we have encountered the concept of iteration in the preceding chapter, I formally define the notion here.
\begin{definition}
  (Iteration)
  Let $X$ be a metric space, and $F$ a continuous mapping.
  \textit{The $n$th iteration of F}, denoted $\itr{F}{n}$, is defined as follows
  \begin{equation*}
    \itr{F}{n} \ceq \underbrace{F \circ F \circ \cdots \circ F}_{n} .
  \end{equation*}
\end{definition}
%%%
\begin{definition}
  (Orbit)
  Let $F: X \to X$. 
  The \textit{orbit} of $x_0$ is defined as a sequence
  \begin{equation*}
    O_F(x_0) \ceq \seq{\itr{F}{n}(x_0)}{\infty}{n = 0}
  \end{equation*}
  $O_F(x_0)$ will simply be denoted as $O(x_0)$ when there is no ambiguity in the choice of $F$ in the context.
  \label{def:orbit}
  \index{definition of!orbit}
\end{definition}
%%%
\begin{definition}
  (Periodic points)
  If $\itr{F}{n}(p) = p$ for some $n > 0$, and $\itr{F}{m}(p) \neq p$ for each $1 \leq m < n$, then $p$ is called a periodic point of period $n$, or simply a $n$-periodic point.
  \label{def:porbit}
  \index{periodic!orbit}
\end{definition}

It is convenient for our purpose to extend the notion of metric to a point and an orbit.
\begin{definition}
  (Distance between a point and an orbit)
  Let $X$ be a metric space, and $F: X \to X$ a continuous mapping.
  For each $x,y \in X$, the metric between $O(x)$ and $y$ is defined as follows
  \begin{equation*}
    \metric{O(x),y} \ceq \sup\limits_{x' \in O(x)} \metric{x',y}
  \end{equation*}
\end{definition}

In topological dynamical systems, we are interested in properties that are preserved under continuous mappings.
Thus, two dynamical systems are ``the same'' when there exists a (homeomorphic) conjugacy between them.
All definitions of chaos that we discuss in this book are topological invariants.
Since properties that we are specifically interested are global properies of a given space, we will require a conjugacy to be surjective. 
\begin{definition}
  We say that $G$ is \textit{semi-conjugate} to $F$ (by $\phi$), if there exists a continuous and surjective mapping $\phi$ such that $\phi\circ F = G\circ\phi$.
  Also, we say that $G$ is \textit{conjugate} to $F$ if $G$ is semi-conjugate to $F$ and vice versa (i.e. $\phi$ is a homeomorphism).
\end{definition}

\bibliographystyle{../../pjgsm}
\bibliography{../../bibliography/thesis}

\printindex
\end{document}
