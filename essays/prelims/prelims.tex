\documentclass[12pt,twoside,draft]{book}
\usepackage{../../thesis}

\makeindex
\begin{document}

\chapter{Preliminaries}
A metric space underlies a dynamical system.
I will start this chapter with a discussion of metric space.
Then I will define dynamical systems, and introduce notions that will be used frequently.
The purpose of this chapter is to introduce concepts and notations used in the subsequent chapters. 
The reader may prefer to come back to this chapter as he/she encounters unfamiliar notations.

%%% Metric
\section{Metric Spaces}
\begin{definition}
  (Metric space)
  A binary operation $\delta$ is called a \textit{metric} if it satisfies the following properties for each $x,y,z \in X$:
  \begin{enumerate}
    \item $\metric{x,x} = 0$
    \item $\metric{x,y} = \metric{y,x}$
    \item $\metric{x,y} \leq \metric{x,z} + \metric{z,y}$.
  \end{enumerate}
  A \textit{metric space} $X$ is a set equipped with a metric.
  \index{metric}
\end{definition}

\begin{definition}
  (Neighborhood)
  Let $X$ be a metric space, and $\delta$ be a metric on $X$.
  For $\epsilon \in \R$, and $x \in X$, a (open) \textit{neighborhood} $\oball{\epsilon}{x}$ is defined as 
  \begin{equation*}
    \oball{\epsilon}{x} \ceq \setst{y}{\metric{y,x} < \epsilon}
  \end{equation*}
  I shall use the terms ``open ball'' and ``neighborhood'' interchangeably.
\end{definition}

\begin{definition}
  (Closed neighborhood)
  Let $X$ be a metric space, and $\delta$ be a metric on $X$.
  For $\epsilon \in \R$, and $x,y \in X$, a \textit{closed neighborhood} $\cball{\epsilon}{x}$ is defined as the closure of the corresponding open neighborhood, i.e.
  \begin{equation*}
    \cball{\epsilon}{x} \ceq \cl(\oball{\epsilon}{x}).
  \end{equation*}
\end{definition}

\section{Dynamical Systems}

\begin{definition}
  (Orbit)
  Let $F: \R^n \to \R^n$. 
  The \textbf{orbit of $F$} for (initial value) $x_0$ is defined as
  \begin{equation*}
    O_F(x_0) = \setst{x}{x = F^n(x_0) \mbox{ for some } n\in \N}
  \end{equation*}
  \label{def:orbit}
  \index{definition of!orbit}
\end{definition}
$O_F$ will simply be denoted as $O$ when there is no ambiguity in the choice of $F$ in the context.

\begin{definition}
  (Periodic orbits)
  \label{def:porbit}
  \index{periodic!orbit}
\end{definition}

The usual notion of limit points are extended to limit set of an orbit.
\begin{definition}
  (Limit set)
  The \textit{limit set}, $L_F(x_0)$, of $O_F(x_0)$ is the set of limit points of $x \in O_F(x_0)$.
  \label{def:limset}
  \index{limit set}
\end{definition}

With the notion of limit sets, we may speak of aympototical periodicity of an orbit.
\begin{definition}
  (Asymptotically periodic orbit)
  An orbit $O(x_0)$ is said to be \textit{asymptotically periodic} if its limit set is a periodic orbit.
  \label{def:asymporb}
  \index{asymptotically periodic!orbit}
\end{definition}

\begin{definition}
  (Aperiodic orbit)
  The orbit $O_F(x_0)$ is said to be \textit{aperiodic} if its limit set $L_F(x_0)$ is not finite.
  \label{def:aporbit}
  \index{aperiodic!orbit}
\end{definition}

Although we have encountered the concept of iteration in the preceding chapter, I formally define the notion here.
\begin{definition}
  (Iteration)
  Let $X$ be a metric space, and $F$ a mapping.
  \textit{The $n$th iteration of F}, denoted $\itr{F}{n}$, is defined as follows
  \begin{equation*}
    \itr{F}{n} \ceq \underbrace{F \circ F \circ \cdots \circ F}_{n} .
  \end{equation*}
\end{definition}
%
It is convenient for our purpose to extend the notion of metric to a point and an orbit.
\begin{definition}
  (Distance between a point and an orbit)
  Let $X$ be a metric space, and $F$ a mapping.
  For each $x,y \in X$, the metric between $O_F(x)$ and $y$ is defined as follows
  \begin{equation*}
    \metric{O_F(x),y} \ceq \sup\limits_{x' \in O_F(x)} \metric{y,x'}
  \end{equation*}
\end{definition}

%%% Measure
\section{Measure Spaces}
\begin{definition}
  (Null Set)
\end{definition}
\begin{definition}
  (Measure)
\end{definition}

\bibliographystyle{../../pjgsm}
\bibliography{../../bibliography/thesis}

\printindex
\end{document}
