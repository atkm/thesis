\documentclass[12pt,twoside]{book}
\usepackage{../../thesis}

\makeindex
\begin{document}
\chapter{Statistical Behaviors of Chaotic Maps}
Li and Yorke says, in an analogy to the title of their famous article, that ``chaos implies statistical regularity'' \citep[Exploring Chaos on an Interval]{ueda-abraham}.

\section{Absolutely Continuous Measure}
\citet{sternberg}.
Let $I = [0,1]$, and partition $I$ into $N$ subintervals given by
\begin{align*}
  I_k &= \left[\frac{k-1}{N}, \frac{k}{N} \right),\quad (k = 1, \ldots, N-1) \\
    I_N &= \left[\frac{N-1}{N}, 1 \right].
\end{align*}

For any map $f: I \rar I$ and an initial value $x_0$, count the number
of the iterates $x_0, x_1 = f(x_0), x_2 = f(f(x_0)), \ldots, x_m = f^m(x_0)$
that lie in $I_k$ and call the number $n_k(m)$. Then define $p_k(m)$ as

\begin{equation*}
  p_k(m) := \frac{n_k(m)}{m+1},
\end{equation*}
and also let
\begin{equation*}
  p_k := \lim_{m \to \infty} p_k(m)
\end{equation*}

Roughly speaking, $p_k$ is the probability that an iterate lies in $I_k$.

\subsection{Derivation of the Perron-Frobenius Equation}
The goal of this section is to derive Perron-Frobenius equation.
\begin{equation}
  \sigma(y) = \sum\limits_{f(x) = y} \frac{\rho(x)}{|f'(x)|}
  \label{pfeqn}
\end{equation}

A measure $\mu$ is called \textbf{absolutely continuous} with respect to Lebesgue measure if a non-negative integrable function, $\rho$ satisfying the following equation
\begin{equation*}
  \mu(I) = \int_I \rho(x) dx
\end{equation*}
exists.
We refer to $\rho$ as the \textbf{density} of $\mu$. 
For any continuous function $\phi$, we define the integral of $\phi$ with respect to $\mu$ as
\begin{equation*}
  \int \phi\mu \equiv \int \phi(x)\rho(x)dx  
\end{equation*}

Now let $F_*\mu$ be the push forward measure of $\mu$ by $F$.
Suppose that $F$ is a piecewise differentiable mapping,which satisfies $|F'(x)| \neq 0$ for all $x$ except at a finite number of \textit{critical points}. 
Suppose that $A$ is an interval containing no critical values, and $F^{-1}(A)$ is the union of a finite numberof intervals, i.e. $F^{-1}(A) = \bigcup\limits_{i}J_k$, where each $J_k$ is mapped monotonically onto $A$.

In general, we have
\begin{equation*}
  \int_A g(y)dy = \int_{J_k} g(F(x))|F'(x)| dx,
\end{equation*}
where $y = F(x)$. Set
\begin{equation*}
  g(y) = \rho(x)\frac{1}{|F'(x)|}
\end{equation*}
then we get
\begin{equation*}
  \int \rho(x)\frac{1}{|F'(x)|}dy = \int_{J_k} \rho(x)dx = \mu(J_k).
\end{equation*}

Then we see that $F_*\mu$ has the density on $A$ given by
\begin{equation*}
  \sigma(y) = \sum\limits_{F(x) = y} \frac{\rho(x)}{|F'(x)|},
\end{equation*}
which is the desired result.

\section{Mixing}
\subsection{Weak Mixing}
Positive topological entropy implies weak mixing, and weak mixing implies chaos in Devaney's sense.
\subsection{Strong Mixing}

\bibliographystyle{../../bibliography/pjgsm}
\bibliography{../../bibliography/thesis}

\printindex

\end{document}
