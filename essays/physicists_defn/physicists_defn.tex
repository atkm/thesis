\documentclass[11pt]{article}
\usepackage{thesis}

\begin{document}

\section{What do physicists say?}
(Abarbanel)
In the same way power spectrum with a sharp peak characterizes a linear system,
invariants such as Lyapunov exponents, fractal dimension, and (KS or topological) entropy
characterize nonlinear systems.
Each orbit is sensitive to initial conditions; however, these invariants
are preserved under small perturbations, which are prevalent in real-life situations.

Two complementary attributes sometimes used to define chaos are
(i) exponentially sensitive dependence and
(ii) complex orbit structure.
A quantifier for attribute (i) is the largest Lyapunov exponent, while
a quantifier for attribute (ii) is the entropy (e.g, the metric entropy
or the topological entropy).
(Ott, p379)

Although there is no universally accepted definition of chaos,
most experts would concur that chaos is the {\it aperiodic,
  long-term behavior of a bounded, deterministic system
that exhibits sensitive dependence on initial conditions} (Sprott, p104)

broad (Fourier) power spectrum is a first indication of chaotic behavior,
though it alone does not characterize chaos (Abarbanel p1338)

Positive Lyapunov exponent, (Kantz)
Aperiodic and unstable orbits (claim by Martelli)

``We say that motion on an attractor is chaotic if
it displays sensitive dependence on initial conditions.'' (Ott p11)
``[\ldots] two key aspects of chaos are the stretching of infinitesimal displacements
and the complex orbit structure in the form of a vast variety of possible orbits.'' (Ott p31)

Noise reduction means that one tries to decompose a time series into
two components, one of which supposedly contains the signal and the other
one contains random fluctuations. Thus we always assume that the data
can be thought of as an additive superposition of two different components which
have to be distinguishable by some objective criterion. The classical statistical
tool for obtaining this distinction is the power spectrum. Random noise has
a flat, or at least a broad, spectrum, whereas periodic or quasi-periodic
signals have sharp spectral lines. After both components have been identified
in teh spectrum, a Wiener filter can be used to separate the time series
accordingly.
This approach fails for deterministic chaotic dynamics because the output
of such systems usually leads to broad band spectra itself and thus possesses
spectral properties generally attributed to random noise. Even if parts of the
spectrum can be clearly associated with the signal, a separation into signal and
noise fails for most parts of the frequency domain. Chaotic deterministic systems
are of particular interest because the determinism yields an alternative criterion
to distinguish the signal and the noise. (Kantz p51)

\end{document}
