\documentclass[12pt,twoside,draft]{book}
\usepackage{../../thesis}
\graphicspath{ {../../images/} }

\makeindex
\begin{document}
\chapter{Outer Billiard}
In this chapter, we illustrate an example of chaotic map, \textit{outer billiards}.
The dynamical system is a one-parameter map on $\R^2$, and for two special cases, we prove that the map is not chaotic in any sense.
For the other vast majority of cases, however, the map seems to be chaotic in some sense.
We present the result of

\section{The Billiard Map and Kite}
An \textit{outer billiard} is a dynamical system with a simple geometrical interpretation.
An outer billard is a dynamical system $(X, B_S)$, where $X$ is a set of points in $\R^2$, and $B_S$ is a \textit{billiard map} determined by a \textit{shape} $S$.
(Make figures)

This dynamical system was popularized by Jürgen Moser as a toy model of celestial mechanics~\citep{moser,moserbook}.
We can interpret the shape as the Sun, and the points as planets orbiting the Sun.
The Moser-Neumann question asks if there is an outer billiard system that has a point whose orbit is unbounded.
In an analogy to the Solar system, an unbounded orbit correspond to a planet escaping the system.
\citet{moser} says that the Moser-Neumann question is
``a very simple geometrical problem that actually contains some of the difficulties of the $n$ body problem and may serve as a crude model for planetary motion.''
Regardless of the legitimacy of his claim, the Moser-Neumann question is an interesting mathematical problem with an intriguing geometrical nature (See Figure).
%a tried and true technique of mathematics to extract the essential properties of a problem and to idealize it
\citet{moserbook} showed that an outer billiard on $S$ (the shape) does not have an unbounded orbit provided that $S$ is at least $C^6$ smooth and has positive curvature. 
\citet{schwartz} studies outer billiards when the shape is a Penrose kite.
Other known results are listed in \citet[p. 2]{schwartz}.

In this chapter, we study a family of one-parameter shapes, which we will refer to as \textit{kites}.
\footnote{The particular shape was suggested by Ray Mayer. Reed College, Portland, OR !ask for permission}
\begin{definition}
  (Kite $S_d$)
  Let $S_d$ be 
  We construct $S_d$, a shape with one parameter $d$, as follows.
  Let $C_1$ be an arc for $\theta \in (\pi/4, 3\pi/4)$ of a circle of radius $1+d$ centered at $(0,-d)$.
  Rotate $C_1$ by $\pi/4$ around the origin to obtain $C_2$.
  Similarly obtain $C_3$ and $C_4$ by rotations by $\pi/4$.
  Then, let $S_d = C_1 \cup C_2 \cup C_3 \cup C_4$ (See Figure).

  We define $B_{S_d}$, the corresponding billiard map as follows.
  We refer to $S_d$ as a \textit{kite}.
\end{definition}
\begin{figure}[ht]
  \begin{center}
    \includegraphics[width=0.8\textwidth]{kite_d10}
  \caption{An example of a kite. The kite shown here is $S_d$ for $d = 1.0$.}
  \label{fig:kiteeg}
  \end{center}
\end{figure}

\section{Special Case (1): $d = 0$}
When $d = 0$, the kite is a unit circle (Figure~\ref{fig:kite-circle}).

\begin{figure}[ht]
  \begin{center}
    \includegraphics[width=0.8\textwidth]{kite_d0}
  \caption{$d = 0$. The kite is a unit circle.}
  \label{fig:kite-circle}
  \end{center}
\end{figure}

We only consider the dynamics of the points that are outside the kite, since the mapping of points in or on the kite is defined to be the identity map.
Consider a point $(r, \theta)$ with $r > 1$.
We show that $r$ determines the dynamics of the point.
Note that, for each $R \in \R$, the circle $\setst{(r,\theta)}{r = R}$ is an invariant set of the billiard map.

\section{Special Case (2): $d \to \infty$}
As $d \to \infty$, the kite approaches a square (Figure~\ref{fig:kite-square}).

\begin{figure}[ht]
  \begin{center}
    \includegraphics[width=0.8\textwidth]{kite_d100}
  \caption{$d = 100$. When $d$ is large, the kite is approximately a square.}
  \label{fig:kite-square}
  \end{center}
\end{figure}


\section{Other Cases: $d \neq 0$, $d < \infty$}
So far, we solved two special cases that 
The orbits of any points in a bounded set seem to be bounded.
But we have no proof for this.
 
\bibliographystyle{../../bibliography/pjgsm}
\bibliography{../../bibliography/thesis}

\printindex
\end{document}

