\documentclass[12pt,twoside,draft]{book}
\usepackage{../../thesis}
\graphicspath{ {../../images/} }

\makeindex
\begin{document}
\chapter{Outer Billard}
In this chapter, we illustrate an example of chaotic map.
The map is a one-parameter map, and for special cases, we prove that the map is not chaotic in any sense.
For vast majority of cases, however, the map seems to be chaotic in some sense.

\section{Historical Remarks}
Moser (1973):
Outer Billard on $K$ (the shape) has all bounded orbits provided that $\partial K$ is at least $C^6$ smooth and has positive curvature. (Schwartz p.2)

Vivaldi-Shaidenko

Thus, the shape determines the dynamics of the system.
The one-parameter 
We refer to the parametrized shape as a \textit{kite}.
The particular shape is suggested by Ray Mayer.\footnote{Reed College, Portland, OR}

\section{Special Case 1: $d = 0$}
When $d = 0$, the kite is a unit circle.

\section{Special Case 2: $d \to \infty$}
 
\bibliographystyle{../../bibliography/pjgsm}
\bibliography{../../bibliography/thesis}

\printindex
\end{document}

