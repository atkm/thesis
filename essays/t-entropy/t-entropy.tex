\documentclass[12pt,twoside,draft]{book}
\usepackage{../../thesis}
\graphicspath{ {../../images/} }

\makeindex
\begin{document}
\chapter{Topological Entropy}
The notion of topological entropy was first introdeced by \citet{akm}.
Topological entropy mimics the notion of entropy employed in statistical mechanics.
We first discuss basic properties of topological entropy, then compute the topological entropy of the logistic map.


\section{Definition and properties of topological entropy}
Throughout this section, $(X,d)$ is a compact metric space.
$F$ is a continuous mapping from $X$ to $X$ unless noted otherwise.
Our definition of topological entropy uses open covers of a space. 

\begin{definition}
  (Join of Open Covers)
  Let $\alpha$ and $\beta$ be open covers of $X$.
  Their join, denoted $\alpha \vee \beta$, is defined as
  \begin{equation*}
    \alpha \vee \beta \ceq \setst{A \cap B}{A \in \alpha, B \in \beta}.
  \end{equation*}
  We define the join of a countable collection of open covers in the same manner, and denote it as $\bigvee\limits_{i = 1}^{n} \alpha_i$.
\end{definition}

\begin{definition}
  (Refinement of an Open Cover)
  An open cover $\beta$ is said to be a refinement of another open cover $\alpha$, if for each $A \in \alpha$, there exists $B \in \beta$ such that $B \subseteq A$.
  Note that $\alpha < \alpha \vee \beta$.
\end{definition}

\begin{definition}
  (Preimage of an Open Cover)
  Let $\alpha$ be an open cover of $X$, and $F: X \to X$ be a continuous mapping.
  Then $F^{-1}(\alpha)$ is defined as the following:
  \begin{equation*}
    F^{-1}(\alpha) \ceq \setst{F^{-1}(A)}{A \in \alpha}.
  \end{equation*}
\end{definition}
\begin{proposition}
  Let $\alpha$ and $\beta$ be open covers of $X$.
  We have
  \begin{enumerate}[(i)]
    \item $F^{-1}(\alpha \vee  \beta) = F^{-1}(\alpha) \vee F^{-1}(\beta)$
    \item $\alpha < \beta \Rar F^{-1}(\alpha) < F^{-1}(\beta)$.
  \end{enumerate}
  \begin{proof}
    (i) The statement follows from the fact that $F^{-1}(A \cap B) = F^{-1}(A) \cap F^{-1}(B)$. \\
    (ii) If $A \subseteq B$ then $F^{-1}(A) \subseteq F^{-1}(B)$.
  \end{proof}
\end{proposition}

\begin{definition}
  (Topological Entropy)
  The topological entropy of an open cover $\alpha$ is defined as
  \begin{equation*}
    h(\alpha) = \log N(\alpha),
  \end{equation*}
  where $N(\alpha)$ is the minimum of cardinalities of subcovers of $\alpha$, i.e.
  \begin{equation*}
    N(\alpha) \ceq \min\setst{ \#(\beta) }{ \beta \mbox{ is a subcover of } \alpha}.
  \end{equation*}
  The topological entropy of a mapping $F$ relative to a open cover $\alpha$ is defined as (prove the existence of this limit)
  \begin{equation*}
    h(F,\alpha) = \lim\limits_{n \to \infty} \frac{1}{n} \cdot h\paren{ \bigvee\limits_{i=0}^n \itr{F}{(-i)}(\alpha) }.
  \end{equation*}
  Finally, the \textit{topological entropy} of a mapping $F$, denoted $h(F)$, is defined as the supremum of $h(F, \alpha)$ taken over all open covers of $X$:
  \begin{equation*}
    h(F) = \sup\limits_\alpha h(F,\alpha).
  \end{equation*}
  \label{defn:t-entropy}
  \index{topological entropy}
\end{definition}

As we claimed at the beginning of this chapter, the topological entropy is invarant under topological conjugacy.
The precise statement is the following theorem.
\begin{theorem}
  (Topological Entropy is an Invariant of Topological Conjugacy)
  (Misiurewicz)
  Let $X_1$ and $X_2$ be metric spaces, and $F_1: X_1 \to X_1$ and $F_2: X_2 \to X_2$ are continuous mappings.
  Suppose $f\circ \phi = \phi \circ g$.
  If $\phi$ is injective, then $h(f) \geq h(g)$.
  If $\phi$ is surjective, then $h(f) \leq h(g)$.
  It follows that, if $\phi$ is a homeomorphism, then $h(f) = h(g)$.
  If $\phi: X_1 \to X_2$ is a homeomorphism, then $h(F_1) = h(F_2)$.
  \label{thm:t-ent-conj}
\end{theorem}
We require a lemma in order to prove the theorem.

\begin{lemma}
  Let $F: X\to X$ be a continuous mapping.
  Then $h(F^{-1}(\alpha)) \leq h(\alpha)$.
  Furthermore, if $F$ is also surjective, then $h(F^{-1}(\alpha)) = h(\alpha)$.
  \begin{proof}
    Suppose $n = N(\alpha)$, and $\set{A_1, \ldots, A_{n}}$ is a subcover of $\alpha$ of minimal cardinality.
    Then, $\set{F^{-1}(A_1), \ldots, F^{-1}(A_{n})}$ is a subcover of $F^{-1}(\alpha)$.
    This proves the first statement.
    To prove the second statement, suppose $m = N(F^{-1}(\alpha))$.
    Let $\set{F^{-1}(A_1), \ldots, F^{-1}(A_{m})}$ be a subcover of $F^{-1}(\alpha)$ of minimal cardinality.
    If $F$ is surjective, then $\set{A_1, \ldots, A_{m}}$ is a subcover of $\alpha$.
    Hence $h(F^{-1}(\alpha)) \geq h(\alpha)$.
  \end{proof}

\end{lemma}

Now we proceed to prove the theorem.
\begin{proof}[Proof of Theorem~\ref{thm:t-ent-conj}]
  Let $\alpha$ be an open cover of $X_2$.
  We have
  \begin{align*}
    h(F_2, \alpha)
    &= \lim\limits_{n \to \infty} \frac{1}{n} \cdot h \paren{\bigvee\limits_{i = 0}^{n-1} \itr{F_2}{(-i)}(\alpha)} \\
    &= \lim\limits_{n \to \infty} \frac{1}{n} \cdot h \paren{\bigvee\limits_{i = 0}^{n-1} \phi^{-1} \circ \itr{F_1}{(-i)} \circ \phi (\alpha)} \\
    &= \lim\limits_{n \to \infty} \frac{1}{n} \cdot h \paren{\phi^{-1} \bigvee\limits_{i = 0}^{n-1} \itr{F_1}{(-i)} (\phi (\alpha))} \\
    &= \lim\limits_{n \to \infty} \frac{1}{n} \cdot h \paren{\bigvee\limits_{i = 0}^{n-1} \itr{F_1}{(-i)} (\phi (\alpha))} \mbox{ (by the lemma)} \\
    &= h(F_1, \phi(\alpha)).
  \end{align*}
  %
  As $\alpha$ ranges over all open covers of $X_1$, $\phi(\alpha)$also ranges over all open covers of $X_2$, since $\phi$ is a homeomorphism.
  Therefore,
  \begin{equation*}
    h(F_2) = h(F_1).
  \end{equation*}
\end{proof}

The definition of topological entropy calls for computation of $h(F,\alpha)$ for all possible open covers of $X$, and thus, computing $h(F)$ using the definition is untenable.
As a reasonable method for computing $h(F)$, we introduce a theorem, which states that $h(F)$ equals to the limit of the topological entropy of $F$ relative to a sequence of open covers that become finer and finer later in the sequence.
We call such a sequence a \textit{refining sequence}, defined as follows.

\begin{definition}
  (Refining Sequence)
  A sequence $\set{\alpha_n}$ of open covers of a compact space $X$ is called a \textit{refining sequence}, if it satisfies the following two criteria:
  \begin{enumerate}[(i)]
    \item $\alpha_n < \alpha_{n+1}$ for each $n\geq 0$
    \item if $\beta$ is an open cover of $X$, then there exists $m$ such that $\beta < \alpha_m$.
  \end{enumerate}
\end{definition}

\begin{theorem}
  Suppose $\set{\alpha_n}$ is a refining sequence of $X$, and $F:X\to X$ is a continuous mapping.
  Then,
  \begin{equation*}
    h(F) = \lim\limits_{n\to \infty} h(F, \alpha_n).
  \end{equation*}
  \label{thm:t-ent-ref-seq}
  %
  \begin{proof}
    Since $h(F) = \sup\limits_{\alpha} h(F, \alpha)$, the inequality
    \begin{equation*}
      h(F) \geq \lim\limits_{n\to \infty} h(F, \alpha_n).
    \end{equation*}
    holds.
    To show that the opposite inequality also holds, note that for each open cover $\beta$ of $X$, there exists $k_0$ such that, for each $k > k_0$, $\beta < \alpha_k$.
    It follows that
    \begin{equation*}
      h(F, \beta) \leq h(F, \alpha_k).
    \end{equation*}
    Take the supremum over $\beta$ of both sides to obtain
    \begin{equation*}
      \sup\limits_{\beta} h(F, \beta) = h(F) \leq h(F, \alpha_k).
    \end{equation*}
    Finally, take the limit $k \to \infty$ to get
    \begin{equation*}
      h(F) \leq \lim\limits_{k \to \infty} h(F, \alpha_k).
    \end{equation*}
    Hence, $h(F) = h(F, \alpha_k)$.
  \end{proof}
\end{theorem}

We will use Lebesgue's Covering Lemma in the next section to show that some sequence of open covers satisfies the second condition of the definition of refining sequence.
\begin{lemma}
  (Lebesgue's Covering Lemma)
  Let $X$ be a compact metric space, and $\alpha$ be an open cover of $X$.
  There exists $\delta > 0$ such that, if $B$ is a subset of $X$ whose diameter is less than or equal to $\delta$, i.e. $\diam(B) \leq \delta$, then there exists a member $A$ of $\alpha$ such that $B \subseteq A$.
  \label{lem:covering}
  \index{Lebesgue's covering lemma}
  \begin{proof}
    We omit the proof, since this is a classical result.
    For example, see \citet{royden}.
    %  Proof by contradiction.
    %  Consider a sequence $\set{B_n}$ of subsets of $X$ such that $\diam(B_n) = 1/n$, and each $B_n$ is not contained in any member of $\alpha$.
    %  For each $B_n$, fix $x_n \in B_n$.
    %  Since $X$ is compact, there exists a subsequence of $\set{x_n}$ that converges.
    %  Call this subsequence $\set{x_{n_k}}$, and suppose it converges to $x$.
    %  Also, suppose $x$ is a member of $A \in \alpha$, and let $D = \diam(A)$.
    %  Choose $l$ large enough so that $n_l > \frac{2}{D}$ and $\metric{x_{n_l}, x} < \frac{D}{2}$.
    %  Finally, for any point $y$ in $B_{n_l}$, we have
    %  \begin{equation*}
    %    \metric{y, x_{n_l}}
    %    \leq \metric{y,x} + \metric{x, x_{n_l}}
    %    < \diam(B_{n_l})  + \frac{D}{2}
    %    = \frac{1}{n_l} + \frac{D}{2}
    %    = D.
    %  \end{equation*}
    %  This is a contradiction, since $\metric{y, x_{n_l}} < D$ implies that $B_{n_l}$ is contained in $A$.
  \end{proof}
\end{lemma}

\section{The Topological Entropy of The Logistic Map}
\begin{definition}
  (Definition of chaos by topological entropy: chaos in the sense of AKM) 
  Let $F: X \to X$ be a continuous mapping.
  If $h(F)$ is positive, then we say that $F$ is \textit{chaotic in the sense of AKM}.
  (\citet{akm} does not use the term ``chaos,'' but we attribute this definition to the authors.)
\end{definition}
We saw in the introduction that $L_u$, the logistic map, which is defined as
\begin{align*}
  L_\mu&: I \to I \quad\mbox{ where } I \equiv [0,1] \\
  L_\mu&: x \mapsto \mu x (1-x),
\end{align*}
is chaotic when $\mu = 4$.
As a demonstration of topological entropy, we will compute $h(L_\mu)$ for $\mu = 4$.
We expect that $h(L_4)$ is positive, and as we will see, this is indeed the case. 
We first show that $L_4$ is topologically conjugate to the tent map, defined as
\begin{equation*}
  T(x) = 
  \begin{cases}
    2x \mbox{ if } x \in [0,1/2] \\
    2 - 2x \mbox{ if } x \in [1/2,1].
  \end{cases}
\end{equation*}
Once we establish this conjugacy, by Theorem~\ref{thm:t-ent-conj}, the problem reduces to computation of $h(T)$.
Let
\begin{equation*}
  \phi(x) = \sin^2(\frac{\pi x}{2}).
\end{equation*}
Our goal is to show that $L_4 \circ \phi = \phi \circ T$.
For any $x \in I$, letting $\theta \equiv \frac{\pi x}{2}$ for convenience, we have
\begin{align*}
   L_4 \circ \phi(x)
   &= 4\sin^2\theta(1 - \sin^2\theta) \\
   &= 4\sin^2\theta \cos^2\theta \\
   &= \sin^2 2\theta \\
   &= \sin^2 (\pi x).
\end{align*}
For any $x$ in $[0,1/2]$,
\begin{equation*}
  \phi \circ T(x)
  = \sin^2(\pi x),
\end{equation*}
and for any $x$ in $[1/2,1]$,
\begin{align*}
  \phi \circ T(x)
  &= \sin^2(\pi - \pi x) \\
  &= \sin^2(\pi x).
\end{align*}
Therefore, $L_4 \circ \phi = \phi \circ T$ on $I$, so that, by Theorem~\ref{thm:t-ent-conj}, we obtain $h(L_4) = h(T)$.
Our next task is to construct a refining sequence $\set{\alpha_n}$ of $I$.
Then, we would obtain $h(T)$ by Theorem~\ref{thm:t-ent-ref-seq}.
Our construction is as follows.
First, let $\alpha_0 = \set{I}$.
Then, for $n \geq 1$, recursively define $\alpha_n$ as
\begin{equation*}
  \alpha_n = T^{-1}(\alpha_{n-1}).
\end{equation*}
%
This recursive definition can be stated more explicitly as
\begin{align*}
  \alpha_n &= T^{-1}(\alpha_{n-1}) \\
  &= \bigcup\limits_{A \in \alpha_{n-1}} \paren{ \setst{J}{T(J) = I, J \subseteq [0,1/2]} \cup \setst{J}{T(J) = I, J \subseteq [1/2,1]}}.
\end{align*}
The expression is motivated by the fact that the tent map is a two-to-one mapping, except for when $x = \frac{1}{2}$.
For each $x \in I - \set{\frac{1}{2}}$, the preimage of $x$ is given by the formula
\begin{equation*}
  T^{-1}(x) = \set{\frac{1}{2} x, 1 - \frac{1}{2} x}.
\end{equation*}
Notice that if $\frac{1}{2} x \in [0,1/2)$, then $1 - \frac{1}{2} x \in (1/2,1]$; if $\frac{1}{2} x \in (1/2,1]$, then $1 - \frac{1}{2} x \in [0,1/2)$.
This observation holds for the preimage of an interval.
For example,
\begin{align*}
  \alpha_1 &= \set{\brac{0,\frac{1}{2}}, \brac{\frac{1}{2}, 1}}, \\
  \alpha_2 &= \set{\brac{0,\frac{1}{4}}, \brac{\frac{3}{4}, 1}, \brac{\frac{1}{4}, \frac{1}{2}}, \brac{\frac{1}{2}, \frac{3}{4}}},
\end{align*}
and so on.
Thus, $N(\alpha_{n}) = 2^n$.
Now, we want to show that $\set{\alpha_n}$ is a refining sequence.
$\set{\alpha_n}$ clearly satisfies the first condition:
\begin{equation*}
  \alpha_n < \alpha_{n+1}.
\end{equation*}
To show the other condition, that for each open cover $\beta$ of $X$, there exists $k_0$ such that $\beta < \alpha_k$ for each $k \geq k_0$, we invoke Lebesgue's covering lemma.
Let $\delta > 0$ be the constant given by applying the lemma to $\beta$.
For each $J_n \in \alpha_n$, we have $\diam(J_n) = 1/{2^n}$.
Choose $n$ large enough so that $\diam(J_n) \leq \delta$.
Then, for each $A \in \alpha_n$, the lemma guarantees the existence of $B \in \beta$ such that $A \subseteq B$.
This implies $\beta < \alpha_n$.
We conclude that $\set{\alpha_n}$ is a refining sequence.
Finally, we compute $\lim\limits_{n \to \infty} \frac{1}{n} h(T, \alpha_n)$.
Recall that $N(\alpha_{n}) = 2^n$.
Then,
\begin{align*}
  h(T, \alpha_n)
  &= \lim\limits_{m \to \infty} \frac{1}{m} \log N\paren{ \bigvee\limits_{i = 0}^{m-1} T^{-i}(\alpha_n)}  \\
  &= \lim\limits_{m \to \infty} \frac{1}{m} \log N(\alpha_{n+m-1})  \\
  &= \lim\limits_{m \to \infty} \frac{1}{m} \log 2^{n + m - 2}  \\
  &= \lim\limits_{m \to \infty} \frac{n + m - 2}{m} \log 2 \\
  &= \log 2.
\end{align*}
Therefore,
\begin{equation*}
  h(L_4) = h(T) = \lim\limits_{n \to \infty} h(T, \alpha_n) = \log 2.
\end{equation*}

\bibliographystyle{../../bibliography/pjgsm}
\bibliography{../../bibliography/thesis}

\printindex
\end{document}

