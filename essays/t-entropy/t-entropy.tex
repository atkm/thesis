\documentclass[10pt,twoside,draft]{book}
\usepackage{../../thesis}
\graphicspath{ {../../images/} }

\makeindex
\begin{document}
\chapter{Topological Entropy}
\label{chap:entropy}
In the current chapter, we introduce another widely accepted notion of chaos that is defined using the topological equivalent of entropy.
The notion of topological entropy was first introduced by \citet{akm} as a measure of complexity of a topological dynamical system, mimicking the measure-theoretic notion of entropy.
Roughly, topological entropy measures the extent to which the mapping scatters points around the space.
%exponential growth rate of the number of distinguishable orbits as time advances. (Scholarpedia)
%``The topological entropy is indeed a measure of the absolute information content of a transformation, or, equivalently, of the extent to which the transformation scatters points around the space $X$.'' (Petersen)
We first discuss basic properties of topological entropy, then compute the topological entropy of the logistic map for $\mu = 4$.
Throughout this section, $(X,d)$ is a compact metric space.
$\log$ is the logarithm base $e$ (the choice of base is arbitrary).

\section{Definition and Properties}
The topological entropy of a system is defined using open covers of a space. 
The material in this section is from \citep{akm}, unless stated otherwise.
\begin{definition}
  (Join of open covers)
  Let $\alpha$ and $\beta$ be open covers of $X$.
  Their join, denoted $\alpha \vee \beta$, is defined as
  \begin{equation*}
    \alpha \vee \beta \ceq \setst{A \cap B}{A \in \alpha, B \in \beta}.
  \end{equation*}
  We define the join of a countable collection of open covers in the same manner, and denote it $\bigvee\limits_{i = 1}^{n} \alpha_i$.
\end{definition}
%%%
\begin{definition}
  (Refinement of an open cover)
  An open cover $\beta$ is said to be a refinement of another open cover $\alpha$, if for each $B \in \beta$, there exists $A \in \alpha$ such that $B \subseteq A$.
  Note that for any open covers $\alpha, \beta$, we have $\alpha < \alpha \vee \beta$.
\end{definition}

\begin{definition}
  (Preimage of an Open Cover)
  Let $\alpha$ be an open cover of $X$, and $F: X \to X$ be a continuous mapping.
  Then $F^{-1}(\alpha)$ is defined as the following:
  \begin{equation*}
    F^{-1}(\alpha) \ceq \setst{F^{-1}(A)}{A \in \alpha}.
  \end{equation*}
\end{definition}
\begin{proposition}
  (Properties of open covers)
  Let $\alpha$ and $\beta$ be open covers of $X$, and $F: X \to X$ be a continuous mapping.
  We have
  \begin{enumerate}[(i)]
    \item $F^{-1}(\alpha \vee  \beta) = F^{-1}(\alpha) \vee F^{-1}(\beta)$
    \item $\alpha < \beta \Rar F^{-1}(\alpha) < F^{-1}(\beta)$.
  \end{enumerate}
  \begin{proof}
  \begin{enumerate}[(i)]
    \item 
      The statement follows from the fact that $F^{-1}(A \cap B) = F^{-1}(A) \cap F^{-1}(B)$ for each $A \in \alpha$ and $B \in \beta$.
      %%%
    \item Let $A \in \alpha$ and $B \in \beta$. 
      If $A \subseteq B$ then $F^{-1}(A) \subseteq F^{-1}(B)$.
      %%%
  \end{enumerate}
  \end{proof}
\end{proposition}

\begin{definition}
  (Topological Entropy)
  The topological entropy of an open cover $\alpha$ is defined as
  \begin{equation*}
    h(\alpha) \ceq \log N(\alpha),
  \end{equation*}
  where $N(\alpha)$ is the minimum of cardinalities of subcovers of $\alpha$, i.e.
  \begin{equation*}
    N(\alpha) \ceq \min\setst{ \#(\beta) }{ \beta \mbox{ is a subcover of } \alpha}.
  \end{equation*}
  We refer to a subcover of $\alpha$ with cardinality $N(\alpha)$ as a \textit{minimal subcover}.
  The topological entropy of a mapping $F$ relative to an open cover $\alpha$ is defined as
  \begin{equation*}
    h(F,\alpha) \ceq \lim\limits_{n \to \infty} \frac{1}{n} \cdot h\paren{ \bigvee\limits_{i=0}^{n-1} \itr{F}{(-i)}(\alpha) }.
  \end{equation*}
  For the existence of this limit, see Proposition~\ref{prop:entropy-limit}.
  Finally, the \textit{topological entropy} of a mapping $F$, denoted $h(F)$, is defined as the supremum of $h(F, \alpha)$ taken over all open covers of $X$:
  \begin{equation*}
    h(F) \ceq \sup\limits_\alpha h(F,\alpha).
  \end{equation*}
  We use $h$ to denote three things: the topological entropy of an open cover, of an open cover with respect to a mapping, and of a mapping.
  However, since the meaning of $h$ changes according to its argument(s), there should be no ambiguity.
  \label{defn:t-entropy}
  \index{topological entropy}
  \index{minimal subcover}
\end{definition}
%%%
We have defined $h(F)$ only for mappings from a compact metric space to itself, but it is possible to extend the definition to any metric space \citep{bowen}.
See \citep{walters} for an exposition of this alternate definition of topological entropy.
Now, we state the definition of chaos that employs topological entropy.
\begin{definition}
  (Definition of chaos by topological entropy: chaos in the sense of AKM) 
  Let $F: X \to X$ be a continuous mapping.
  If $h(F)$ is positive, then we say that $F$ is \textit{chaotic in the sense of AKM}.
  (\citet{akm} does not use the term ``chaos,'' but we attribute this definition to the authors.)
  \index{definition of chaos!topological entropy}
\end{definition}
%%%

In order to prove the existence of the limit in the definition of $h(F, \alpha)$, we need to use a few properties of $h(\alpha)$.
We first prove some properties of $N(\alpha)$.
\begin{proposition}
  (Properties of $N(\alpha)$)
  Let $F: X \to X$ be continuous, $\alpha$ and $\beta$ be open covers of $X$.
  \begin{enumerate}[(i)]
    \item 
      $N(\itr{F}{-1}(\alpha)) \leq N(\alpha)$.
      The equality holds if $F$ is surjective.
      %%%
    \item 
      If $\alpha < \beta$, then $N(\alpha) \leq N(\beta)$.
      %%%
    \item 
      $N(\alpha \vee \beta) \leq N(\alpha) \cdot N(\beta)$.
      %%%
    \item 
      If $\alpha < \beta$, then $N(\alpha \vee \beta) = N(\beta)$.
    %\item $h(\itr{F}{k}) = k h(F)$.
  \end{enumerate}
  \begin{proof}
    \begin{enumerate}[(i)]
      \item 
        Let $\beta$ be a minimal subcover of $\alpha$.
        By definition, the cardinality of $\beta$ is $N(\alpha)$.
        $\itr{F}{-1}(\beta)$ is a subcover of $\itr{F}{-1}(\alpha)$, and its cardinality is also $N(\alpha)$.
        However, it is possible that there exists a different subcover of $\itr{F}{-1}(\alpha)$ whose cardinality is less than $N(\alpha)$.
        Hence, 
        \begin{equation*}
          N(\itr{F}{-1}(\alpha)) \leq N(\alpha).
        \end{equation*}
        Now, suppose that $F$ is surjective.
        We may assume that for some open cover $\beta$, $\itr{F}{-1}(\beta)$ is a minimal subcover of $\itr{F}{-1}(\alpha)$.
        Then, $F(\itr{F}{-1}(\beta)) = \beta$ is a subcover of $\alpha$.
        By the same argument as before, we obtain $N(\itr{F}{-1}(\alpha)) \geq N(\alpha)$.
        %%%
      \item
        Let $\beta_0$ be a minimal subcover of $\beta$.
        Since $\alpha < \beta$, $\beta_0$ is also a subcover of $\alpha$.
        Hence, $N(\alpha) \leq N(\beta)$.
        %%%
      \item
        Let $\alpha_0$ and $\beta_0$ be minimal subcovers of $\alpha$ and $\beta$, respectively.
        Then,
        \begin{equation*}
          \gamma \ceq \setst{A_i \cap B_j}{i = 1, \ldots, N(\alpha), j = 1, \ldots, N(\beta)}
        \end{equation*}
        is a subcover of $\alpha \vee \beta$.
        Since the cardinality of $\gamma$ is $N(\alpha) \cdot N(\beta)$, we obtain
        \begin{equation*}
          N(\alpha \vee \beta) \leq N(\alpha) \cdot N(\beta).
        \end{equation*}
      %%%
      \item
        In general, we have $\beta < \alpha \vee \beta$, so $N(\beta) \leq N(\alpha \vee \beta)$.
        Since $\alpha < \beta$, we also have $\alpha \vee \beta < \beta$.
        It follows that $N(\beta) \geq N(\alpha \vee \beta)$.
      %%%
    \end{enumerate}
  \end{proof}
\end{proposition}
%%%
Properties of $N(\alpha)$ propagate to $h(\alpha)$, which is defined as $h(\alpha) \equiv \log N(\alpha)$.
The following corollary lists the properties of $h(\alpha)$ that we immediately obtain from the preceding proposition.
\begin{corollary}
  (Properties of $h(\alpha)$)
  Let $F: X \to X$ be continuous, $\alpha$ and $\beta$ be open covers of $X$.
  \begin{enumerate}[(i)]
    \item 
      $h(\itr{F}{-1}(\alpha)) \leq h(\alpha)$.
      The equality holds if $F$ is surjective.
    \item 
      If $\alpha < \beta$, then $h(\alpha) \leq h(\beta)$.
    \item
      $h(\alpha \vee \beta) \leq h(\alpha) + h(\beta)$.
    \item
      If $\alpha < \beta$, then $h(\alpha \vee \beta) = h(\beta)$.
  \end{enumerate}
  \label{cor:ha-props}
\end{corollary}
%%%
We need one more lemma, and we are ready to prove the existence of the limit.
\begin{lemma}
  \citep[Theorem 4.9]{walters}
  Let $\seq{a_n}{\infty}{n = 0}$ be a sequence of real numbers such that 
  \begin{equation*}
    a_{m+n} \leq a_n + a_m \mbox{ for all } m,n.
  \end{equation*}
  Then, the limit
  \begin{equation*}
    \lim\limits_{n \to \infty} \frac{a_n}{n} 
  \end{equation*}
  exists, and its value is $\inf a_n / n$.
  \begin{proof}
    For any $m$, each $n > 0$ can be written as $n = km + r$ for some non-negative integers $k,r$ such that $r < m$.
    By hypothesis, we have 
    \begin{equation*}
      \frac{a_n}{n}
      = \frac{a_{km + r}}{km + r}
      \leq \frac{a_{km + r}}{km}
      \leq \frac{a_{km} + a_r}{km}
      \leq \frac{k a_{m}}{km} + \frac{a_r}{km}
      = \frac{a_{m}}{m} + \frac{a_r}{km}.
    \end{equation*}
    As $n \to \infty$, $k \to \infty$ so that
    \begin{equation*}
      \limsup\limits_{n \to \infty} \frac{a_n}{n} \leq \frac{a_m}{m}.
    \end{equation*}
    Hence,
    \begin{equation*}
      \limsup\limits_{n \to \infty} \frac{a_n}{n} \leq \inf\limits_m \frac{a_m}{m}.
    \end{equation*}
    We also have
    \begin{equation*}
      \inf\limits_{m \to \infty} \frac{a_m}{m} \leq \liminf\limits_{n \to \infty} \frac{a_n}{n}
    \end{equation*}
    by definition of liminf.
    Therefore the limit $\lim\limits_{n \to \infty} a_n/n$ exists, and equals $\inf a_n/n$.
  \end{proof}
\end{lemma}
%%%
\begin{proposition}
  \begin{equation*}
    h(F,\alpha) \equiv \lim\limits_{n \to \infty} \frac{1}{n} \cdot h\paren{ \bigvee\limits_{i=0}^{n-1} \itr{F}{(-i)}(\alpha) }
  \end{equation*}
  exists.
  \label{prop:entropy-limit}
  \begin{proof}
    Denote $k_n \ceq h(\alpha \vee \cdots \vee \itr{F}{-n+1})$.
    For any positive integers $n$ and $m$, we have
    \begin{align*}
      k_{m+n} &\equiv h(\alpha \vee \cdots \vee \itr{F}{-m-n+1}(\alpha))   \\
      &= h(\alpha \vee \itr{F}{-m+1}(\alpha) \vee \itr{F}{-m}(\itr{F}{-1}(\alpha) \cdots \vee \itr{F}{-n+1}(\alpha)))   \\
      &\leq  h(\alpha \vee \itr{F}{-m+1}(\alpha)) + h(\itr{F}{-m}(\itr{F}{-1}(\alpha) \cdots \vee \itr{F}{-n+1}(\alpha)))   \\
      &\leq  h(\alpha \vee \itr{F}{-m+1}(\alpha)) + h(\itr{F}{-1}(\alpha) \cdots \vee \itr{F}{-n+1}(\alpha)))   \\
      &\equiv k_m + k_n.
    \end{align*}
    Thus, $k_{m+n} \leq k_m + k_n$ for all $m,n$, and it follows from the preceding lemma that the limit exists.
  \end{proof}
\end{proposition}
%%%
$h(F, \alpha)$ has the following properties.
\begin{proposition}
  (Properties of $h(F, \alpha)$)
  Let $F: X \to X$ be continuous, $\alpha$ and $\beta$ be open covers of $X$.
  \begin{enumerate}[(i)]
    \item If $\alpha < \beta$, then $h(F, \alpha) \leq h(F, \beta)$.
    %%%
    \item $h(F, \alpha) \leq h(\alpha)$.
    %%%
  \end{enumerate}
  \label{prop:hfa-props}
  \begin{proof}
   \begin{enumerate}[(i)]
    \item 
      $\alpha < \beta$ implies
      \begin{equation*}
        \bigvee\limits_{i=0}^{n-1} \itr{F}{-i}(\alpha)
        <
        \bigvee\limits_{i=0}^{n-1} \itr{F}{-i}(\beta).
      \end{equation*}
      Then, by Corollary~\ref{cor:ha-props}(ii),
      \begin{equation*}
        h\paren{ \bigvee\limits_{i=0}^{n-1} \itr{F}{-i}(\alpha) }
        \leq
        h\paren{ \bigvee\limits_{i=0}^{n-1} \itr{F}{-i}(\beta) }.
      \end{equation*}
      Divide both sides of the last equation by $n$ and take $n \to \infty$.
      We obtain $h(F, \alpha) \leq h(F, \beta)$.
    %%%
    \item 
      We have
      \begin{align*}
        h\paren{ \bigvee\limits_{i=0}^{n-1} \itr{F}{-i}(\alpha) }
        &\leq \sum\limits_{i=0}^{n-1} h(\itr{F}{-i}(\alpha)) 
        \mbox{ (by Corollary~\ref{cor:ha-props}(iii))} \\
        &\leq \sum\limits_{i=0}^{n-1} h(\alpha) 
        \mbox{ (by Corollary~\ref{cor:ha-props}(i))}  \\
        &= n h(\alpha).
      \end{align*}
      Then, note that
      \begin{equation*}
        h\paren{ \bigvee\limits_{i=0}^{n-1} \itr{F}{-i}(\alpha) }
        \leq n h(\alpha)
        \Rar 
        \frac{ h\paren{ \bigvee\limits_{i=0}^{n-1} \itr{F}{-i}(\alpha) } }{n}
        \leq h(\alpha).
      \end{equation*}
      By taking the limit $n \to \infty$, we obtain the desired result.
    %%%
  \end{enumerate}
  \end{proof}
\end{proposition}

As we claimed at the beginning of this chapter, the topological entropy is a topological invariant.
\begin{theorem}
  (Topological Entropy is an Invariant of Topological Conjugacy)
  Let $X$ and $Y$ be metric spaces, and $F: X \to X$ and $G: Y \to Y$ be continuous mappings.
  Suppose $G\circ \phi = \phi \circ F$.
  %If $\phi$ is injective, then $h(F) \geq h(G)$.
  If $\phi$ is surjective, then $h(G) \leq h(F)$.
  \label{thm:ent-conj}
  \begin{proof}[Proof of Theorem~\ref{thm:ent-conj}]
    Let $\alpha$ be an open cover of $X_2$.
    We have
    \begin{align*}
      h(G, \alpha)
      &\equiv \lim\limits_{n \to \infty} \frac{1}{n} \cdot h \paren{\bigvee\limits_{i = 0}^{n-1} \itr{G}{(-i)}(\alpha)} \\
      &= \lim\limits_{n \to \infty} \frac{1}{n} \cdot h \paren{\phi^{-1} \bigvee\limits_{i = 0}^{n-1} \itr{G}{(-i)} (\alpha)} \\
      &= \lim\limits_{n \to \infty} \frac{1}{n} \cdot h \paren{\bigvee\limits_{i = 0}^{n-1} (\phi^{-1} \circ \itr{G}{(-i)}) (\alpha)} \\
      &= \lim\limits_{n \to \infty} \frac{1}{n} \cdot h \paren{ \bigvee\limits_{i = 0}^{n-1} (\itr{F}{(-i)} \circ \itr{\phi}{-1}) (\alpha))} \\
      &= \lim\limits_{n \to \infty} \frac{1}{n} \cdot h \paren{\bigvee\limits_{i = 0}^{n-1} \itr{F}{(-i)} (\itr{\phi}{-1} (\alpha))} \mbox{ (by Corollary~\ref{cor:ha-props} (i))} \\
      &\equiv h(F, \itr{\phi}{-1}(\alpha)).
    \end{align*}
  %
    Since $\phi^{-1}(\alpha) < \alpha$, by Proposition~\ref{prop:hfa-props}(i), we obtain
    \begin{equation*}
      h(G,\alpha) 
      = h(F,\itr{\phi}{-1}(\alpha))
      \leq h(F, \alpha).
    \end{equation*}
    This implies
    \begin{equation*}
      h(G) \leq h(F).
    \end{equation*}
  \end{proof}
\end{theorem}
%%%
\begin{corollary}
  Let $X$ and $Y$ be metric spaces, and $F: X \to X$ and $G: Y \to Y$ are continuous mappings.
  Suppose $G\circ \phi = \phi \circ F$.
  If $\phi$ is homeomorphic, then $h(G) = h(F)$.
\begin{proof}
  Both $\phi$ and $\itr{\phi}{-1}$ are surjective, so we have $h(G) \leq h(F)$ as well as $h(G) \leq h(F)$.
\end{proof}
\end{corollary}
%%%

The definition of topological entropy calls for computation of $h(F,\alpha)$ for all possible open covers of $X$, so computing $h(F)$ using the definition is untenable.
Since the space is compact, we only need to consider open covers whose cardinalities are finite, but the number of such coverings is still large.
However, it is possible to reduce the computation to a single sequence of open covers called a \textit{refining sequence}.
Roughly, a refining sequence is a sequence of open covers that become finer and finer.
Using a refining sequence vastly simplifies the computation of $h(F)$.
%%%
\begin{definition}
  (Refining Sequence)
  A sequence $\seq{\alpha_n}{\infty}{n = 1}$ of open covers of a compact space $X$ is called a \textit{refining sequence}, if it satisfies the following two criteria:
  \begin{enumerate}[(i)]
    \item $\alpha_n < \alpha_{n+1}$ for each $n\geq 1$
    \item If $\beta$ is an open cover of $X$, then there exists $m$ such that $\beta < \alpha_m$.
  \end{enumerate}
\end{definition}
%%%
\begin{theorem}
  Suppose $\set{\alpha_n}$ is a refining sequence of $X$, and $F:X\to X$ is a continuous mapping.
  Then,
  \begin{equation*}
    h(F) = \lim\limits_{n\to \infty} h(F, \alpha_n).
  \end{equation*}
  \label{thm:t-ent-ref-seq}
  %
  \begin{proof}
    Since $h(F) = \sup\limits_{\alpha} h(F, \alpha)$, we have the inequality
    \begin{equation*}
      h(F) \geq \lim\limits_{n\to \infty} h(F, \alpha_n).
    \end{equation*}
    Next, we show that the inequality in the other direction also holds.
    By definition of a refining sequence, for each open cover $\beta$ of $X$, there exists $k_0$ such that $\beta < \alpha_k$ for each $k > k_0$.
    It follows that
    \begin{equation*}
      h(F, \beta) \leq h(F, \alpha_k)
    \end{equation*}
    for each $k > k_0$.
    Take the supremum over $\beta$ of both sides to obtain
    \begin{equation*}
      \sup\limits_{\beta} h(F, \beta) = h(F) \leq h(F, \alpha_k).
    \end{equation*}
    Finally, take the limit $k \to \infty$ to get
    \begin{equation*}
      h(F) \leq \lim\limits_{k \to \infty} h(F, \alpha_k).
    \end{equation*}
    Hence, $h(F) = h(F, \alpha_k)$.
  \end{proof}
\end{theorem}
%%%

We may not be able to find a refining sequence for any metric space.
For a compact metric space, however, Lebesgue's covering lemma ensures the existence of a refining sequence.
\begin{proposition}
  (Lebesgue's Covering Lemma)
  Let $X$ be a compact metric space, and $\alpha$ be an open cover of $X$.
  There exists $\delta > 0$ such that, if $B$ is a subset of $X$ whose diameter is less than $\delta$, i.e. $\diam(B) < \delta$, then there exists a member $A$ of $\alpha$ such that $B \subseteq A$.
  We refer to $\delta$ as a \textit{Lebesgue number} of $\alpha$.
  \label{lem:covering}
  \begin{proof}
    This is a classical result, so we omit the proof.
    For example, see \citet{royden}.
    %  Proof by contradiction.
    %  Consider a sequence $\set{B_n}$ of subsets of $X$ such that $\diam(B_n) = 1/n$, and each $B_n$ is not contained in any member of $\alpha$.
    %  For each $B_n$, fix $x_n \in B_n$.
    %  Since $X$ is compact, there exists a subsequence of $\set{x_n}$ that converges.
    %  Call this subsequence $\set{x_{n_k}}$, and suppose it converges to $x$.
    %  Also, suppose $x$ is a member of $A \in \alpha$, and let $D = \diam(A)$.
    %  Choose $l$ large enough so that $n_l > \frac{2}{D}$ and $\metric{x_{n_l}, x} < \frac{D}{2}$.
    %  Finally, for any point $y$ in $B_{n_l}$, we have
    %  \begin{equation*}
    %    \metric{y, x_{n_l}}
    %    \leq \metric{y,x} + \metric{x, x_{n_l}}
    %    < \diam(B_{n_l})  + \frac{D}{2}
    %    = \frac{1}{n_l} + \frac{D}{2}
    %    = D.
    %  \end{equation*}
    %  This is a contradiction, since $\metric{y, x_{n_l}} < D$ implies that $B_{n_l}$ is contained in $A$.
  \end{proof}
\end{proposition}
Now, we relate Lebesgue's covering lemma to refining sequences.
The following corollary is simply a rephrasing of the lemma.
\begin{corollary}
  Define the diameter of an open cover $\alpha$ by
  \begin{equation*}
    \diam(\alpha) \ceq \sup\limits_{A \in \alpha} \diam(A).
  \end{equation*}
  Let $\alpha, \beta$ be open covers of $X$, and $\delta > 0$ be a Lebesgue number of $\alpha$.
  If $\diam(\beta) < \delta$, then $\alpha < \beta$.
  \begin{proof}
    For each $B \in \beta$, we have $\diam(B) < \delta$.
    By Lebesgue's lemma, there exists $A \in \alpha$ such that $B \in A$.
  \end{proof}
\end{corollary}
Hence, we can make the following characterization of a refining sequence.
\begin{corollary}
  If $\seq{\alpha_n}{\infty}{n = 1}$ is a sequence of open covers such that
  \begin{equation*}
    \alpha_n < \alpha_{n+1} 
    \quad\mbox{and}\quad
    \lim\limits_{n \to \infty} \diam(\alpha_n) = 0,
  \end{equation*}
  then $\set{\alpha_n}$ is a refining sequence.
  \begin{proof}
    The first condition of a refining sequence is satisfied by hypothesis.
    To show that the second condition is also satisfied, consider an arbitrary open cover $\beta$, and let $\delta$ be its Lebesgue number.
    Since $\lim\limits_{n \to \infty} \diam(\alpha_n) = 0$, there exists $n_0$ such that if $m > n_0$ then $\diam(\alpha_m) < \delta$.
    Then, for each $m > n_0$, we have $\beta < \alpha_m$.
  \end{proof}
\end{corollary}
Finally, we construct a refining sequence for an arbitrary compact metric space.
\begin{corollary}
  Any compact metric space has a refining sequence.
  \begin{proof}
    For each $n \in \N^+$, define
    \begin{equation*}
      \alpha_n \ceq \setst{\oball{1/n}{x}}{x \in X}.
    \end{equation*}
    Clearly, $\alpha_n$ is an open cover of $X$.
    We have
    \begin{equation*}
      \alpha_n < \alpha_{n+1}
      \quad\mbox{and}\quad
      \lim\limits_{n \to \infty} \diam(\alpha_n) = 0,
    \end{equation*}
    so $\set{\alpha_n}$ is a refining sequence.
  \end{proof}
\end{corollary}
We will use this result extensively in the following sections.


\section{Topological Entropy of the Logistic Map}
%%%
We saw in the introduction that $L_u$, the logistic map, which is defined as
\begin{align*}
  L_\mu&: I \to I \quad\mbox{ where } I \equiv [0,1] \\
  L_\mu&: x \mapsto \mu x (1-x),
\end{align*}
is chaotic when $\mu = 4$.
%As a demonstration of topological entropy, we will compute $h(L_\mu)$ for $\mu = 4$.
We expect that $h(L_4)$ is positive, and as we will see, this is indeed the case. 
We first show that $L_4$ is topologically conjugate to the tent map, which we first introduced in Chapter~\ref{chap:liyorke}, defined as
\begin{equation*}
  T(x) = 
  \begin{cases}
    2x \mbox{ if } x \in [0,1/2] \\
    2 - 2x \mbox{ if } x \in [1/2,1].
  \end{cases}
\end{equation*}
Once we establish this conjugacy, by Theorem~\ref{thm:ent-conj}, the problem reduces to computation of $h(T)$.
Let
\begin{equation*}
  \phi(x) = \sin^2(\frac{\pi x}{2}).
\end{equation*}
Our goal is to show that $L_4 \circ \phi = \phi \circ T$.
For any $x \in I$, letting $\theta \equiv \frac{\pi x}{2}$ for convenience, we have
\begin{align*}
  L_4 \circ \phi(x)
  &= 4\sin^2\theta(1 - \sin^2\theta) \\
  &= 4\sin^2\theta \cos^2\theta \\
  &= \sin^2 2\theta \\
  &= \sin^2 (\pi x).
\end{align*}
For any $x$ in $[0,1/2]$,
\begin{equation*}
  \phi \circ T(x)
  = \sin^2(\pi x),
\end{equation*}
and for any $x$ in $[1/2,1]$,
\begin{align*}
  \phi \circ T(x)
  &= \sin^2(\pi - \pi x) \\
  &= \sin^2(\pi x).
\end{align*}
Therefore, $L_4 \circ \phi = \phi \circ T$ on $I$.
By Theorem~\ref{thm:ent-conj}, we obtain $h(L_4) = h(T)$.
Our next task is to construct a refining sequence $\set{\alpha_n}$ of $I$.
Then, we would obtain $h(T)$ by Theorem~\ref{thm:t-ent-ref-seq}.
Our construction is as follows.
First, let $\alpha_0 = \set{I}$.
Then, for $n \geq 1$, recursively define $\alpha_n$ as
\begin{equation*}
  \alpha_n = T^{-1}(\alpha_{n-1}).
\end{equation*}
%
This recursive definition can be stated more explicitly as
\begin{align*}
  \alpha_n &= T^{-1}(\alpha_{n-1}) \\
  &= \bigcup\limits_{A \in \alpha_{n-1}} \paren{ \setst{J}{T(J) = I, J \subseteq [0,1/2]} \cup \setst{J}{T(J) = I, J \subseteq [1/2,1]}}.
\end{align*}
The expression is motivated by the fact that the tent map is a two-to-one mapping, except for when $x = \frac{1}{2}$.
For each $x \in I \bs \set{\frac{1}{2}}$, the preimage of $x$ is given by the formula
\begin{equation*}
  T^{-1}(x) = \set{\frac{1}{2} x, 1 - \frac{1}{2} x}.
\end{equation*}
Notice that if $\frac{1}{2} x \in [0,1/2)$, then $1 - \frac{1}{2} x \in (1/2,1]$; if $\frac{1}{2} x \in (1/2,1]$, then $1 - \frac{1}{2} x \in [0,1/2)$. %]indentation fix
This observation holds for the preimage of an interval.
For example,
\begin{align*}
  \alpha_1 &= \set{\brac{0,\frac{1}{2}}, \brac{\frac{1}{2}, 1}}, \\
  \alpha_2 &= \set{\brac{0,\frac{1}{4}}, \brac{\frac{3}{4}, 1}, \brac{\frac{1}{4}, \frac{1}{2}}, \brac{\frac{1}{2}, \frac{3}{4}}},
\end{align*}
and so on.
Thus, $N(\alpha_{n}) = 2^n$.
Now, we want to show that $\set{\alpha_n}$ is a refining sequence.
$\set{\alpha_n}$ clearly satisfies the first condition:
\begin{equation*}
  \alpha_n < \alpha_{n+1}.
\end{equation*}
To show the other condition, that for each open cover $\beta$ of $X$, there exists $k_0$ such that $\beta < \alpha_k$ for each $k \geq k_0$, we invoke Lebesgue's covering lemma.
Let $\delta > 0$ be the constant given by applying the lemma to $\beta$.
For each $J_n \in \alpha_n$, we have $\diam(J_n) = 1/{2^n}$.
Choose $n$ large enough so that $\diam(J_n) \leq \delta$.
Then, for each $A \in \alpha_n$, the lemma guarantees the existence of $B \in \beta$ such that $A \subseteq B$.
This implies $\beta < \alpha_n$.
We conclude that $\set{\alpha_n}$ is a refining sequence.
Finally, we compute $\lim\limits_{n \to \infty} \frac{1}{n} h(T, \alpha_n)$.
Recall that $N(\alpha_{n}) = 2^n$.
Then,
\begin{align*}
  h(T, \alpha_n)
  &= \lim\limits_{m \to \infty} \frac{1}{m} \log N\paren{ \bigvee\limits_{i = 0}^{m-1} T^{-i}(\alpha_n)}  \\
  &= \lim\limits_{m \to \infty} \frac{1}{m} \log N(\alpha_{n+m-1})  \\
  &= \lim\limits_{m \to \infty} \frac{1}{m} \log 2^{n + m - 2}  \\
  &= \lim\limits_{m \to \infty} \frac{n + m - 2}{m} \log 2 \\
  &= \log 2.
\end{align*}
Therefore,
\begin{equation*}
  h(L_4) = h(T) = \lim\limits_{n \to \infty} h(T, \alpha_n) = \log 2.
\end{equation*}

\section{Topological Entropy of Other Maps}
The topological entropy of full one-sided or two-sided shift with $n$ symbols is known to be $\log n$ \citep[p.177]{walters}.
Since the logistic map is conjugate (by a homeomorphism) to the full one-sided shift of two symbols $\set{0,1}$, our result from Chapter~\ref{chap:symbolic} agrees with this property of shift systems.
One important observation about topological entropy is that, even if a mapping has positive topological entropy, its behavior may be complicated only locally.
\begin{proposition}
  Let $F: X \to X$ be continuous, and $Y$ be a closed (hence compact), invariant subset of $X$.
  Then, $h(F_{|Y}) \leq h(F)$.
  \label{prop:local-entropy}
  \begin{proof}
    Let $\alpha$ be an open cover of $Y$.
    For each $A \in \alpha$, there exists an open subset $A' \subseteq X$ such that $A = A' \cap Y$.
    Let $\alpha'$ be the collection of such $A'$.
    Then, $\beta \ceq \alpha' \cup \set{X\bs Y}$ is an open cover of $X$.
    Define $G \ceq F_{|Y}$ and let $n$ be any positive integer.
    Since $\alpha$ is trivially a refinement of $\beta$, we have
    \begin{equation*}
      h(\alpha \vee \itr{G}{-1}(\alpha) \vee \cdots \vee \itr{G}{-n+1}(\alpha)) 
      \leq
      h(\beta \vee \itr{F}{-1}(\beta) \vee \cdots \vee \itr{F}{-n+1}(\beta)).
    \end{equation*}
    It follows that
    \begin{equation*}
      h(G, \alpha) \leq h(F, \beta) \leq h(F).
    \end{equation*}
    By taking the supremum over all $\alpha$, we obtain $h(G) \leq h(F)$.
  \end{proof}
\end{proposition}
%%%
For example, consider extending the domain of the tent map beyond $[0,1]$.
\begin{example}
  Define $\tilde{T}: \R \to \R$ as
  \begin{equation*}
    \tilde{T}(x) 
    \ceq \begin{cases}
      &T(x) \mbox{ if } (x \in [0,1]) \\
      &0 \mbox{ otherwise}.
    \end{cases}
  \end{equation*}
  Since $\tilde{T}_{|[0,1]}(x) = T(x)$, we have $h(\tilde{T}_{|[0,1]}(x)) = \log 2$. 
  By Proposition~\ref{prop:local-entropy}, $h(T) \geq \log 2 > 0$.
  Clearly, $\tilde{T}: \R \to \R$ is not topologically transitive, but has positive topological entropy.
  $\square$
\end{example}
%%%
We defined a mapping to be chaotic if it has positive topological entropy.
This situation is similar to what we encountered in Devaney's definition, which requires the existence of a compact subset on which the conditions are satisfied.
Thus, we can reformulate the definition as follows to capture its essence.
\begin{definition}
  (Chaos in the sense of AKM, redefined)
  $F$ is said to be chaotic in the sense of AKM if there exits a compact subset $Y \subseteq X$ such that $F: X \to X$ restricted to $Y$ is topologically transitive and $h(F_{|Y})$ is positive.
  \index{definition of chaos!topological entropy}
\end{definition}
%%%

The maps that we have discussed so far had positive topological entropies.
Now, we show a few examples of mappings that have zero topological entropies.
Our first example is an isometry.
It is easy to see that an isometry cannot exhibit sensitive dependence on initial conditions, and as expected, every isometry has zero topological entropy.
\begin{proposition}
  \citep{akm}
  If $F: X \to X$ is an isometry, then $h(F) = 0$.
  \begin{proof}
    Let $\alpha_r$ be the set of all open sets of diameter less than $1/r$.
    Clearly, $\alpha_r$ is an open cover of $X$.
    Since $F$ is an isometry (hence injective), and preserves diameters of sets, we have
    \begin{equation*}
      \itr{F}{-1}(\alpha_r) = \alpha_r,
    \end{equation*}
    which implies that
    \begin{equation*}
      \alpha_r = \alpha_r \vee \itr{F}{-1}(\alpha_r) \vee \cdots \vee \itr{F}{-n+1}(\alpha_r).
    \end{equation*}
    Then, for each $n$,
    \begin{equation*}
      h\paren{ \bigvee\limits_{i=0}^{n-1} \itr{F}{(-i)}(\alpha_r) }   
      = h\paren{\alpha_r}.
    \end{equation*}
    Hence, for each $r$,
    \begin{equation*}
      h(F, \alpha_r)
      = \lim\limits_{n \to \infty} \frac{1}{n} \cdot h(\alpha_r)
      = 0.
    \end{equation*}
    Since $\seq{\alpha_r}{\infty}{r = 1}$ is a refining sequence, we conclude that $h(F) = 0$.
  \end{proof}
\end{proposition}
Another example is a homeomorphism from a circle to itself.
\begin{proposition}
  \citep{akm}
  Let $S \ceq \setst{x \in \R^2}{\norm{x} = 1}$.
  If $F: S \to S$ is a homeomorphism, then $h(F) = 0$.
  \begin{proof}
    Let $\alpha_r$ be the set of all open arcs of $S$ of length $1/r$.
    Then, $\alpha_r$ is an open cover of $S$.
    Since $F$ is a homeomorphism, 
    \begin{equation*}
      \bigvee\limits_{i = 0}^{n-1} \itr{F}{-i}(\alpha_r)
    \end{equation*}
    is also a covering of $S$ by open arcs.
    We have
    \begin{equation*}
      N(\itr{F}{-1}(\alpha)) \leq N(\alpha),
    \end{equation*}
    which implies
    \begin{equation*}
      N\paren{ \bigvee\limits_{i = 0}^{n-1} \itr{F}{-i}(\alpha_r) }
      \leq \sum\limits_{i = 0}^{n-1} N(\itr{F}{-i}(\alpha_r))
      \leq \sum\limits_{i = 0}^{n-1} N(\alpha_r)
      = n N(\alpha_r).
    \end{equation*}
    It follows that, for each $r$,
    \begin{align*}
      h(F, \alpha_r) 
      &\equiv \lim\limits_{n \to \infty} \frac{1}{n} \cdot \log N\paren{ \bigvee\limits_{i = 0}^{n-1} \itr{F}{-i}(\alpha_r) }  \\
      &\leq \lim\limits_{n \to \infty} \frac{1}{n} \cdot \log (n N(\alpha_r))  \\
      &= \lim\limits_{n \to \infty}\paren{ \frac{\log n}{n} + \frac{\log N(\alpha_r)}{n} } \\
      &= 0.
    \end{align*}
    Since $\seq{\alpha_r}{\infty}{r = 1}$ is a refining sequence, we obtain $h(F) = 0$.
  \end{proof}
\end{proposition}
Also, a homeomorphism from a compact interval to itself has zero topological entropy \citep[p.180]{walters}.
Thus, these mappings that we expect to be non-chaotic indeed have zero topological entropies. 
However, as we will see in Chapter~\ref{chap:comparison}, mappings with zero topological entropy can be chaotic in the sense of Li-Yorke, and exhibit complex behaviors.

\bibliographystyle{../../bibliography/pjgsm}
\bibliography{../../bibliography/thesis}

\printindex
\end{document}

