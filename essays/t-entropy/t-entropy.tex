\documentclass[12pt,twoside,draft]{book}
\usepackage{../../thesis}
\graphicspath{ {../../images/} }

\makeindex
\begin{document}
\chapter{Topological Entropy}
The notion of topological entropy was first introdeced by \citet{akm}.
Topological entropy mimics the physical notion of entropy employed in statistical mechanics.
I will first discuss basic properties of topological entropy, then compute topological entroies for some mappings.
I will also discuss relation of topological entropy to chaos in Li-Yorke's sense.

\section{Background}
Throughout this section, $(X,d)$ is a compact metric space.
$F$ is a continuous mapping from $X$ to $X$ unless noted otherwise.
Our definition of topological entropy uses open covers of a space. 

\begin{definition}
  (Join of Open Covers)
  Let $\alpha$ and $\beta$ be open covers of $X$.
  Their join, denoted $\alpha \vee \beta$, is defined as
  \begin{equation*}
    \alpha \vee \beta \ceq \setst{A \cap B}{A \in \alpha, B \in \beta}.
  \end{equation*}
  We define the join of a countable collection of open covers in the same manner, and denote it as $\bigvee\limits_{i = 1}^{n} \alpha_i$.
\end{definition}

\begin{definition}
  (Refinement of an Open Cover)
  An open cover $\beta$ is said to be a refinement of another open cover $\alpha$, if for each $A \in \alpha$, there exists $B \in \beta$ such that $B \subseteq A$.
  Note that $\alpha < \alpha \vee \beta$.
\end{definition}

\begin{definition}
  (Preimage of an Open Cover)
  Let $\alpha$ be an open cover of $X$, and $F: X \to X$ be a continuous mapping.
  Then $F^{-1}(\alpha)$ is defined as the following:
  \begin{equation*}
    F^{-1}(\alpha) \ceq \setst{F^{-1}(A)}{A \in \alpha}.
  \end{equation*}
\end{definition}
\begin{proposition}
  Let $\alpha$ and $\beta$ be open covers of $X$.
  We have
  \begin{enumerate}[(i)]
    \item $F^{-1}(\alpha \vee  \beta) = F^{-1}(\alpha) \vee F^{-1}(\beta)$
    \item $\alpha < \beta \Rar F^{-1}(\alpha) < F^{-1}(\beta)$.
  \end{enumerate}
  \begin{proof}
    (i) The statement follows from the fact that $F^{-1}(A \cap B) = F^{-1}(A) \cap F^{-1}(B)$. \\
    (ii) If $A \subseteq B$ then $F^{-1}(A) \subseteq F^{-1}(B)$.
  \end{proof}
\end{proposition}

\begin{definition}
  (Topological Entropy)
  The topological entropy of an open cover $\alpha$ is defined as
  \begin{equation*}
    h(\alpha) = \log N(\alpha),
  \end{equation*}
  where $N(\alpha)$ is the minimum of cardinalities of subcovers of $\alpha$, i.e.
  \begin{equation*}
    N(\alpha) \ceq \min\setst{ \#(\beta) }{ \beta \mbox{ is a subcover of } \alpha}.
  \end{equation*}
  The topological entropy of a mapping $F$ relative to a open cover $\alpha$ is defined as (prove the existence of this limit)
  \begin{equation*}
    h(F,\alpha) = \lim\limits_{n \to \infty} \frac{1}{n} \cdot h\paren{ \bigvee\limits_{i=0}^n \itr{F}{(-i)}(\alpha) }.
  \end{equation*}
  Finally, the \textit{topological entropy} of a mapping $F$, denoted $h(F)$, is defined as the supremum of $h(F, \alpha)$ taken over all open covers of $X$:
  \begin{equation*}
    h(F) = \sup\limits_\alpha h(F,\alpha).
  \end{equation*}
  \label{defn:t-entropy}
  \index{topological entropy}
\end{definition}

As we claimed at the beginning of this chapter, the topological entropy is invarant under topological conjugacy.
The precise statement is the following theorem.
\begin{theorem}
  (Topological Entropy is an Invariant of Topological Conjugacy)
  Let $X_1$ and $X_2$ be metric spaces, and $F_1: X_1 \to X_1$ and $F_2: X_2 \to X_2$ are continuous mappings.
  If $\phi: X_1 \to X_2$ is a homeomorphism, then $h(F_1) = h(F_2)$.
  \label{thm:t-ent-conj}
\end{theorem}
We require a lemma in order to prove the theorem.

\begin{lemma}
  Let $F: X\to X$ be a continuous mapping.
  Then $h(F^{-1}(\alpha)) \leq h(\alpha)$.
  Furthermore, if $F$ is also surjective, then $h(F^{-1}(\alpha)) = h(\alpha)$.
  \begin{proof}
    Suppose $n = N(\alpha)$, and $\set{A_1, \ldots, A_{n}}$ is a subcover of $\alpha$ of minimal cardinality.
    Then, $\set{F^{-1}(A_1), \ldots, F^{-1}(A_{n})}$ is a subcover of $F^{-1}(\alpha)$.
    This proves the first statement.
    To prove the second statement, suppose $m = N(F^{-1}(\alpha))$.
    Let $\set{F^{-1}(A_1), \ldots, F^{-1}(A_{m})}$ be a subcover of $F^{-1}(\alpha)$ of minimal cardinality.
    If $F$ is surjective, then $\set{A_1, \ldots, A_{m}}$ is a subcover of $\alpha$.
  Hence $h(F^{-1}(\alpha)) \geq h(\alpha)$.
  \end{proof}
  
\end{lemma}

Now we proceed to prove the theorem.
\begin{proof}[Proof of Theorem~\ref{thm:t-ent-conj}]
  Let $\alpha$ be an open cover of $X_2$.
  We have
  \begin{align*}
    h(F_2, \alpha)
    &= \lim\limits_{n \to \infty} \frac{1}{n} \cdot h \paren{\bigvee\limits_{i = 0}^{n-1} \itr{F_2}{(-i)}(\alpha)} \\
    &= \lim\limits_{n \to \infty} \frac{1}{n} \cdot h \paren{\bigvee\limits_{i = 0}^{n-1} \phi^{-1} \circ \itr{F_1}{(-i)} \circ \phi (\alpha)} \\
    &= \lim\limits_{n \to \infty} \frac{1}{n} \cdot h \paren{\phi^{-1} \bigvee\limits_{i = 0}^{n-1} \itr{F_1}{(-i)} (\phi (\alpha))} \\
    &= \lim\limits_{n \to \infty} \frac{1}{n} \cdot h \paren{\bigvee\limits_{i = 0}^{n-1} \itr{F_1}{(-i)} (\phi (\alpha))} \mbox{ (by the lemma)} \\
    &= h(F_1, \phi(\alpha)).
  \end{align*}
  %
  As $\alpha$ ranges over all open covers of $X_1$, $\phi(\alpha)$also ranges over all open covers of $X_2$, since $\phi$ is a homeomorphism.
  Therefore,
  \begin{equation*}
    h(F_2) = h(F_1).
  \end{equation*}
\end{proof}


The definition of topological entropy calls for computation of $h(F,\alpha)$ for all possible open covers of $X$, and thus, computing $h(F)$ using the definition is untenable.
As a reasonable method for computing $h(F)$, we introduce a theorem, which states that $h(F)$ equals to the limit of the topological entropy of $F$ relative to a sequence of open covers that become finer and finer later in the sequence.
We call such a sequence a \textit{refining sequence}, defined as follows.

\begin{definition}
  (Refining Sequence)
  A sequence $\set{\alpha_n}$ of open covers of a compact space $X$ is called a \textit{refining sequence}, if it satisfies the following two criteria:
  \begin{enumerate}[(i)]
    \item $\alpha_n < \alpha_{n+1}$ for each $n\geq 0$
    \item if $\beta$ is an open cover of $X$, then there exists $m$ such that $\beta < \alpha_m$.
  \end{enumerate}
\end{definition}

\begin{theorem}
  Suppose $\set{\alpha_n}$ is a refining sequence of $X$, and $F:X\to X$ is a continuous mapping.
  Then,
  \begin{equation*}
    h(F) = \lim\limits_{n\to \infty} h(F, \alpha_n).
  \end{equation*}
  \label{thm:t-ent-ref-seq}
  %
  \begin{proof}
    Since $h(F) = \sup\limits_{\alpha} h(F, \alpha)$, the inequality
    \begin{equation*}
      h(F) \geq \lim\limits_{n\to \infty} h(F, \alpha_n).
    \end{equation*}
    holds.
    To show that the opposite inequality also holds, note that for each open cover $\beta$ of $X$, there exists $k_0$ such that, for each $k > k_0$, $\beta < \alpha_k$.
    It follows that
    \begin{equation*}
      h(F, \beta) \leq h(F, \alpha_k).
    \end{equation*}
    Take the supremum over $\beta$ of both sides to obtain
    \begin{equation*}
      \sup\limits_{\beta} h(F, \beta) = h(F) \leq h(F, \alpha_k).
    \end{equation*}
    Finally, take the limit $k \to \infty$ to get
    \begin{equation*}
      h(F) \leq \lim\limits_{k \to \infty} h(F, \alpha_k).
    \end{equation*}
    Hence, $h(F) = h(F, \alpha_k)$.
  \end{proof}
\end{theorem}

We will use Lebesgue's Covering Lemma in the next section to show that some sequence of open covers satisfies the second condition of the definition of refining sequence.
\begin{lemma}
  (Lebesgue's Covering Lemma)
  Let $X$ be a compact metric space, and $\alpha$ be an open cover of $X$.
  There exists $\delta > 0$ such that, if $B$ is a subset of $X$ whose diameter is less than or equal to $\delta$, i.e. $\diam(B) \leq \delta$, then there exists a member $A$ of $\alpha$ such that $B \subseteq A$.
  \label{lem:covering}
    \index{Lebesgue's covering lemma}

    \begin{proof}
    Proof by contradiction.
    Consider a sequence $\set{B_n}$ of subsets of $X$ such that $\diam(B_n) = 1/n$, and each $B_n$ is not contained in any member of $\alpha$.
    For each $B_n$, fix $x_n \in B_n$.
    Since $X$ is compact, there exists a subsequence of $\set{x_n}$ that converges.
    Call this subsequence $\set{x_{n_k}}$, and suppose it converges to $x$.
    Also, suppose $x$ is a member of $A \in \alpha$, and let $D = \diam(A)$.
    Choose $l$ large enough so that $n_l > \frac{2}{D}$ and $\metric{x_{n_l}, x} < \frac{D}{2}$.
    Finally, for any point $y$ in $B_{n_l}$, we have
    \begin{equation*}
      \metric{y, x_{n_l}}
      \leq \metric{y,x} + \metric{x, x_{n_l}}
      < \diam(B_{n_l})  + \frac{D}{2}
      = \frac{1}{n_l} + \frac{D}{2}
      = D.
    \end{equation*}
    This is a contradiction, since $\metric{y, x_{n_l}} < D$ implies that $B_{n_l}$ is contained in $A$.
  \end{proof}
\end{lemma}

\section{The Topological Entropy of Logistic Map}
We saw in the introduction that $L_u$, the logistic map, which is defined as
\begin{equation*}
  L_\mu(x) = \mu x (1-x),
\end{equation*}
 exhibits a sensitive dependence on initial conditions when $\mu$ is greater than a value approximate to $3.57\ldots$ (though there are exceptions. See the bifurcation diagram for $L_\mu$). 
As a demonstration of topological entropy, we will compute $h(L_\mu)$ for $\mu = 4$.
Since $L_\mu$ is chaotic when $\mu = 4$, we expect that $h(L_4)$ is positive.
As we will see, this is indeed the case; by applying Theorem~\ref{thm:t-ent-ref-seq} topological conjugacy and 

\bibliographystyle{../../bibliography/pjgsm}
\bibliography{../../bibliography/thesis}

\printindex
\end{document}

