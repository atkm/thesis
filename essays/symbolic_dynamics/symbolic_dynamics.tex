\documentclass[12pt,twoside,draft]{book}
\usepackage{../../thesis}
\graphicspath{ {../../images/} }

\makeindex
\begin{document}
\chapter{Symbolic Dynamics}
\section{Invariant Set of The Horseshoe Map}
(\citep{wiggins})
Recall the Baker's mapping that we defined ealier ref{def:baker}.
Consider the following variant of this mapping:
\begin{definition}
  (Horseshoe Map)
  Let $D = I \times I$, where $I = [0,1]$.
  Define a two-parameter map $H_{\lambda, \mu}: D \to \R^2$ as
  \begin{equation*}
    H_{\lambda, \mu}(x,y) =
    \begin{cases}
      &H_0(x,y) \quad\mbox{ if } 0 \leq y \leq \frac{1}{\mu}   \\
      &H_1(x,y) \quad\mbox{ if } 1 - \frac{1}{\mu} \leq y \leq 1,
    \end{cases}
  \end{equation*}
  where
  \begin{align*}
    H_0(x,y) &= (\lambda x, \mu y)  \\
    H_0(x,y) &= (1 - \lambda x, \mu(1 - y)) 
  \end{align*}.
  $H_{\lambda, \mu}$ is called the horseshoe map.
  (We will often omit the subscripts.)
\end{definition}

VISUALIZATION

In the next, section, we will prove the following properties of $H$ and its invariant set:
\begin{enumerate}[(i)]
  \item $H$ has a countable infinity of periodic orbits of any period.
  \item $H$ has an uncountable infinity of non-periodic orbits.
  \item $H$ has a dense orbit (analogous to transitivity).
  \item $H$ has a \textit{sensitive dependence on initial condition} on its invariant set.
\end{enumerate}

\section{Two-Sided Shift}
($D$, $H$) is conjugate to the \textit{two-sided shift} ($\Sigma$, $\sigma$), defined as follows.
\begin{definition}
  (Two-Sided Shift)
  Let $S$ be a finite set of symbols, and let $\Sigma$ be a set of bi-infinite sequences of the form
  \begin{equation*}
    s = \set{ \cdots s_{-n} \cdots s_{-1} * s_0 \cdots s_n \cdots },
  \end{equation*}
  where $s_i \in S$, and the asterisk is a placeholder to denote the 0th place (i.e. the symbol after $*$ is the 0th place).
  In particular, we use $S = \set{0,1}$.
\end{definition}

\begin{proposition}
  $d$ is indeed a metric.
  \label{prop:symb-metric}
\end{proposition}
\begin{proposition}
  $\sigma$ is continuous.
  \label{prop:symb-sigma-cont}
\end{proposition}
As a consequence, ($D$, $H$) possesses any topological property of ($\Sigma$, $\sigma$), and vice versa.

On a related note, the logistic map, tent map, and doubling maps are all conjugate to the \textit{one-sided shift}, the one-dimensional analogoue of the two-sided shift.

\bibliographystyle{../../bibliography/pjgsm}
\bibliography{../../bibliography/thesis}

\printindex
\end{document}

