\documentclass[11pt]{article}
\usepackage{thesis}

\begin{document}

\section{What do physicists say?}
Although there is no universally accepted definition of chaos,
most experts would concur that chaos is the {\it aperiodic,
  long-term behavior of a bounded, deterministic system
that exhibits sensitive dependence on initial conditions} (Sprott, p104)
\section{Chaos in Li-Yorke sense}
\begin{proposition}
  Li, Yorke.
\end{proposition}

But this doesn't apply to baker's map.

\section{Devaney's Definition}
Devaney defines chaos in topological dynamics as follows:


\begin{definition}
  (Topological Transitivity) $f: J \rar J$ is said to be topologically
  transitive if for any pair of open sets $U$, $V \subseteq J$
  there exists $k > 0$ such that $f^k(U) \cap V \neq \emptyset$.
\end{definition}
\begin{definition}
  (Sensitive Dependence on Initial Conditions) $f: J \rightarrow J$ 
  is said to have sensitive initial conditions if there exists (for any???) $\delta > 0$
  such that, for any $x \in J$ and any neighborhood $N$ of $x$,
  there exists $y\in N$ and $n\leq 0$ such that $|f^n(x) - f^n(y)|>\delta$.
\end{definition}

\begin{definition}
  (Chaotic Mapping) Let $V$ be a set. $f: V\rar V$ is said to be chaotic on $V$ if
  \begin{enumerate}
    \item $f$ has sensitive dependence on initial conditions.
    \item $f$ is topologically transitive.
    \item Periodic points are dense in $V$.
  \end{enumerate}
\end{definition}

Note that we do not require that
``\it{all} points near $x$ need eventually separate from $x$
under iteration, but there must be at least one such point in
every neighborhood of $x$.

\begin{definition}
  (Expansiveness) $f: J\rar J$ is expansive if there exists $\nu > 0$
  such that, for any $x,y\in J$, there exists $n$ such that
  $|f^nx-f^ny| > \nu$.
\end{definition}

What happens if we required expansiveness?

Can we derive one from the others? Yes. Sensitive dependence
can be derived from transivity and dense property if the mapping
is continuous and $X$ is a compact, invariant set. (Banks et al. 1992)
Furthermore, for an interval, {\bf the only property necessary is
topological transitivity} (Vellekoop-Berglund, 1994).

There seems to be a connection between topological transitivity and
``period three implies chaos''.

A system is chaotic in $X$ in the sense of Devaney if and only if
for every pair of open sets in $X$ there exists a periodic orbit 
which visits both sets (Touhey, 1997).

\begin{proposition}
  (Martelli et al., 1998)(Equivalence of stability and transitivity)
  Let $X\subset R^n$ be compact and $F: X\rar X$ be continuous.
  Then $F$ is topologically transitive in $X$ if and only if
  there exists $x_0\in X$ such that $O(x_0)$ is dense in $X$.
  In addition, $F$ has sensitive dependence on initial conditions 
  with respect to $X$ if and only if $O(x_0)$ is unstable in $X$.
\end{proposition}

Importantly, theorem still holds if the condition on $F$ is 
weakened to {\it quasi-continuous} function, a class of functions
where baker's map belongs to.

Also an important notion:
\begin{definition}
  (Structural Stability) $f: J \rar J$ is said to be $C^r$-structurally
  stable on $J$, if there exists $\epsilon > 0$ such that whenever
  $d_r(f,g) < \epsilon$ for $g: J\rar J$, it follows that $f$
  is topologically conjugate to $g$.
\end{definition}

Then, chaos can be defined in terms of stability.
\begin{definition}
  Let $U$ be open in $\R^n$, $F$ a continuous mapping $F: U\rar \R^n$ and $X\subset U$
  closed, bounded (the two implies compact space), and invariant.
  $F$ is said to be {\it chaotic} in $X$ if there exists $x_0\in X$
  such that
  \begin{enumerate}
    \item $O(x_0)$ is dense in $X$.
    \item $O(x_0)$ is unstable in $X$.
  \end{enumerate}
\end{definition}

It turns out structural stability is equivalent to sensitive dependence
on initial conditions.

\begin{proposition}
  (Conjugacy, Transitivity, and Homeomorphism) 
  Suppose that $g \compose h = h \compose f$. If $h$ is continuous and
  surjective and $f$ is transitive, then $g$ is transitive.
  If $h$ is a homeomorphism, then $f$ is transitive if and only if
  $g$ is transitive.
\end{proposition}

\section{Ergodic Theoretical Definition}
In Ergodic Theory

\end{document}
