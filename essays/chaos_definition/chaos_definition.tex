\documentclass[11pt]{article}
\usepackage{thesis}

\begin{document}
\section{Devaney's Definition}
Devaney defines chaos in topological dynamics as follows:

\begin{definition}
  (Topological Transitivity) $f: J \rar J$ is said to be topologically
  transitive if for any pair of open sets $U$, $V \subseteq J$
  there exists $k > 0$ such that $f^k(U) \cap V \neq \emptyset$.
\end{definition}
\begin{definition}
  (Sensitive Dependence on Initial Conditions) $f: J \rightarrow J$ 
  is said to have sensitive initial conditions if there exists (for any???) $\delta > 0$
  such that, for any $x \in J$ and any neighborhood $N$ of $x$,
  there exists $y\in N$ and $n\leq 0$ such that $|f^n(x) - f^n(y)|>\delta$.
\end{definition}

\begin{definition}
  (Chaotic Mapping) Let $V$ be a set. $f: V\rar V$ is said to be chaotic on $V$ if
  \begin{enumerate}
    \item $f$ has sensitive dependence on initial conditions.
    \item $f$ is topologically transitive.
    \item Periodic points are dense in $V$.
  \end{enumerate}
\end{definition}

Note that we do not require that
``\it{all} points near $x$ need eventually separate from $x$
under iteration, but there must be at least one such point in
every neighborhood of $x$.

\begin{definition}
  (Expansiveness) $f: J\rar J$ is expansive if there exists $\nu > 0$
  such that, for any $x,y\in J$, there exists $n$ such that
  $|f^nx-f^ny| > \nu$.
\end{definition}

What happens if we required expansiveness?
Can we derive one from the others? In particular, sensitive dependence
can be derived from transivity and dense property.

Also an important notion:
\begin{definition}
  (Structural Stability) $f: J \rar J$ is said to be $C^r$-structurally
  stable on $J$, if there exists $\epsilon > 0$ such that whenever
  $d_r(f,g) < \epsilon$ for $g: J\rar J$, it follows that $f$
  is topologically conjugate to $g$.
\end{definition}

\begin{proposition}
  (Conjugacy, Transitivity, and Homeomorphism) 
  Suppose that $g \compose h = h \compose f$. If $h$ is continuous and
  surjective and $f$ is transitive, then $g$ is transitive.
  If $h$ is a homeomorphism, then $f$ is transitive if and only if
  $g$ is transitive.
\end{proposition}

\section{Ergodic Theoretical Definition}
In Ergodic Theory
\end{document}
