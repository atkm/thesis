\documentclass[11pt]{book}
\usepackage{../../thesis}

\begin{document}
\section{Martelli's Definition}

\begin{definition}
  (Expansiveness) $f: J\rar J$ is expansive if there exists $\nu > 0$
  such that, for any $x,y\in J$, there exists $n$ such that
  $|f^{n}(x)-f^{n}(y)| > \nu$.
  \index{expansiveness}
\end{definition}

Let us remind ourselves of the definition of chaos due to Devaney.

It turns out structural stability is equivalent to sensitive dependence on initial conditions.
\begin{proposition}
  (Martelli et al., 1998)(Equivalence of stability and transitivity)
  Let $X\subset R^n$ be compact and $F: X\rar X$ be continuous.
  Then $F$ is topologically transitive in $X$ if and only if there exists $x_0\in X$ such that $O(x_0)$ is dense in $X$.
  In addition, $F$ has sensitive dependence on initial conditions with respect to $X$ if and only if $O(x_0)$ is unstable in $X$.
\end{proposition}

Importantly, theorem still holds if the condition on $F$ is weakened to {\it quasi-continuous} function, a class of functions where baker's map belongs to.

\begin{definition}
  (Structural Stability) $f: J \rar J$ is said to be $C^r$-structurally
  stable on $J$, if there exists $\epsilon > 0$ such that whenever
  $d_r(f,g) < \epsilon$ for $g: J\rar J$, it follows that $f$
  is topologically conjugate to $g$.
  \index{structural stability}
\end{definition}

Then, chaos can be defined in terms of stability.

\begin{definition}
  (Chaos in Martelli's sense)
  Let $U$ be open in $\R^n$, $F: U \to \R^n$, and $X\subset U$ be closed, bounded and invariant.
  Assume that $F$ is continuous in $X$.
  Then $F$ is said to be \textit{chaotic} in $X$ if there exists $x_0 \in X$ that meets the following conditions
  \begin{enumerate}
    \item $O(x_0)$ is dense in $X$.
    \item $O(x_0)$ is unstable in $X$.
  \end{enumerate}
  \label{defn:chaosmartelli}
  \index{definition of chaos!Martelli}
\end{definition}

Wiggins came up with a definiton that is similar to Deveney's expect for not requiring dense periodic orbits.

\begin{definition}
  (Chaos in Wiggins's sense)
   Let $U$ be open in $\R^n$, $F: U \to \R^n$, and $X\subset U$ be closed, bounded and invariant.
  Assume that $F$ is continuous in $X$.
  Then $F$ is chaotic in $X$ if the mapping satisfies the following conditions
  \begin{enumerate}
    \item $F$ is topologically transitive in $X$
    \item $F$ has sensitive dependence on initial conditions on $X$.
  \end{enumerate}
  \label{defn:chaoswiggins}
\end{definition}

The outline of proof is in \cite{martelli98}.
\begin{theorem}
  (Equivalence of Martelli's and Wiggins's definitions)
  Let $X$ be a compact invariant subset of $\R^n$ and $F: X\to X$ a continuous map.
  Then $F$ is chaotic in Martelli's sense if and only if $F$ is chaotic in Wiggins's sense.
  \label{thm:martelliwigginsequiv}
\end{theorem}

\begin{proof}
   \begin{enumerate}[(i)]
    \item $F$ is topologically transitive in $X$ if and only if there exists $x_0 \in X$ such that $L(x_0) = X$.

    It is straightforward to show that topological transitivity implies the existence of an orbit whose limit set is $X$.
    Suppose $U,V$ are open subsets of $X$.
    The intersection of $U$ and $L(x_0)$ is clearly non-empty, so let $K = U \cap L(x_0)$.
    For sufficiently large $n$, we obtain $F^n(K) = L(x_0)$ because $K$ contains a point in $O(x_0)$.
    Therefore, $F^n(K) \cap V \neq \emptyset$.

    Next, suppose that $L(x_0) = X$.
    A finite open cover of $X$ exists because $X$ is compact.
    Choose $m$ points from $X$ such that $X = \bigcup\limits_{i=1}^m N_{1/m}(x_i)$
    so that the set $\set{N_{1/m}(x_1), \ldots, N_{1/m}(x_m)}$ forms a finite open cover of $X$.
    (Note that choice of $x_1, \ldots, x_m$ depends on $m$.
    So, for example, $x_1$ should be written as $x_{1,m}$.
    However, we omit this $m$ to avoid cluttering up the argument.)
    Define sets $Y_1, \ldots, Y_m$ as
    \begin{align*}
      Y_i &:= F^{n(i)}(N_{1/m}(Y_{i-1})) \cap N_{1/m}(x_{i+1}) \quad (\mbox{for } i \geq 2) \\
      Y_1 &:= N_{1/m}(x_1),
    \end{align*}
    where $n(i)$ is an integer large enough to ensure each $Y_i$ is not empty, which is guaranteed to exist by topological transitivity.
    Finally, let $T = F^{-\sum\limits_{i=1}^{m}n(i)}(Y_m) \cap Y_1$.
    By construction, a point in $T$ visits every $N_{1/m}(x_i)$.
    Furthermore, as we take $m \to \infty$, we may assume that $\set{x_{1,m}}_m$ converges to a point, 
    since, if it doesn't, we can find a subsequence of $\set{x_{1,m}}_m$ that converges and replace the original sequence with the convergent subsequence.
    In the limit $m \to \infty$, any point in $T$, say $x_0$, satisfies $L(x_0) = X$.

    \item $F$ has sensitive dependence on initial conditions with respect to $X$ if and only if $O(x_0)$ is unstable with respect to $X$.

      The ``only if'' part is obvious, since sensitive dependence on initial conditions is a stronger condition.
      Of course, the ``if'' part is not true in general. We will show, however, 
      that the two conditions are equivalent if $F$ has an unstable orbit
      whose limit set fills up $X$.
      So assume that $O(x_0)$ is an unstable orbit such that $L(x_0) = X$,
      and let $r(x_0)$ be its separation constant.
      It follows that for each $y_0\in X$ and $d > 0$ there exists $p>0$ such that $\metric{y_0, F^p(x_0)} \leq d/2$.
      One case is that there exists $n$ such that $\metric{F^n(y_0), F^{p+n}(x_0)} > r(x_0)/2$.
      In the other case (i.e. for each $n$, $\metric{F^n(y_0), F^{p+n}(x_0)} \leq r(x_0))/2$,
      there must exist a different point $w_0$ in a neighborhood of $F^p(x_0)$ such that 
      \begin{equation*}
        \metric{w_0, F^p(x_0)} \leq \frac{d}{2} \quad \mbox{ and } \quad \metric{F^m(w_0), F^{m+p}(x_0)} > r(x_0).
      \end{equation*}
      Note that
      \begin{equation*}
        \metric{y_0, w_0} \leq \metric{y_0, F^p(x_0)} + \metric{F^p(x_0), w_0} \leq d
      \end{equation*}
      by the triangle inequality.
      Moreover,
      \begin{align*}
        r(x_0) &< \metric{F^m(w_0), F^{m+p}(x_0)} \\
        &\leq \metric{F^{m+p}(x_0), F^m(y_0)} + \metric{F^m(y_0), F^m(w_0)}  \\
        &\leq \frac{r(x_0)}{2} + \metric{F^m(y_0), F^m(w_0)}
      \end{align*}
      Hence in both cases we have
      \begin{equation*}
        \frac{r(x_0)}{2} < \metric{F^m(y_0), F^m(w_0)}.
      \end{equation*}
      Thus, $F$ has sensitive dependence on initial conditions with respect to $X$, and 
      $r(x_0)/2$ is the separation constant for any point in $X$.
  \end{enumerate}
 
\end{proof}

\bibliographystyle{pjgsm}
\bibliography{../../bibliography/thesis}

\end{document}
