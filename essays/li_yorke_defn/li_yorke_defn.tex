\documentclass[12pt,draft,twoside]{book}
\usepackage{../../thesis}

\makeindex
\begin{document}

\chapter{Chaos in Li-Yorke sense}

\section{Period Three Implied Chaos}
This section follows the treatment in \citet{li-yorke}, a seminal paper in the early era of chaos theory.

\begin{definition}
  (Chaos in Li-Yorke's sense)
  Let $I$ be an interval and $F: I\to I$ be continuous. $F$ is said to be \textit{Li-Yorke chaotic}, or 
  \textit{chaotic in Li-Yorke's sense} if there is an uncountable set $S \subseteq I$ containing no
  periodic points such that each $p,q \in S$ satisfy
  \begin{equation*}
    \limsup\limits_{n \to \infty}\; \abs{\itr{F}{n}(p) - \itr{F}{n}(q)} > 0
  \end{equation*}
  and
  \begin{equation*}
    \liminf\limits_{n \to \infty}\; \abs{\itr{F}{n}(p) - \itr{F}{n}(q)} = 0.
  \end{equation*}
  \index{Li-Yorke!definition of!chaos}
\end{definition}

The definition says that the two points are separated by some (perhaps very small) constant infinitely many times, and, at the same time, come arbitrarily close to each other infinitely many times (the second condition). 
Thus, even if $p$ and $q$ are very close to each other at some time $t$, after some time, say $t + t_0$, the two points can be far apart.
The main results of \citet{li-yorke} are theorem~\ref{thm:liyorke1} and theorem~\ref{thm:liyorke2}.
Both theorems assume that $F$, a continuous map, has a 3-periodic orbit.

\begin{theorem}
  (Period three implies all periods)
  Let $I$ be an interval and $F: I\to I$ be continuous. Suppose there is a point $a \in I$ for which
  the points
  \begin{equation*}
  b = F(a), \quad c = F(b), \quad d = F(c)
  \end{equation*}
  satisfy
  \begin{equation*}
    d \leq a < b < c.
  \end{equation*}
  Then $F$ has a periodic orbit of every period.
  (Note that the hypothesis is satisfied is $F$ has a periodic orbit of period 3.)
  \label{thm:liyorke1}
\end{theorem}

\begin{theorem}
  (Period three implies Li-Yorke chaos)
  Suppose the same hypothesis as theorem~\ref{thm:liyorke1}.
  Then $F$ is chaotic in Li-Yorke's sense.
  \label{thm:liyorke2}
\end{theorem}

We will prove later that both logistic map and baker's map are Li-Yorke chaotic by showing that the mappings have 3-periodic orbits.
One may be surprised that the innocent condition gives rise to a complex behaviour.
Indeed, there is something special about a 3-periodic orbit.
Theorem~\ref{thm:liyorke1} was proved by Alexander Sarkovskii, a Soviet mathematician, ten years before the publication \citep{sarkovskii}.
Sarkovskii's theorem extends Li-Yorke's theorem~\ref{thm:liyorke1} by proving analogous results for period 5, 7, 9, \ldots, and so on.
Then, Sarkovskii's shows that there exists an ordering of $\mathbb{N}$---called \textit{Sarkovskii's ordering}---such that if a mapping has a $k$-periodic orbit, then it has periodic orbits of \textit{all} periods that come after $k$ in the ordering.
Appendix contains a proof of Sarkovskii's ordering.

Now we proceed to prove theorem~\ref{thm:liyorke1} and theorem~\ref{thm:liyorke2}.
We shall start with proving three lemmas.

\begin{lemma}
  Let $I$ be an interval, $F: I \to \R$ a continuous map.
  For each compact interval $I_1 \subseteq F(I)$, there exists a compact interval $Q \subseteq I$ such that $G(Q) = I_1$.
  \label{lem:liyorke1}
  \begin{proof}
    Suppose $I_1 = [F(p),F(q)]$ for some $p,q \in I$.
    Without loss of generality, we may assume $p < q$, for otherwise, we merely switch the roles of $p$ and $q$ in the proof.
    Let $Q = [r,s]$, where $r$ is the greatest point in $[p,q]$ such that $F(r) = F(p)$, and $s$ is the least point in $(r,q]$ such that $F(s) = F(q)$.
    Since $F$ is continuous, $F(Q) = I_1$.
  \end{proof}
\end{lemma}

\begin{lemma}
  Let $I$ be an interval, and $F: I\to I$ be a continuous map.
  Suppose that for each $n$, $\seq{I_n}{\infty}{n=0}$ is a sequence of compact intervals with $I_n \subseteq I$ and $I_{n+1} \subseteq F(I_n)$.
  Then, there exists a sequence $\seq{Q_n}{\infty}{n=0}$ of compact intervals such that 
  \begin{equation*}
    Q_{n+1} \subseteq Q_{n} \subseteq I_0,
  \end{equation*}
and, for each $n$,
  \begin{equation*}
    \itr{F}{n}(Q_n) = I_n.
  \end{equation*}
  Furthermore, for any $x \in Q \equiv \bigcap\limits_{n=0}^{\infty}Q_n$, we have for each $n$
  \begin{equation*}
    \itr{F}{n}(x) \in I_n.
  \end{equation*}
  \label{lem:liyorke2}
  %%%
  \begin{proof}
    Proof by induction.
    Let $Q_0 = I_0$.
    Clearly, we have $\itr{F}{0}(Q_0) = I_0$.
    For each $n$, define $Q_{n-1}$ as a compact interval such that $\itr{F}{n-1}(Q_{n-1}) = I_{n-1}$.
    Note that $I_n \subseteq F(I_{n-1}) = \itr{F}{n}(Q_{n-1})$.
    Lemma~\ref{lem:liyorke1} applied to $\itr{F}{n}$ on $Q_{n-1}$ proves the existence of a compact interval $Q_n \subseteq Q_{n-1}$ such that $\itr{F}{n}(Q_n) = I_n$.
  \end{proof}
\end{lemma}

\begin{lemma}
  Let $I$ be an interval, and $F: I\to I$ be a continuous map.
  \label{lem:liyorke3}
\end{lemma}
We already know that points in $S$ are not periodic.
However, we may draw an even stronger conclusion.
\begin{theorem}
  (Points in $S$ are not asymptotically periodic)
  For every $p \in S$ and periodic point $q \in I$,
  \begin{equation*}
    \limsup\limits_{n \to \infty}\; \abs{\itr{F}{n}(p) - \itr{F}{n}(q)} > 0.
  \end{equation*}
  \label{thm:liyorke3}
\end{theorem}

We will see that Marotto's definition is equivalent to Li-Yorke's.
Moreover, Marotto's definition is generalizable to $n$-space.

But this doesn't apply to baker's map, because the transformation is not continuous.
\begin{definition}
  (Baker's transformation)
  Let $I = [0,1]$, and $B: I \to I$. Baker's transformation is
  \begin{equation*}
    B(x) = 2x - \ceil{2x}.
  \end{equation*}
\end{definition}

Also, there are mappings  (somewhat pathological) that is chaotic on negligible set.
\begin{proposition}
  Let $I = [0,1]$ and $f: I \to I$ defined as
  \begin{equation*}
    f(x) =
    \begin{cases}
      0 \quad & 0\leq x < \frac{1}{4} \\
      4x - 1 \quad  & \frac{1}{4} \leq x < \frac{1}{2} \\
      -4x + 3 \quad & \frac{1}{2} \leq x < \frac{3}{4} \\
      0 \quad & \frac{3}{4} \leq x \leq 1. \\
    \end{cases}
  \end{equation*}
  For each $x_0 \in I$, $O(x_0)$ converges to 0 almost everywhere.
\end{proposition}

This definition inspired many other definitions of chaos.
$\omega$-chaos and distributional chaos, which I do not discuss in this paper, are some definitions analogous to Li-Yorke's chaos.
Li-Yorke's definition is limited to one-dimensional maps.
In the next chapter, we discuss a definition due to Marotto.
Marotto a similar result was obtained for higher-dimensional systems.

\bibliographystyle{../../bibliography/pjgsm}
\bibliography{../../bibliography/thesis}

\printindex
\end{document}
