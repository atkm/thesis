\documentclass[11pt]{article}
\usepackage{thesis}

\begin{document}

\section{Chaos in Li-Yorke sense}

\subsection{Alexander Sarkovskii}
\begin{definition}
  (Aperiodic orbit)
  The orbit $O(x_0)$ is said to be \textit{aperiodic} if its limit set $L(x_0)$ is not finite.
\end{definition}


The image of a map whose domain is a closed, bounded interval is also a bounded and closed interval.
This implies the Intermediate Value Theorem (IVT), from which we derive the following theorem about
fixed points of a mapping.
\begin{proposition}
  (Fixed point theorem)
  Let $F: [a,b] \to \R$ be continuous and suppose one of the following facts holds
  \begin{itemize}
    \item $F(a), F(b)\in [a,b]$
    \item $a, b\in F([a,b])$.
  \end{itemize}
  Then $F$ has a fixed point in $[a,b]$.
\end{proposition}

The theorem can be extended to show the existence of $p-$periodic points by applying the theorem to
$F^p$. In such case, however, we must be sure that fixed points found are \textbf{not} also fixed by
$F^m$ ($m < p$).
Furthermore, the existence of a $2-$periodic orbit ensures the existence of a fixed point.
\begin{proposition}
  Suppose $F: I\to I$ has a $2-$periodic orbit, $\set{p,q}$ ($p<q$). Then $F$ has a fixed point in $[p,q]$.
\end{proposition}
\begin{proof}
  By definition of a periodic orbit, 
  \begin{equation*}
    F(p) = q > p, \quad F(q) = p < q.
  \end{equation*}
  It follows from the fixed point theorem that there exists a fixed point in $[p,q]$.
\end{proof}

This observation was extended to all natural numbers by Alexander N. Sarkovskii.
\begin{proposition}
  (Sarkovskii)
  Let $F: I\to I$ be continuous. Assume that $F$ has a periodic orbit of period $p$.
  Then $F$ has a periodic orbit of every period, which follows $p$ in the Sarkovskii
  ordering of $N$.
\end{proposition}

\begin{definition}
  (Sarkovskii's ordering, 1964)
  \[
    \begin{array}{ccccccccc}
    3, & 5, & 7, & 9, & 11, & 13, & \ldots & \ldots & \ldots\\
    2\cdot 3, & 2\cdot 5, & 2\cdot 7, & 2\cdot 9, & 2\cdot 11, & \ldots & \ldots & \ldots\\
    2^2\cdot 3, & 2^2\cdot 5, & 2^2\cdot 7, & 2^2\cdot 9, & \ldots & \ldots & \ldots\\
    2^3\cdot 3, & 2^3\cdot 5, & 2^3\cdot 7, & \ldots & \ldots & \ldots\\
    \vdots & \vdots & \vdots & \vdots & \vdots & \vdots & \vdots & \vdots  & \vdots \\
    \\
    \ldots & \ldots & \ldots & \ldots & 2^4, & 2^3, & 2^2, & 2, & 1.
  \end{array}
  \]
  The $n$th row $(n = 0, 1, \ldots)$ contains all integers of the form $2^n\cdot (2m + 1),\; m \in \Z$ in increasing order.
  The last row contains all powers of 2 in decreasing order.
  \label{defn:sarkovskiiordering}
  %\index{Sarkovskii's ordering}
\end{definition}

\subsection{Chaos in Li-Yorke sense}
\begin{definition}
  (Chaos in Li-Yorke sense)
  Let $I$ be an interval and $F: I\to I$ be continuous. $F$ is said to be \textit{Li-Yorke chaotic}, or 
  \textit{chaotic in Li-Yorke's sense} if there is an uncountable set $S \subseteq I$ containing no
  periodic points, which satisfies the following conditions:
  \begin{equation*}
    \limsup\limits_{n \to \infty}\; \abs{F^n(p) - F^n(q)} > 0.
  \end{equation*}
  \begin{equation*}
    \liminf\limits_{n \to \infty}\; \abs{F^n(p) - F^n(q)} = 0.
  \end{equation*}

\end{definition}

\begin{proposition}
  (Period three implies all periods. \cite{li-yorke})
  Let $I$ be an interval and $F: I\to I$ be continuous. Suppose there is a point $a \in I$ for which
  the points
  \begin{equation*}
  b = F(a), \quad c = F(b), \quad d = F(c)
  \end{equation*}
  satisfy
  \begin{equation*}
    d \leq a < b < c.
  \end{equation*}
  Then $F$ has a periodic orbit of every period.
  (Note that the hypothesis is satisfied is $F$ has a periodic orbit of period 3.)
  \label{thm:liyorke1}
\end{proposition}

\begin{proposition}
  (Period three implies Li-Yorke Chaos. \cite{li-yorke})
  Suppose the same hypothesis as Proposition~\ref{thm:liyorke1} for $F$,
  a continuous map.
  Then $F$ is chaotic in Li-Yorke's sense.
\end{proposition}

We already know that points in $S$ are not periodic.
However, we may draw an even stronger conclusion.
\begin{proposition}
  (Points in $S$ are not asymptotically periodic)
  For every $p \in S$ and periodic point $q \in I$,
  \begin{equation*}
    \limsup\limits_{n \to \infty}\; \abs{F^n(p) - F^n(q)} > 0.
  \end{equation*}
\end{proposition}

  We will see that Marotto's definition is equivalent to Li-Yorke's.
Moreover, Marotto's definition is generalizable to $n$-space.

But this doesn't apply to baker's map, because the transformation is not continuous.
\begin{definition}
  (Baker's transformation)
  Let $I = [0,1]$, and $B: I \to I$. Baker's transformation is
  \begin{equation*}
    B(x) = 2x - \ceil{2x}.
  \end{equation*}
\end{definition}

Also, there are mappings  (somewhat pathological) that is chaotic on negligible set.
\begin{proposition}
  Let $I = [0,1]$ and $f: I \to I$ defined as
  \begin{equation*}
    f(x) =
    \begin{cases}
      0 \quad & 0\leq x < \frac{1}{4} \\
      4x - 1 \quad  & \frac{1}{4} \leq x < \frac{1}{2} \\
      -4x + 3 \quad & \frac{1}{2} \leq x < \frac{3}{4} \\
      0 \quad & \frac{3}{4} \leq x \leq 1. \\
    \end{cases}
  \end{equation*}
  For each $x_0 \in I$, $O(x_0)$ converges to 0 almost everywhere.
\end{proposition}

Also, Li-Yorke's definition is limited to one-dimensional maps.
However, the result was extended to higher-dimensional systems by Marotto (1978 ``Snapback repeller'')

\bibliographystyle{plain}
\bibliography{../../bibliography/thesis}
\end{document}
