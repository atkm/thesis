\documentclass[11pt]{article}
\usepackage{thesis}

\begin{document}

\section{Chaos in Li-Yorke sense}
\begin{definition}
  (Aperiodic orbit)
  The orbit $O(x_0)$ is said to be \textit{aperiodic} if its limit set $L(x_0)$ is not finite.
\end{definition}

\begin{proposition}
  (Sarkovskii)
  Let $F: I\to I$ be continuous. Assume that $F$ has a periodic orbit of period $p$.
  Then $F$ has a periodic orbit of every period, which follows $p$ in the Sarkovskii
  ordering of $N$.
\end{proposition}

The image of a map whose domain is a closed, bounded interval is also a bounded and closed interval.
This implies the Intermediate Value Theorem (IVT), from which we derive the following theorem about
fixed points of a mapping.
\begin{proposition}
  (Fixed point theorem)
  Let $F: [a,b] \to \R$ be continuous and suppose one of the following facts holds
  \begin{itemize}
    \item $F(a), F(b)\in [a,b]$
    \item $a, b\in F([a,b])$.
  \end{itemize}
  Then $F$ has a fixed point in $[a,b]$.
\end{proposition}

The theorem can be extended to show the existence of $p-$periodic points by applying the theorem to
$F^p$. In such case, however, we must be sure that fixed points found are \textbf{not} also fixed by
$F^m$ ($m < p$).
Furthermore, the existence of a $2-$periodic orbit ensures the existence of a fixed point.
\begin{proposition}
  Suppose $F: I\to I$ has a $2-$periodic orbit, $\set{p,q}$ ($p<q$). Then $F$ has a fixed point in $[p,q]$.
\end{proposition}
\begin{proof}
  By definition of a periodic orbit, 
  \begin{equation*}
    F(p) = q > p, \quad F(q) = p < q.
  \end{equation*}
  It follows from the fixed point theorem that there exists a fixed point in $[p,q]$.
\end{proof}
This observation was extended to all natural numbers by Alexander N. Sarkovskii.
\begin{definition}
  (Sarkovskii's ordering, 1964)
  
\end{definition}

\begin{proposition}
  (Li, Yorke. 1975)
  Let $I$ be an interval and $F: I\to I$ be continuous. Suppose $F$ has a periodic orbit of period 3.
  Then $F$ has a periodic orbit of every period and there is an infinite set $S$ contained in
  $I$ such that every orbit starting from a point of $S$ is aperiodic.
\end{proposition}

The existence of a periodic point of period 3 for continuous and one-dimensional maps is
equivalent to the existence of a snapback repeller.
\begin{definition}
  (Marotto, Martelli, Snapback repeller)
 $F: R^n\to R^n \in C^1$ in some open ball centered at a stationary state $x_s$
has uncountably many aperiodic and unstable orbits if $\rho(F'(x_s)) > 1$, and
there exists $z_1$ in the unstable manifold $M_U$ of $x_s$ such that
\begin{equation*}
  det((F^m)'(z_1)) \neq 0, \quad F^m(z_1) = x_s.
\end{equation*}
 
  Such $x_s$ is called a snapback repeller.
\end{definition}

But this doesn't apply to baker's map.
\begin{definition}
  (Baker's transformation)
  Let $I = [0,1]$, and $B: I \to I$. Baker's transformation is
  \begin{equation*}
    B(x) = 2x - \ceil{2x}.
  \end{equation*}
  
\end{definition}

Also, Li-Yorke's definition is limited to one-dimensional maps.

\end{document}
