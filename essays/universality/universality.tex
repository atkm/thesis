\documentclass[11pt]{article}
\usepackage{thesis}

\begin{document}
A class of mappings shows a similar behaviour.
When two mappings can be related to each other by a homeomorphism,
the two mappings are the ``same'' dynamically.
For example, suppose a mapping $f$ is conjugate to a mapping $g$.
If $g$ is sensitive to initial conditions, then $f$ is also sensitive
to initial conditions.
If $g$ is topologically transitive, then $f$ is also topologically transitive.
Then it follows that if $g$ is chaotic then $f$ is also chaotic.
Not surprisingly, {\it all} quadratic maps are conjugate to each other.
Feigenbaum found a universal trait of a large class of functions.
A dynamical system can be studied by looking at another dynamical system
that is conjugate to it.
Let us illustrate this by relating the shift map $S$ and the itinerary map $\tau$.
\end{document}
