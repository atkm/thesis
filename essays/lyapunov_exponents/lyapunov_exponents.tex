\documentclass[11pt]{article}
\usepackage{thesis}

\begin{document}
\section{Lyapunov Exponents}
The technique of Lyapunov exponents was developed by Alexander M. Lyapunov
to study stability of solutions for systems of differential equations.

\subsection{Invariants}
The critical feature of the invariants is that they are not sensitive to initial conditions
or small perturbations of an orbit, while individual orbits of the system are exponentially
sensitive to such perturbations. (Abarbanel, p1334)
\subsection{In One Dimension}
\begin{definition}
  (Lyapunov number)
  Given a map $F$ and a $p-$periodic point $x_p$, the Lyapunov number of the point $N(x_p)$ is
  defined to be the spectral radius of the Jacobian matrix of the map.
  \begin{equation*}
    N(x_p) = \left[ \rho\left( F^p(x_p) \right) \right]^{1/p}
  \end{equation*}
  \label{def:lyapnum}
\end{definition}

\begin{definition}
  (Lyapunov Number, alternative)
  Let
  \begin{equation*}
    O(x_1) = \set{x_1, x_2, x_3, \ldots}
  \end{equation*}
  and define matrices
  \begin{equation*}
    A_1 = \frac{d}{dx}F(x_1),\; A_2 = \frac{d}{dx}F(x_2),\; \ldots,\; A_n = \frac{d}{dx}F(x_n).
  \end{equation*}
  Then consider their spectral radii
  \begin{equation*}
    \rho_1(x_1), \rho_2(x_1),\; \ldots,\; \rho_n(x_1).
  \end{equation*}
  Then the alternate version of the Lyapunov number $N(x_1)$ of $O(x_1)$ is defined as 
  \begin{equation*}
    N(x_1) = \lim\limits_{n \to \infty} (\rho_n(x_1))^{1/n}.
  \end{equation*}
\end{definition}
The alternative definition can be shown to be equivalent to the first definition.

Lyapunov exponent is computationally preferable to the Lyapunov number.
\begin{definition}
  (Lyapunov exponent)
  \begin{equation*}
    \Lambda(x_1) = \log N(x_1).
  \end{equation*}
  In particular,
  \begin{equation*}
    \Lambda(x_1) = \lim\limits_{n \to \infty} \paren*{\log\abs{F'(x_n)} + \log\abs{F'(x_{n-1})} + \cdots + \log\abs{F'(x_1)}}
  \end{equation*}
\end{definition}

In some cases, this yet another alternate definition is more convenient.
This version ignores transient states.
\begin{definition}
  (Lyapunov Number, yet another alternative)
  \begin{equation*}
    N(x_1, k) = N(x_k)
  \end{equation*}
  Then the Lyapunov number is defined to be
  \begin{equation*}
    N(x_1) = \lim\limits_{k \to \infty} N(x_1,k).
  \end{equation*}
\end{definition}

It can be shown that the Lyapunov exponent of a map is (almost everywhere) equivalent to the Lyapunov exponent of any orbit of a map (Lyapunov exponent is independent of orbit).


\subsection{Lyapunov Exponents and Conjugacy}
Lyapunov number of one-dimensional dynamical systems that are conjugate is the same
as long as $h'(x) \neq 0$ and
\begin{equation*}
  \lim\limits_{n\to \infty} \frac{\log\abs{h'(x_n)}}{n} = 0.
\end{equation*}

By the definition of conjugacy,
\begin{equation*}
  h'(x_{n+1})F'(x_n)\ldots F'(x_1) = G'(y_n)\ldots G'(y_1)h'(x_1).
\end{equation*}
Then 
\begin{align*}
  \Lambda(x_1) &= \lim\limits_{n\to \infty} \frac{1}{n}\paren{\log\abs{F'(x_n)} + \cdots + \log\abs{F'(x_n)} } \\
  &= \lim\limits_{n\to \infty} \frac{1}{n}\left(\log\abs{F'(x_n)} + \cdots + \log\abs{F'(x_n)} + \log\abs{h'(x_{n+1})}\right) \\
  &= \lim\limits_{n\to \infty} \frac{1}{n} \paren*{ \log\abs{G'(x_n)} + \cdots + \log\abs{G'(x_n)} + \log\abs{h'(x_1)} } \\
  &= \lim\limits_{n\to \infty} \frac{1}{n} \paren*{ \log\abs{G'(x_n)} + \cdots + \log\abs{G'(x_n)} } \\
  &= \Lambda(y_1),
\end{align*} 
as desired.

\end{document}
