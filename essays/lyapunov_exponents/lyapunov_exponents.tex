\documentclass[11pt]{article}
\usepackage{thesis}

\begin{document}
\section{Lyapunov Exponents}
The technique of Lyapunov exponents was developed by Alexander M. Lyapunov
to study stability of solutions for systems of differential equations.

\begin{definition}
  (Lyapunov number)
  Given a map $F$ and a $p-$periodic point $x_p$, the Lyapunov number of the point $N(x_p)$ is
  defined to be the spectral radius of the Jacobian matrix of the map.
  \begin{equation*}
    N(x_p) = \left[ \rho\left( F^p(x_p) \right) \right]^{1/p}
  \end{equation*}
  \label{def:lyapnum}
\end{definition}

\begin{definition}
  (Lyapunov Number for an arbitrary orbit)
  Let
  \begin{equation*}
    O(x_1) = \set{x_1, x_2, x_3, \ldots}
  \end{equation*}
  and define matrices
  \begin{equation*}
    A_1 = \frac{d}{dx}F(x_1),\; A_2 = \frac{d}{dx}F(x_2),\; \ldots,\; A_n = \frac{d}{dx}F(x_n).
  \end{equation*}
  Then consider their spectral radii
  \begin{equation*}
    \rho_1(x_1), \rho_2(x_1),\; \ldots,\; \rho_n(x_1).
  \end{equation*}
  
\end{definition}

\end{document}
