\documentclass[10pt,twoside,draft]{book}
\usepackage{../../thesis}
\graphicspath{ {../../images/} }

\begin{document}
We have looked at several definitions of chaos in order to seek explanations of complex dynamics present in systems such as the logistic map and outer billiards.
Li-Yorke's and Wiggins's definitions seem to be the most reasonable definitions of chaos, since they are the weakest among the definitions that we have considered.
Although neither of them implies the other in a compact metric space, the two consist of similar conditions.
The non-asymptotic relation and sensitivity embody the same idea that a small change in the initial state has significant effects on later states.
Transitivity means that any two points in the space can come arbitrarily close to each other, which is what the proximal relation is.
We see an intricacy of mathematics in action: \wig and \liy, despite representing the same idea, are not equivalent.
The inequivalence may also be due to the other aspect in which the two differ.
While \wig requires the existence of a compact, invariant subset with the prescribed conditions, \liy asks for an uncountable number of pairs with the properties. 
\citet{mai}'s result that transitivity, sensitivity, and the existence of a periodic point imply Li-Yorke's chaos indicates that there is a delicate relation between \liy and \wig.


Other definitions that we discussed should not be dismissed, for they provide different perspectives to think about chaos.
Symbolic dynamics, not only a means of defining chaos but also a tool to study complex dynamics, shows us what a bare-boned chaos would look like.
The proof that the full one-sided shift is chaotic in Devaney's sense served as an abstract, yet simple description of what it means for a system to be sensitive to initial conditions.
Topological entropy, which measures the complexity of the orbit structure of a given mapping, can be used to compare complexities of different systems.
%all chaotic systems are complicated enough that the degree of complexity may not be all that important.
Moreover, positive topological entropy implies sensitivity, an indispensable element of chaos.
This fact suggests that associating a geometrically complicated dynamics with chaos is, though not rigorous, not unreasonable.


Proving that a system is chaotic practically tells us nothing, because when we encounter a complicated system, we know it.
Interacting with chaotic system numerically gives us a good sense of the complexity of a chaotic system.
What we are looking for is more than just a theoretical justification for saying, ``hey, this system is complicated.''
Through our exploration of definitions of chaos, we learned what elements are commonly present in chaotic dynamics.
Finding an appropriate definition is a prerequisite for the next steps, which may be developments of a control theory of chaotic systems \citep{openproblems, chaos-frontiers}, data analysis techniques of chaotic time-series \citep{kantz-schreiber, sprott, abarbanel}, or other applications, such as digital communications \citep{chaos-communication}.
The result of our comparisons in a compact interval indicates that it does not really matter which definition we use--they are more or less the same, with the exception of Li-Yorke's definition.
Therefore, for one-dimensional systems, we can use any of the definitions to characterize chaotic systems, while we must be aware that there are outliers, which are chaotic in Li-Yorke's sense but not others \citep{smital, misiurewicz1}.
For systems defined for more general metric spaces, however, the situation is not as simple (Figure~\ref{fig:chaos-metric}).
As mentioned earlier, \wig and \liy are not equivalent.
Both \dev and \akm imply \liy, but neither of them implies the other.
\blcp, which is stronger than \akm, has the same relation with \dev: neither \blcp nor \akm implies the other.


Finally, let us remark on some problems that we left open as well as possible directions to extend the work presented in this exposition.
We do no know whether positive topological entropy implies \wig or \blcp.
A closely related question is whether positive topological entropy implies transitivity. 
Since positivie topological entropy implies sensitivity, if this is true, then \akm implies \wig.
Chaotic systems are known to exhibit statistical properties.
Lacosta-Yorke: "On the existence of invariant measures for piecewise monotonic transformations"
(aka "chaos implies statistical regularity")
-> Li found how to compute the invariant measure in "Finite approximation for the Frobenius-Perron operator"
(Chaos Avant-Garde)
For example, the logistic map is known to po \citep{sternberg}.
Although we concentrated on topological dynamics, it is entirely possible that measure-theoretic dynamics or differentiable dynamics may be the right setting to define chaos.
Stephen Silverman (the author of Theorem~\ref{thm:silverman}) asks whether the following example exhibits sensitivity.\footnote{Personal communication, March 2013}
\begin{example}
  Let $X$ be a perfect metric space, and $F: X \to X$ be a continuous map.
  Suppose $X$ can be written as $X = O(x) \cup \set{p}$, where $O(x)$ is the orbit of some point in the space and $p$ is a fixed point.
  Does $F$ exhibit sensitivity on $X$?
  $\square$
\end{example}
If this system turns out to be sensitive, it would be the simplest chaotic system.

\bibliographystyle{../../bibliography/pjgsm}
\bibliography{../../bibliography/thesis}

\end{document}
