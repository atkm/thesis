\documentclass[10pt,twoside,draft]{book}
\usepackage{../../thesis}
\graphicspath{ {../../images/} }

\begin{document}
We have looked at several definitions of chaos in order to seek explanations of complex dynamics present in systems such as the logistic map and outer billiards.
Li-Yorke's and Wiggins's definitions seem to be the most reasonable definitions of chaos, since they are the weakest among the ones that we considered.
Although neither of them implies the other in a compact metric space, the two consist of similar conditions.
The non-asymptotic relation and sensitivity embody the same idea that a small change in the initial state has significant effects on later states.
Transitivity means that any two points in the space can come arbitrarily close to each other, which is what the proximal relation represents.
We see an intricacy of mathematics in action: \wig and \liy, despite representing the same idea, are not equivalent.
The inequivalence may also be due to the other aspect in which the two differ.
While \wig requires the existence of a compact, invariant subset with the prescribed conditions, \liy asks for an uncountable number of pairs with the properties. 
\citet{mai}'s result that transitivity, sensitivity, and the existence of one periodic point imply Li-Yorke's chaos indicates that there is a subtle relation between \liy and \wig.


Other definitions that we discussed should not be dismissed, for they provide different perspectives to understand chaos.
Symbolic dynamics, not only a means of defining chaos but also a tool to study complex dynamics, shows us what a bare-boned chaos would look like.
The proof that the full one-sided shift is chaotic in Devaney's sense serves as an abstract, yet simple description of what it means for a system to be sensitive to initial conditions.
Topological entropy, which measures the complexity of the orbit structure of a given mapping, can be used to compare complexities of different systems.
%all chaotic systems are complicated enough that the degree of complexity may not be all that important.
Moreover, positive topological entropy implies sensitivity, an indispensable element of chaos.
This fact suggests that associating a geometrically complicated dynamics with chaos is, though not rigorous, not unreasonable.


Along the way, we studied several systems and proved that they are chaotic in some senses.
Proving that a system is chaotic, however, tells us virtually nothing, because when we encounter a complicated system, we know it.
Interacting with a chaotic system through numerical simulations gives us a good sense of the complexity of a chaotic system, and finding a mathematical justification to support the observation is, though an interesting theoretical problem, practically unnecessary.
What we really gained from this exposition is a better understanding of the elements that compose chaotic dynamics and their relations with each other.
Finding an appropriate definition, or knowing what causes chaos, is a prerequisite for the next steps, which may be developments of a control theory of chaotic systems \citep{openproblems, chaos-frontiers}, techniques for time-series analysis \citep{kantz-schreiber, sprott, abarbanel}, or other applications, such as digital communications \citep{chaos-communication}.


Finally, let us remark on some problems that we left open as well as possible directions to extend the work presented in this exposition.
We do no know whether positive topological entropy implies \wig or \blcp.
A closely related question is whether positive topological entropy implies transitivity. 
Since positive topological entropy implies sensitivity, if this is true, then \akm implies \wig.
Although we concentrated on topological dynamics, it is entirely possible that measure-theoretic dynamics or differentiable dynamics \citep{smale} may be the right setting to define chaos.
For instance, chaotic systems are known to exhibit statistical properties \citep{lasota}.
The orbits of the logistic map, even though they seem to randomly move around the closed interval $[0,1]$, are known to follow a certain statistical distribution \citep{sternberg}.
Yorke calls this property "chaos implies statistical regularity," imitating the title of his well-known paper \citep{ueda-abraham}.
%Furthermore, Li found how to compute the invariant measure in "Finite approximation for the Frobenius-Perron operator"
One may be interested in further studies of sensitive dependence on initial conditions, the hallmark of chaos.
Steve Silverman (the author of Theorem~\ref{thm:silverman}) asks whether the following example exhibits sensitivity.\footnote{Personal communication, March 2013}
\begin{example}
  Let $X$ be a perfect metric space, and $F: X \to X$ be a continuous map.
  Suppose $X$ can be written as $X = O(x) \cup \set{p}$, where $O(x)$ is the orbit of some point in the space and $p$ is a fixed point.
  Does $F$ exhibit sensitivity on $X$?
  $\square$
\end{example}
If this system turns out to be sensitive, it would serve as an example of the simplest chaotic system.

\bibliographystyle{../../bibliography/pjgsm}
\bibliography{../../bibliography/thesis}

\end{document}
