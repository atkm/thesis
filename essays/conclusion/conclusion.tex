\documentclass[12pt,twoside,draft]{book}
\usepackage{../../thesis}
\graphicspath{ {../../images/} }

\makeindex
\begin{document}
\chapter{Conclusion}
We have concentrated on the comparisons of definitions, but the field of dynamical systems is more than just that.
For example, one may be interested in finding what conditions imply \sdic.
We proved that \dpp and \tt are sufficient conditions for \sdic, but can we use a weaker condition to derive \sdic?
Are there other conditions that imply \sdic?
Do topologically transitive maps have positive topological entropy?
One could define chaos for measure-theoretic dynamics \citep{downarowicz} or differentiable dynamical systems \citep{ruellebook}.
One could also study how complicated dynamics arise.
For example, period doubling route to chaos and intermittency route to chaos \citep{sternbergbook}

\section{Some Open Problems}
\begin{enumerate}
  \item Steve's problem
  \item Does topological transitivity imply positive topological entropy?
\end{enumerate}
 
\bibliographystyle{../../bibliography/pjgsm}
\bibliography{../../bibliography/thesis}

\printindex
\end{document}

