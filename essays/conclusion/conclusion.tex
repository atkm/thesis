\documentclass[10pt,twoside,draft]{book}
\usepackage{../../thesis}
\graphicspath{ {../../images/} }

\begin{document}
In a compact metric space, Li-Yorke's and Wiggins's definitions, neither of which implies the other, are the weakest of the definition that we consider.
The conditions in the two least demanding definitions guarantee complex dynamics, and these seem to be the most reasonable definitions of chaos.
\citet{mai}'s result that transitivity, sensitivity, and the existence of a periodic point imply Li-Yorke's chaos shows that there is an intricate relation between \liy and \wig.
Other definitions that we considered cannot be ignored, however, for they provide different perspectives for thinking about chaos..
Symbolic dynamics shows us what chaos looks likes in its bare-boned form.
It is also an useful tool to study complicated dynamics.
AKM's chaos tells us the complexity of orbit structure is a good measure of chaos.
We need to know whether positive topological entropy implies transitivity  Wiggins's or Devaney's chaos.
Sensitivity follows from positive topological entropy.
Topological entropy also gives us a numerical measure of the complexity of a system, allowing us to compare complexities of different systems (although all chaotic systems are complicated enough that the degree of complexity may not be all that important).
Although we concentrated on topological dynamics, it is entirely possible that measure-theoretic dynamics or differentiable dynamics may be the right setting to define chaos.


The conditions that we discussed are criteria for a system to be complicated, which do tells us nothing.
When we see a complex system, there is no need to prove that it is complex.
Finding an appropriate definition connects to the next steps, which may be control of systems, data analysis, or other applications, such as \citet{chaos-communication} and \citet{openproblems, chaos-frontiers}.
The result of our comparisons in a compact interval indicates that it does not really matter which definition we use--they are more or less the same, with the exception of Li-Yorke's definition.
Therefore, for one-dimensional systems, we can use .
But we must be aware

\bibliographystyle{../../bibliography/pjgsm}
\bibliography{../../bibliography/thesis}

\end{document}
