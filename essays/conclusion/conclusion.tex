\documentclass[10pt,twoside,draft]{book}
\usepackage{../../thesis}
\graphicspath{ {../../images/} }

\begin{document}
We have looked at several definitions of chaos in order to seek explanations of complex dynamics seen in systems such as the logistic map and outer billiards.
Li-Yorke's and Wiggins's definitions seem to be the most reasonable definitions of chaos, since they are the weakest among the definitions that we have considered.
Although neither of them implies the other in a compact metric space, the two consist of similar conditions.
The non-asymptotic relation and sensitivity embody the same idea that a small change in the initial state has a significant effect on later states.
Transitivity means that any two points in the space can come arbitrarily close to each other, which is what the proximal relation is.
We see the intricacy of mathematics in action: \wig and \liy, despite representing the same idea, are not equivalent.
The inequivalence could be due to the other aspect in which the two differ.
While \wig requires the existence of a compact, invariant subset with the prescribed conditions, \liy asks for an uncountable number of pairs with the properties. 
\citet{mai}'s result that transitivity, sensitivity, and the existence of a periodic point imply Li-Yorke's chaos indicates that there is a delicate relation between \liy and \wig.


Other definitions that we considered should not be dismissed, for they provide different perspectives to think about chaos.
Symbolic dynamics, not only a means of defining chaos but also a tool to study complex dynamics, shows us what a bareboned chaos would look like.
The full one-sided shift is the simplest example of systems that are both transitive and sensitive.
AKM's definition by topological entropy tells us the complexity of orbit structure is a good measure of chaos.
Sensitivity follows from positive topological entropy.
Topological entropy also gives us a numerical measure of the complexity of a system, allowing us to compare complexities of different systems (although all chaotic systems are complicated enough that the degree of complexity may not be all that important).
Although we concentrated on topological dynamics, it is entirely possible that measure-theoretic dynamics or differentiable dynamics may be the right setting to define chaos.

The conditions that we discussed are criteria for a system to be complicated, which do tells us nothing.
When we see a complex system, there is no need to prove that it is complex.
Finding an appropriate definition connects to the next steps, which may be control \citep{openproblems, chaos-frontiers}, data analysis, or other applications, such as digital communications \citep{chaos-communication}.
The result of our comparisons in a compact interval indicates that it does not really matter which definition we use--they are more or less the same, with the exception of Li-Yorke's definition.
Therefore, for one-dimensional systems, we can use .
But we must be aware

Open problems
We need to know whether positive topological entropy implies transitivity  Wiggins's or Devaney's chaos.

\bibliographystyle{../../bibliography/pjgsm}
\bibliography{../../bibliography/thesis}

\end{document}
