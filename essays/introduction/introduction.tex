\documentclass[10pt,twoside,draft]{book}
\usepackage{../../thesis}
\graphicspath{ {../../images/} }

\begin{document}
Chaos is a natural phenomenon commonly seen in nonlinear systems.
Edward Lorentz is often regarded as the first discoverer of chaos.
In his seminal article ``Deterministic Nonperiodic Flow'', Lorentz discusses the chaotic behavior of a system of three differential equations, which were reduced from the original twelve equations, which were meant to be a simplified model for the dynamics of the atmosphere.
\begin{figure}[ht]
  \centering
  \includegraphics[width=0.8\textwidth]{golden_lorentz_attractor}
  \caption{The Lorentz Attractor}
  \label{fig:lorentz}
\end{figure}
He found extremely complicated behaviors through his numerical simulations.
The plot shown in Figure~\ref{fig:lorentz} is the orbit of an arbitrary point in the space.
The orbit moves towards the butterfly-shaped attractor.
Moreover, as an orbit moves closer and closer to the attractor, it never crosses itself.
The attractor does not have a simple structure that we might usually expect.
It is known to have a fractal dimension.

Yoshisuke Ueda, though lesser known, was also one of the first scientists to recognize chaotic behaviors.
In 1961 (a sketch of the ``Japanese Attractor''), while Lorentz's paper was published in 1963.
David Ruelle, a prominent French chaos researcher, introduced Ueda in \citep{ruelle}.

%%%
While Lorentz's discovery of chaos was in a computer simulation, Ueda's was in an analog computer, a physical system.
\citet[p.383]{campbell}
\begin{quotation}
  analog computeres solve coupled nonlinear equations by mimicking a physical system.
  The error properties of a particular digital integration scheme need not ever be considered.
  Indeed, within the range of accuracy limited by the tolerance of the components, the unsystematic errors caused by the thermal fluctuations and electronic noise in an analog simulation can actually be useful; this is the case, for example, in qualitative studies of chaotic dynamical systems.
% Crutchfiled, Farmer, Huberman 'Fluctuations and simple chaotic dynamics'
% Crutchfield, Packard 'Symbolic dynamics of one dimensional maps: Entropies, finite precision, and noise'
  Specifically, these fluctuations obliterate the detailed fine structure found in the mathematical description of chaos and thus effectively mimic the coarse-grained behavior that is observed in actual physical experiments in, for example, convecting fluids or nonlinear electronic circuits.
\end{quotation}
%%%
Besides Lorentz and Ueda, Albert Libchaber experimentally confirmed a particular type of chaos using Helium \citep{gleick}.
Also, the logistic map, which we present in the first chapter, is used to model population dynamics.
NH3 laser is known to show chaotic behaviors \citep{kantz-schreiber}.
B-Z reaction
The outer billiards system that we study in Chapter~\ref{chap:billiards} is a model of celestial mechanics.
The goal of this exposition is to mathematically characterize these ``chaotic'' behaviors, or, in other words, to find the definition of chaos.


The first step in defining chaos is to \textit{recognize it}.
Laplacian determinism 
\footnote{``We may regard the present state of the universe as the effect of its past and the cause of its future. An intellect which at a certain moment would know all forces that set nature in motion, and all positions of all items of which nature is composed, if this intellect were also vast enough to submit these data to analysis, it would embrace in a single formula the movements of the greatest bodies of the universe and those of the tiniest atom; for such an intellect nothing would be uncertain and the future just like the past would be present before its eyes.''
  (Pierre Simon Laplace, \textit{A Philosophical Essay on Probabilities})}
\citep{stone} and the belief that linearization can give a good approximation for any system had precluded from acknowledging

\begin{quotation}
As I look back, I feel that after those long exhausting vigils in front of the analog computer, staring at its output, chaos had become a totally natural, everyday phenomenon in my mind.
People call chaos a new phenomenon, but it has always been around.
There's nothing new about it--only people did not notice it.
\citep[p.27]{ueda-abraham}
\end{quotation}


We need a definition that complies with our intuition about chaos.
One can imagine that mathematical definitions, such as those of limit and continuity,
by the complicate appearances of attractors.
The definition of chaos by \textit{fractal dimension} has such an attitude.
If the orbit looks complicated, the system is called chaotic.
A step forward from defining chaos to be a dynamics with complicated appearance is to give additional requirements to it.
\citet{ruelle} defines ``strange attractors.'' 
A strange attractor is, in short, an attractor with a fractal dimension on which the dynamics exhibits sensitive dependence on initial conditions. \citep[p.11]{ott1994}
Boris Chirikov and Felix Izrailev, Russian physicists, famously said that ``a strange attractor seems strange only to a stranger.'' \citep{lorentzbook}
Labeling a system as ``chaotic'' shows our ignorance about the system.


Motivations to study chaos come also from practical purposes.
Chaos time series obtained from a non-linear system could not be analyzed in the same way as linear systems are analyzed.
Sensitive dependence on initial conditions has an interesting interplay with noises, which are inevitable products in experimental settings.
\citet[p.51]{kantz-schreiber}
\begin{quotation}
  Noise reduction means that one tries to decompose a time series into two components, one of which supposedly contains the signal and the other one contains random fluctuations.
  Thus we always assume that the data can be thought of as an additive superposition of two different components which have to be distinguishable by some objective criterion.
  The classical statistical tool for obtaining this distinction is the power spectrum.

  Random noise has a flat, or at least a broad, spectrum, whereas periodic or quasi-periodic signals have sharp spectral lines.
  After both components have been identified in the spectrum, a Wiener filter can be used to separate the time series accordingly.
  This approach fails for deterministic chaotic dynamics because the output of such systems usually leads to broad band spectra itself and thus possesses spectral properties generally attributed to random noise.
  Even if parts of the spectrum can be clearly associated with the signal, a separation into signal and noise fails for most parts of the frequency domain.
  Chaotic deterministic systems are of particular interest because the determinism yields an alternative criterion to distinguish the signal and the noise. 
\end{quotation}
Although a broad Fourier power spectrum is a first indication of chaotic behavior, it alone does not characterize chaos. \citep{abarbanel}


In 1986, the Royal Society held an international conference on chaos and defined ``chaos'' as ``stochastic behavior occurring in a deterministic system.'' \citep{stewart}
The two key aspects of chaos are \citep[p.31]{abarbanel,ott1994} %abarbanel: p379, ott: p31.
%``[\ldots] two key aspects of chaos are the stretching of infinitesimal displacements and the complex orbit structure in the form of a vast variety of possible orbits.'' 
\begin{enumerate}
  \item the stretching of infinitesimal displacement (sensitive dependence on initial condition) 
  \item complex orbit structure
\end{enumerate}
\citet[p.104]{sprott} says that ``there is no universally accepted definition of chaos,'' but ``most experts would concur that chaos is the ''aperiodic, long-term behavior of a bounded, deterministic system that exhibits sensitive dependence on initial conditions.''
\citet[p.103]{berge}:
\begin{quotation}
  Given that no precise scientific definition exists for the noun ``chaos'' or for the adjective ``chaotic,'' we will consider these words to be synonymous with certain typical properties.
  We will say that a dynamical regime is chaotic if its power spectrum contains a continuous part---a broad band---regardless of the possible presence of peaks.
  Or else we may use the criterion that the autocorrelation function of the time signal has finite support, i.e. that it goes to zero in a finite time.
  In either case, the same concept is involved: the loss of memory of the signal with respect to itself.
  Consequently, knowledge of the state of the system for an arbitrarily long time does not enable us to predict its later evolution.
  Essentially, this means that we are making unpredictability the quality which defines chaos.
  The characterization is pragmatic; it lacks rigor and contains unavoidable ambiguities.
  The boundary separating predictability from unpredictability is unclear, leaving open questions such as:
  - On what time scale must the flow be predictable?
  - How precise must the prediction be?
  - Can we allow a statistical prediction?
  The distinction between the theoretical and practical impossibility of prediction is also problematic.
  But given the current state of knowledge, it would be premature to try to do better.
\end{quotation}
Listening to the experts' opinions, we have found three things: complicated orbits, sensitive dependence on initial conditions, and that there is no universally accepted definition of chaos.

We are looking for invariants of a topological dynamical system that characterizes chaos.
The critical feature of the invariants is that they are not sensitive to initial conditions or small perturbations of an orbit, while individual orbits of the system are exponentially sensitive to such perturbations. \citep{abarbanel} %p1334
Dimension is an invariant of dynamical systems.
\textit{Lyapunov exponents} is the other common invariant that characterizes chaos.
The technique of Lyapunov exponents was developed by Alexander M. Lyapunov to study stability of solutions for systems of differential equations.
\begin{quote}
  In the same way power spectrum with a sharp peak characterizes a linear system, invariants such as Lyapunov exponents, fractal dimension, and (KS or topological) entropy characterize nonlinear systems.
  Each orbit is sensitive to initial conditions; however, these invariants are preserved under small perturbations, which are prevalent in real-life situations.
\end{quote}
Although positive Lyapunov exponent is commonly regarded as the definition of chaos \citep{kantz-schreiber} in physics, we do not discuss it in detail, since the Lyapunov exponent is only defined for differentiable mappings.
Since our focus is on topolgical dynamics, we do not go into a great detail explaning Lyapunov exponent.


Finally, let us have a few words on the word ``chaos.''
\citet[p.24]{ueda-abraham}:
\begin{quotation}
  As an academic term, I do not find the word 'chaos' very appropriate.
  But what shall we call it then?
  My proposal has been ``randomly transitional phenomena''; I will explain this below.
  The characteristics of chaos in a physical system can be summarized as follows:
  \begin{itemize}
    \item Random phenomena that occur in deterministic systems.
    \item Random phenomena whose short-term behavior is predictable.
    \item Random phenomena whose long-term behavior is unpredictable.
    \item Although the phenomena are irregular and unpredictable, chaos does have a definite structure.
  \end{itemize}
  The original meaning of chaos, I feel, is a ``total disorder and ultimate unpredictability.''
  But as scientific terminology, the word ``chaos'' seems to overemphasize the unpredictability alone.
  [\ldots]
  Even so I have to use the word ``chaos'' here. [\ldots]
  It is a concise expression which has already filtered into people's minds, and therefore I have decided it is rather pointless to resist it.
\end{quotation}
%%%
Ueda's attitude is the one that we adopt in the exposition.
We do not think that a chaotic map is named so not because it is chaotic in the usual sense.
We think that finding systems that can be labeled as chaotic should the first step in developing better understandings of those mappings with complicated behaviors.
% Li and Yorke's essay in Chaos Avant-Garde: on how chaos developed over a decade since Poincare (202)
\begin{quotation}
  He [Robert May] adopted our [Li and Yorke] use of 'chaos' as a mathematical term, and \textit{Period three implies chaos} therefore began to attract considerable worldwide attention by his strong advocacy in talks and papers.
  \ldots
  As use of the word 'chaos' spread, it became a word people loved to hate: they didn't have a better word but didn't like chaos.
  \citep[p.205]{ueda-abraham}
\end{quotation}
Our goal is to find the sufficient conditions that give rise to complex orbit structures.
%%%
Robert May, a population biologist who is known for studying chaotic behaviors of the logistic map, popularized the term in his articles \citeyearpar{may1,may2}.
Although the term ``chaos'' is commonly seen in literature nowadays, back when Li and Yorke's paper was being published, ``chaos'' was not an often used word in academic papers.
Some editors considered the title of their paper to be inappropriate for a language used in an academic article.
However, some authors admit that the liberal choice of word has to some extent contributed to the popularity that chaos theory currently enjoys \citep[``Exploring Chaos on an Interval'']{ueda-abraham}.
%%%

\subsection*{Outline Of The Exposition}
Chapter~\ref{chap:logistic} and Chapter~\ref{chap:billiards} and, we present a system that seems to be chaotic.
In Chapters~\ref{chap:devaney} and \ref{chap:liyorke}, we will see the two most often talked about definitions of chaos.
In Chapter~\ref{chap:symbolic}, symbolic dynamcis.
In Chapter~\ref{chap:entropy}, topological entropy, which measures the complexity of the orbit structure.

Although chaotic dynamical systems is a large field, the field of dynamical systems in general is even larger.
In dynamical systems, given a system, one asks whether the system is stable or has periodic elements, and so on.
Asking whether a system is chaotic, is, in some sense, a sloppy way of thinking about systems.
To label a system as ``chaotic'' and avoid asking further questions is to neglect the intricacies that a system may have.
Characterization of a system as a chaotic system should be a beginning of the study of systems.
That is, chaotic systems may perhaps be further divided into different categories, and could be understood better.
\citet{devaney}, for example, discusses different ``routes'' to chaos.
As we will see, there are systems that are chaotic in some sense but not so in others.
The goal of this exposition is to find an appropriate definition of chaos.
However, it may be possible that different deterministic systems are unpredictable in different senses.
I hope to 

\bibliographystyle{../../bibliography/pjgsm}
\bibliography{../../bibliography/thesis}

\end{document}
