% About Ueda
\footnote{Doubtful of Ueda's result, Ueda's mentor did not allow publication of Ueda's result until 1970, seven years after the publication of ``Deterministic Nonperiodic Flow.''
  Because of the strict hierarchical structure of the Japanese academia, the first author of Ueda's paper, ``On the Behavior of Self-Oscillatory Systems with External Force'', is his mentor.%\cite[p89]{sprott}
  ``He was the emperor of his laboratory, and yet outwardly he was a mild-mannered gentleman.
  I believed at that time that his was the most feudalistic of any laboratory in the world, and the wall of his authority was impenetrable during his reign, and I still believe it.
  But because of that, we were not swept away by worldly concerns and could concentrate on our research, being faithful to our own ideas.''
  ``After my report, at any rate, Prof. Urabe admonished me personally.
  'What you saw was simply the essence of quasi-periodic oscillations,' he said.
  'You are too young to make conceptual observations.'
% 君が見たのは単に概周期振動にすぎない。概念的な所見を述べるのは、若い君のやることじゃない。
% Your result is no more than an almost periodic oscillation. Don't form a selfish concept of steady states. \citep[p.141]{gleick} : from 'Random Phenomena Resulting from Nonlinearity in the System Described by Duffing's Equation' (1985) in its postscript
  The existence of random oscillations (chaos) was so obvious in my mind, that the negative comment did not crush me.
  Even so, I was deeply disappointed that no one understood it no matter how hard I tried to explain.\citep[p47]{ueda-abraham}
  Ueda left the following remark: ``Chaos, though common phenomena in the nature, has been dismissed because of the difficulty to grasp its full notion.''\citep[p533]{gleick}
%「カオス現象は、われわれが、日常、目にしているありふれた実在の自然現象であるにもかかわらず、その概念把握の困難さのために、かっては見過ごされてきた」
Either way, vigorous research activities on chaos only started in the 70's.}
%``I thought vaccum tubes were broken or something---a common issues with computers of those days.'' \citep{lorentzbook} 
% Ueda: 最初はアナログコンピュータが故障したのかと思った。しかしすぐに、いやそんなことはないと悟った (in Chaos Avant-Garde?)
%私はすぐに真空管が弱ったか何かの、よくあるコンピュータトラブルを疑ったが、修理を頼む前に、どこで間違いが起ったかだけでも調べてみることにした

% Baker's Map
%I shall briefly describe a two-dimensional map called \textit{the baker's map}, which will facilitate our understandings of Lyapunov exponents and fractal dimension.
%Also, the baker's map will be frequently used to illustrate definitions of chaos in later chapters.
%Familiarity with this map would be helpful throughout the rest of the paper.
%\begin{definition}
%  Let $I = [0,1]$, and $0 \leq \alpha, \beta \leq 1$.
%  The baker's map $B_{\alpha, \beta, \gamma, \delta}: \R^2 \to \R^2$ is defined as
%  \begin{equation*}
%    B_{\alpha, \beta, \gamma, \delta}(x,y) = (BX_{\alpha, \beta, \gamma, \delta}(x), BY_{\alpha, \beta, \gamma, \delta}(y)),
%  \end{equation*}
%  where
%  \begin{equation*}
%    BX_{\alpha, \beta, \gamma, \delta}(x) =
%     \begin{cases}
%      \alpha x  &\mbox{ for } y < \gamma \\
%      (1 - \beta) + \beta x  &\mbox{ for } y \geq \gamma,
%    \end{cases}
%  \end{equation*}
%  and
%  \begin{equation*}
%    BY_{\alpha, \beta, \gamma, \delta}(x) =
%     \begin{cases}
%       y/\gamma  &\mbox{ for } y < \gamma \\
%       (y - \gamma)/ \delta  &\mbox{ for } y \geq \gamma.
%    \end{cases}
%  \end{equation*}
%
%  \label{defn:baker}
%  \index{baker's map}
%\end{definition}
%The name of the map comes from the way it operates on a square area.
%It is stretching, squeezing, and folding that characterizes the baker's map.
%And in fact--at least at an intuitive level--that is what produces chaos.
%The best way to see how this simple map gives rise to the aforementioned characteristics of chaos--sensitive dependence on initial conditions and complex orbit structure--is to compute the transformation of $I \times I$ under iteration.
%SHOW THE TRANSFORMATION

%
% Rossler: ``a sausage in a sausage in a sausage in a sausage'

An example where the Hausdorff dimension has an advantage over the box-counting dimension.
Consider the following infinite sequence:
\begin{equation*}
  1, \frac{1}{2}, \frac{1}{3}, \frac{1}{4}, \cdots
\end{equation*}
The sequence, when regarded as a set on the real line, has a non-zero box-counting dimension.
It may be viewed as a dificiency of the box-counting dimension, as one expects that
a set of discrete points is zero-dimensional.
The Hausdorff dimension, however, yields zero for this set.

Fractals are thought to be closed related to chaos.
\citet{schroeder} 
``The word \textit{symmetric} is of ancient Greek parentage and means well-proportioned, well-ordered--certainly nothing even remotely chaotic.
\citet[p.xiii-xv]{schroeder} phrases
``By symmetry we mean an invariance against change: something stays the same, in spite of some potentially consequential alteration.''

%Yet, paradoxically, self-similarity, the topic of this tome, alone among all the symmetries gives birth to its very antithesis: \textit{chaos}, a state of utter confusion and disorder.''

