\documentclass[12pt,twoside,draft]{book}
\usepackage{../../thesis}

\begin{document}
The goal of this exposition is to 
Chaos is a term used in mathematics as well as physical sciences to describe deterministic yet unpredictable systems.
Sensitive dependence on initial conditions, a phenomenon that an infinitesimal difference in the initial state eventually brings about a significiant drift in the outcome, is one way to define the idea more formally.
However, there are different mathematical formulations of this idea, and different authors have proposed non-equivalent mathematical definitions.
After a survey of the history of the subject, we study four definitions of chaos for topological dynamicical systems.
Two of the definitions
The other two definitions describe chaos in a more abstract manner. 
One of the two uses the notion of entropy in topological dynamics
symbolic dynamics.
Theorems and examples that illuminate motivations and advantages of the definitions are at the core of the expositions.
After the expositions of the definitions, we compare the four definitions under two conditions: compact interval, and general compact metric space.
The comparisons allow us to see whether the definitions are appropriate.
Our goal is not to find the best definition, but to think about what chaos might be through studying different definitions.
When the space is a compact interval, three of the four definitions are equivalent, and the other one is strictly weaker than the others.
However, in the compact metric space case, a more general setting, some of the relations between the definitions no longer hold.
%Finally, we give an example of a dynamical system, the outer billiards, to <- for what? perhaps just get rid of the chapter.
Given the result in the compact metric space, chaos is a term that encompasses a wide range of systems, and it seems too simple-minded to use the same term to describe disparate types of systems.

\bibliographystyle{../../bibliography/pjgsm}
\bibliography{../../bibliography/thesis}
\end{document}

