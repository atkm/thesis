\documentclass[10pt,twoside,draft]{book}
\usepackage{../../thesis}

\begin{document}
In this exposition, we study topological dynamical systems that exhibit deterministic yet unpredictable behaviors, or chaos.
Sensitive dependence on initial conditions, or sensitivity for short, a phenomenon that an infinitesimal change in the initial state eventually brings about significant differences in the later states, is considered to be an essential ingredient of chaos. 
However, there are different mathematical formulations of sensitivity; moreover, there are systems that are sensitive but do not behave in a ``chaotic'' manner.
Several non-equivalent definitions of chaos have been proposed, but there is no universally accepted definition.
We study five topological notions of chaos, three of which incorporate sensitivity in their definitions, one of which uses the notion of entropy to measure the complexity of orbit structures, and the other is defined in a more abstract manner. % perhaps mention symbolic and entropy.
Theorems and examples that illuminate motivations, advantages, and limitations of the definitions are at the core of the exposition.
We study different approaches to defining chaos in order to gain insights into properties that prompt us to label a system as chaotic, but not to single out the best definition.

After presenting five definitions, we compare them under two conditions: a compact interval, and compact metric space.
When the space is a compact interval, four of the five definitions are equivalent, and one is strictly weaker than the others.
In the compact metric space case, however, the equivalences no longer hold.
%The comparisons allow us to see whether the definitions are appropriate.
Results of the comparisons suggest us that our intuitive notion of chaos encompasses a wide range of systems, and it seems that a single definition of chaos would not satisfy every property associated with the term ``chaos.''

\bibliographystyle{../../bibliography/pjgsm}
\bibliography{../../bibliography/thesis}
\end{document}

