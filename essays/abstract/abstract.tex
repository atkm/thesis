\documentclass[12pt,twoside,draft]{book}
\usepackage{../../thesis}

\begin{document}
Chaos is a term used in mathematics as well as physical sciences to describe deterministic yet unpredictable systems.
Sensitive dependence on initial conditions, a phenomenon that an infinitesimal difference in the initial state eventually brings about a significiant change, is often used to define the idea more formally.
However, there are different mathematical formulations of this idea, and different authors have proposed non-equivalent mathematical definitions.
After a survey of the history of the subject, we study definitions of sensitive dependence on initial conditions, or chaos, for topological dynamicical systems.
The definitions include those motivated by topological entropy and symbolic dynamics.
Motivations of the definitions and examples that illuminate their advantages are at the core of the expositions.
We also give examples to see whether the definitions are appropriate (e.g. too strong, contain flaws).
After the expositions of the definitions, we compare the four definitions under two conditions: compact interval, and general compact metric space.
When the space is a compact interval, three of the four definitions are equivalent, and the other one is strictly weaker than the others.
However, in the compact metric space case, a more general setting, some of the relations between the definitions no longer hold.
Finally, we give an example of a dynamical system, the outer billiards to(?)
Given the result in the compact metric space, chaos is a term that encompasses a wide range of systems, and it seems too simple-minded to use the same term to describe disparate types of systems.

\bibliographystyle{../../bibliography/pjgsm}
\bibliography{../../bibliography/thesis}
\end{document}

