\documentclass[10pt,twoside,draft]{book}
\usepackage{../../thesis}

\begin{document}
In this exposition, we study topological dynamical systems that exhibit deterministic yet unpredictable behaviors, or chaos.
Sensitive dependence on initial conditions, or sensitivity, a phenomenon that an infinitesimal change in the initial state eventually brings about a significiant difference in the later state, is considered to be an essential ingredient of chaos. 
However, there are different mathematical formulations of this idea; moreover, there are systems that are sensitive but do not behave in a ``chaotic'' manner.
Several non-equivalent definitions of chaos have been proposed, but there is no universally accepted definition.
We study four topological notions of chaos, two of which incorporate sensitivity in their definitions, and the other two are defined in a more abstract manner. % perhaps mention symbolic and entropy.
Theorems and examples that illuminate motivations, advantages, and disadvantages of the definitions are at the core of the expositions.
We study different approaches of defining chaos in order to gain insights into properties that prompt us to label a system as chaotic, but not to single out the best definition of chaos.

After presenting four definitions, we compare the four under two conditions: compact interval, and compact metric space.
When the space is a compact interval, three of the four definitions are equivalent, and one is strictly weaker than the others.
However, in the compact metric space case, the equivalences no longer hold.
%The comparisons allow us to see whether the definitions are appropriate.
We also study a dynamical system, the outer billiards, to contrast the definitions. %<- for what? perhaps just get rid of the chapter.
Results of the comparisons suggest us that our intuitive notion of chaos encompasses a wide range of systems, and it seems that a single definition of chaos would not satisfy every property associated with the term ``chaos''.

\bibliographystyle{../../bibliography/pjgsm}
\bibliography{../../bibliography/thesis}
\end{document}

