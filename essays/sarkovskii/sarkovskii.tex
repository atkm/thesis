\documentclass[10pt,twoside]{book}
\usepackage{../../thesis}
\graphicspath{ {../../images/} }

\makeindex
\begin{document}
\chapter{Sarkovskii's Theorem}
\label{appendix:sarkovskii}
\begin{theorem}
  (Sarkovskii's ordering \citep{sarkovskii})
  The \textit{Sarkovskii ordering} of $\N$ is defined as follows
  \[
    \begin{array}{ccccccccc}
    3, & 5, & 7, & 9, & 11, & 13, & \ldots & \ldots & \ldots\\
    2\cdot 3, & 2\cdot 5, & 2\cdot 7, & 2\cdot 9, & 2\cdot 11, & \ldots & \ldots & \ldots\\
    2^2\cdot 3, & 2^2\cdot 5, & 2^2\cdot 7, & 2^2\cdot 9, & \ldots & \ldots & \ldots\\
    2^3\cdot 3, & 2^3\cdot 5, & 2^3\cdot 7, & \ldots & \ldots & \ldots\\
    \vdots & \vdots & \vdots & \vdots & \vdots & \vdots & \vdots & \vdots  & \vdots \\
    \\
    \ldots & \ldots & \ldots & \ldots & 2^4, & 2^3, & 2^2, & 2, & 1.
  \end{array}
  \]
  The $n$th row $(n = 0, 1, \ldots)$ contains all integers of the form $2^n\cdot (2m + 1),\; m \in \Z$ in increasing order.
  The last row contains all powers of 2 in decreasing order.

  Let $I \subseteq \R$ be a compact interval, and $F: I\to I$ be a continuous mapping.
  Suppose $F$ has a periodic orbit of period $p$.
  If $q$ comes after $p$ in the Sarkovskii ordering, then $F$ has a $q$-periodic point.
  \label{thm:sarkovskii}
  \index{Sarkovskii's ordering}
  \begin{proof}
    See \citep{blockcoppel}.
  \end{proof}
\end{theorem}
%%%
%To illustrate the idea of the proof, we prove the following simple fact.
%\begin{proposition}
%  Suppose $F: I\to I$ has a $2$-periodic point Then $F$ has a fixed point in $[p,q]$.
%\begin{proof}
%  Let $p$ be a 2-periodic point, and suppose that
%  \begin{equation*}
%    F(p) = q, F(q) = p.
%  \end{equation*}
%  Without loss of generality, we may assume that $p < q$.
%  It follows from the fixed point theorem that $\itr{F}{2}$ has a fixed point in $[p,q]$.
%\end{proof}
%\end{proposition}
%%%
Let us state two noteworthy corollaries.
\begin{corollary}
  Let $I \subseteq \R$ be a compact interval, and $F: I\to I$ be a continuous mapping.
  If $F$ has a 3-periodic point, then $F$ has periodic points of all periods.
\end{corollary}
%%%
\begin{corollary}
  Let $I \subseteq \R$ be a compact interval, and $F: I\to I$ be a continuous mapping.
  If $F$ has a $p$-periodic point for some $p$ not a power of 2, then $F$ has periodic points of periods $2^n$ for each $n \geq 0$.
\end{corollary}
%%%
 
\bibliographystyle{../../bibliography/pjgsm}
\bibliography{../../bibliography/thesis}

\printindex
\end{document}
