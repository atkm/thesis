\documentclass[12pt,draft,twoside]{book}
\usepackage{../../thesis}

\makeindex
\begin{document}

\chapter{Devaney's Definition}
\label{chap:devaney}
\section{Chaos in Devaney's Sense}
Throughout this chapter, $X$ is a metric space, and $d$ is the metric associated with the space, $F$ is a continuous map from $X$ to $X$.
\citet{devaney} defines chaos using three properties.
We first state the definition of chaos, then define individual terms.
\begin{definition}
  (Chaos in Devaney's sense) 
  A continuous map $F: X\rar X$ is said to be chaotic on $X$ if
  \begin{enumerate}
    \item Periodic points are dense in $X$.
    \item $F$ is \textit{topologically transitive}.
    \item $F$ has \textit{sensitive dependence on initial conditions}.
  \end{enumerate}
  \index{definition of chaos!Devaney}
\end{definition}
%
\begin{definition}
  (Dense Periodic Points) 
  Let $F: X \to X$.
  Define $P(F)$ to be the set of periodic points of the map.
  We say that $F$ has dense periodic points if $P(F)$ is dense in $X$.
\end{definition}
%
\begin{definition}
  (Topological Transitivity) 
  $F: X \rar X$ is said to be \textit{topologically transitive} if for any pair of open sets $U$, $V \subseteq X$ there exists $k > 0$ such that $\itr{F}{k}(U) \cap V \neq \emptyset$.
  \label{defn:transitivity}
  \index{topological transitivity}
\end{definition}
%
\begin{definition}
  (Sensitive Dependence on Initial Conditions) 
  $F: X \rightarrow X$ is said to have \textit{sensitive dependence on initial conditions} if there exists $\epsilon > 0$ such that, for any $x \in X$ and any neighborhood $N$ of $x$, there exists $y\in N$ and $n\geq 0$ such that 
  \begin{equation*}
    \metric{\itr{F}{n}(x), \itr{F}{n}(y)} > \epsilon.
  \end{equation*}
  \label{defn:sdic}
  \index{sensitive dependence on initial conditions}
\end{definition}
%
\begin{remark}
  Note that sensitive dependence on initial conditions does \textit{not} mean that \textit{all} points near $x$ eventually separate from $x$ under iteration.
The existence of one such point in every neighborhood of $x$ suffices.
Therefore, if there were an isolated point in the space, the dynamical system cannot be sensitive to initial conditions.
\end{remark}

Devaney motivates his definition as follows \citep[p.50]{devaney}:
\begin{quotation}
  [A] chaotic map possesses three ingredients:
  unpredictability, indecomposability, and an element of regularity.
  A chaotic system is unpredictable because of the sensitive dependence on initial conditions.
  It cannot be broken down or decomposed into two subsystems (two invariant open subsets) which do not interact under $f$ because of topological transitivity.
  And, in the midst of this random behavior, we nevertheless have an element of regularity, namely the periodic points which are dense.
\end{quotation}
While topological transitivity and sensitive dependence on initial conditions are natural requirements for chaotic dynamics, the last sentence about periodic orbits is noteworthy.
As we will see later in the examples, classical examples of chaotic systems possess dense periodic orbits.
Nevertheless, whether dense periodic orbits should be included in the definition of chaotic dynamical system should be considered with care.
In fact, some authors simply omit this condition.
We attribute this version of definition to Wiggins.
\begin{definition}
  (Chaos in Wiggins's sense)
  A continuous map $F: X\rar X$ is said to be chaotic on $X$ if
  \begin{enumerate}
    \item $F$ is \textit{topologically transitive}.
    \item $F$ has \textit{sensitive dependence on initial conditions}.
  \end{enumerate}
  \index{definition of chaos!Wiggins}
\end{definition}

We give some examples to illustrate the two definitions.
  \begin{example}
    (Devaney, 1987)
    An irrational rotation of the unit circle is topologically transitive, but not sensitive to initial conditions.
    All points remain the same distance apart after iteration.
  \end{example}
  \begin{example}
    (Devaney, 1987)
    The doubling map on the unit circle is chaotic in Devaney's sense.
  \end{example}
  \begin{example}
    The logistic map $L_\mu$ (we defined this mapping in the introduction) is chaotic on its attractor when $\mu > 2 + \sqrt{5}$.
    Note that we cannot say that $L_\mu$ is chaotic on $I \equiv [0,1]$, since it does not satisfy the second condition, topological transitivity.
  \end{example}
  \begin{example}
    (Rotation by magnitude is chaotic \citep{martelli})
    Let $D$ be a unit disk centered at the origin, and $C_r = \set{v \in \R^2: \norm{v} = r}$.
    Define $F: D \to D$, a rotation, by
    \begin{equation*}
      F(x) = F(r, \theta) = (r, \theta + r).
    \end{equation*}
    For every $r \in (0,1]$, $C_r$ is invariant.
    Clearly, on $C_r$, $F$ does not have sensitive dependence on initial conditions.
    However, $F$ has sensitive dependence on initial conditions on $D$, which is also an invariant set. 
    This example shows why we should include topological transitivity in the definition of chaos.
    %To see this, let $x_0 = (r_0, \theta_0)$ and $d > 0$.
    %Choose $n$ large enough that $\frac{\pi}{n} < d$ and $r_0 - \frac{\pi}{n} > 0$.
    %Let $y_0 = (r_0 - \frac{\pi}{n}, \theta_0)$.
    %Then,
    %\begin{equation*}
    %  x_n = (r_0, \theta_0 + nr_0),\quad
    %  y_n = (r_0 - \frac{\pi}{n}, \theta_0 + nr_0 - \pi),
    %\end{equation*}
    %so
    %\begin{equation*}
    %  \norm{x_0 - y_0} < d \quad\mbox{and}\quad \norm{y_n - x_n} = \paren*{\frac{\pi}{n}}^2 + \pi^2 > 2 > r_0.
    %\end{equation*}
  \end{example}
  %%%
  \begin{example}
    \citep{martelli}
    This example illustrates a dynamical system that is topologically transitive and sensitive to initial conditions, but does not have dense periodic orbits.
    Let $D \subseteq \R^2$ a closed unit disk centered at the origin.
    Let $F: D \to D$ be defined as
    \begin{equation*}
      F: (r, \theta) \mapsto (4r(1 - r), \theta + 1).
    \end{equation*}
    The origin is the only fixed point, and $F$ does not have a periodic orbit of period greater than 1.
    Hence, the periodic points of $F$ is not dense in $D$ so that $F$ is chaotic in the sense of Wiggins, but not in the sense of Devaney.
    \label{eg:notdpp}
  \end{example}
  \begin{example}
    (Twist Map \citep{wiggins})
  \end{example}

\section{Redundancy Of The Definition}
\citet{silverman} (and later \citet{banks}) showed that sensitive dependence on initial conditions is a necessary condition of transivity and dense periodic orbits property.

\begin{theorem}
  (Topological Transitivity and Dense Periodic Points Imply Sensitive Dependence on Initial Conditions)
  Let $F: X\rar X$ be a chaotic transformation. Then there is an $\epsilon > 0$ such that
  for any $x\in X$ and any open set $J$ containing at least $x$ and another point, there is
  a point $y\in J$ and $n\in \N$ with
  \begin{equation*}
    \metric{\itr{F}{n}(x),\itr{F}{n}(y)} > \epsilon.
  \end{equation*}
  \label{thm:silverman}
\end{theorem}
First, we prove a lemma, which states that two distinct periodic orbits cannot intersect.
\begin{lemma}
  Let $F: X\rar X$ be a mapping with (at least) two distinct periodic orbits.
  Then there is a $c > 0$ such that for any $x\in X$ there exists a periodic
  point $p$ such that for each $k$
  \begin{equation*}
    \metric{x, \itr{F}{k}(p)} > c.
  \end{equation*}
  \label{lem:dev1}
  \begin{proof}[Proof of the lemma]
    Let $r$ and $s$ be periodic points with distinct orbits. For each $k$ and $l$ we have
    \begin{equation*}
      \metric{\itr{F}{k}(r), \itr{F}{l}(s)} > 0.
    \end{equation*}
    Choose $c$ so that $2c < \min \metric{\itr{F}{k}(r),\itr{F}{l}(s)}$.
    Then by the triangle inequality
    \begin{equation*}
      2c < \metric{\itr{F}{k}(r),\itr{F}{l}(s)} \leq \metric{\itr{F}{k}(r),x} + \metric{x,\itr{F}{l}(s)}
    \end{equation*}
    for all $k$ and $l$.
    We cannot have $\metric{\itr{F}{k}(r),x} \leq c$ and $\metric{x,\itr{F}{l}(s)} \leq c$, since it violates the inequality just derived.
    Thus,
    \begin{equation*}
      \metric{\itr{F}{k}(r),x} > c \mbox{ or } \metric{x,\itr{F}{l}(s)} > c
    \end{equation*}
    must hold.
  \end{proof}
\end{lemma}
  %
\begin{proof}[Proof of Theorem~\ref{thm:silverman}]
  Let $x$ be any point of $X$ and $J$ any open set containing $x$.
  Since the periodic points of $f$ are dense (in $X$), we may introduce a periodic point
  $q$ of $f$ in $U$ defined as
  \begin{equation*}
    U \ceq J\cap \oball{\epsilon}{x}.
  \end{equation*}
  Let $n$ be the period of $q$.
  Apply Lemma~\ref{lem:dev1} to obtain a periodic orbit $O(p)$ and a constant $c > 0$ such that $\metric{x, O(p)} > c$.
  Let $\epsilon \ceq c/4$ so that $\metric{x, O(p)} > 4\epsilon$.
  Define
  \begin{equation*}
    W_i := \oball{\epsilon}{\itr{F}{i}(p)} \quad\text{ and }\quad V := \bigcap\limits_{i = 1}^{n} \itr{F}{-i}(W_i).
  \end{equation*}
  Since $\itr{F}{i}(p) \in W_i$, we see that $p \in V$, and $V$ is not empty. 
  Also note that $V$ is open.
  By the transitivity property, we can find a $z\in U$ and a positive integer $k$ such that
  $\itr{F}{k}(z) \in V$. Let $j$ be the least integer such that $k < nj$. 
  That is,
  \begin{equation*}
    1 \leq nj - k \leq n.
  \end{equation*}
  Therefore,
  \begin{equation*}
    \itr{F}{nj} (z) = \itr{F}{nj - k} (\itr{F}k(z)) \in \itr{F}{nj-k}(V).
  \end{equation*}
  Also,
  \begin{equation*}
    \itr{F}{nj - k}(V) \subset \itr{F}{nj - k}(\itr{F}{-(nj - k)} W_{nj-k}) 
    = W_{nj-k}
    \equiv \oball{\epsilon}{\itr{F}{nj-k}(p)}.
  \end{equation*}
  The last two equalities imply
  \begin{equation*}
    \metric{\itr{F}{nj}(z), \itr{F}{nj - k}(p)} < \epsilon.
  \end{equation*}
  Furthermore, $\itr{F}{nj}(q) = q$, since $q$ is a $n$-periodic point 
  \begin{equation*}
    \metric{\itr{F}{nj}(q), \itr{F}{nj}(z)} = \metric{q, \itr{F}{nj}(z)}.
  \end{equation*}
  By the triangle inequality, we have
  \begin{equation*}
    \metric{x, \itr{F}{nj-k}(p)} \leq \metric{x,q} + \metric{q, \itr{F}{nj}(z)} + \metric{\itr{F}{nj}(z), \itr{F}{nj - k}(p)}.
  \end{equation*}
  Finally, 
  \begin{align*}
    \metric{\itr{F}{nj}(q), \itr{F}{nj}(z)} 
    &= \metric{q, \itr{F}{nj}(z)}  \\
    &\geq  \metric{x, \itr{F}{nj-k}(p)} - \metric{x,q} - \metric{\itr{F}{nj}(z), \itr{F}{nj - k}(p)}  \\
    &> 4\epsilon - \epsilon - \epsilon 
    = 2\epsilon,
  \end{align*}
  which implies
  \begin{equation*}
    \metric{\itr{F}{nj}(q), \itr{F}{nj}(x)} + \metric{\itr{F}{nj}(x), \itr{F}{nj}(z)} 
    \geq \metric{\itr{F}{nj}(q), \itr{F}{nj}(z)}
    > 2\epsilon.
  \end{equation*}
  It follows from the inequality that 
  \begin{equation*}
    \metric{\itr{F}{nj}(x), \itr{F}{nj}(z)} \geq \epsilon \quad\mbox{ or }\quad \metric{\itr{F}{nj}(x), \itr{F}{nj}(q)} \geq \epsilon 
  \end{equation*}
  must hold.
\end{proof}

Thus, we can regard dense periodic orbits and topological transitivity as sufficient conditions for sensitive dependence on initial conditions.
Using this result, we can restate Devaney's definition in a more concise form, since the third condition, sensitive dependence on initial condition, is redundant.
\begin{definition}
  (Chaos in Devaney's sense) 
  A continuous map $F: X\rar X$ is said to be chaotic on $X$ if
  \begin{enumerate}
    \item Periodic points are dense in $X$.
    \item $F$ is \textit{topologically transitive}.
  \end{enumerate}
\end{definition}
%%%
A natural question to raise at this point is whether we can further reduce the definition to one condition, or whether other combinations of two conditions imply the other.
The answers to these question are negative; \citet{assaf} showed that, in general, dense periodic orbits and sensitivity do not imply transitivity.
Also, Example~\ref{eg:notdpp} shows that sensitivity and transitivity do not necessarily imply dense periodic orbits.
However, when we restrict our space to a compact interval, topological transitivity implies the other two \citep{silverman}.
%A system is chaotic in $X$ in the sense of Devaney if and only if for every pair of open sets in $X$ there exists a periodic orbit which visits both sets (Touhey, 1997).

\section{Chaos Is A Topological Property}
A nice property of chaos in Devaney's sense is that all conditions are topological invariance, that is, the properties are preserved under conjugation by a continuous mapping.
Since the properties of our interest are global properties of a space, we also require that the mapping be 
\begin{definition}
    We say that $G$ is \textit{semi-conjugate} to $F$ (by $\phi$), if there exists continuous and surjective mapping $\phi$ such that $\phi\circ F = G\circ\phi$.
    Also, we say that $G$ is \textit{conjugate} to $F$ if $G$ is semi-conjugate to $F$ and vice versa (i.e. $\phi$ is a homeomorphism).
\end{definition}
We show that the dense periodic orbits and topological transitivity are topological invariants.
It follows that sensitive dependence on initial conditions is also a topological invariance.
  \begin{theorem}
    (Transitivity and Conjugacy) 
    Let $F: X \to X$ and $G: Y \to Y$ be continuous functions.
    Suppose that $G$ is semi-conjugate to $F$.
    If $F$ is transitive, then $G$ is transitive.
    \label{thm:conj-trans}
    \begin{proof}
      Let $A$ and $B$ be open sets of $Y$.
      Since $\phi^{-1}$ is continuous, $\phi^{-1}(A)$ is a union of open sets in $X$.
      Let $A'$ be one of the open sets constituting $\phi^{-1}(A)$.
      Similarly, let $B'$ be one of the open sets constituting $\phi^{-1}(B)$.
      By the transitivity of $F$, there exists a positive integer $n$ and $x \in B'$ such that $\itr{F}{n}(x)\in A'$.
      Let $y \ceq \phi(x)$.
      Then, $y \in B$, and 
      \begin{equation*}
        \itr{G}{n}(y) 
        = \itr{G}{n}(\phi(x))
        = \phi(\itr{F}{n}(x)) \in \phi(A') \subseteq A.
      \end{equation*}
      \end{proof}
  \end{theorem}
%%%
  We prove two lemmas, from which the theorem of our interest immediately follows.
  \begin{lemma}
    (Dense Set and Conjugacy) 
    Let $\phi$ be a continuous and surjective mapping.
    If $D$ is dense in $X$, then $\phi(D)$ is dense in $Y$.
    \label{thm:conj-dense}
    \begin{proof}
      Let $A \subseteq Y$ be an open set.
      Since $\phi$ is continuous, $\phi^{-1} (A)$ is a union of open sets.
      Let $B$ be one of the open sets that constitutes $\phi^{-1}(A)$.
      The intersection $B \cap D$ is nonempty, because $D$ is dense in $Y$.
      For each $y \in B \cap D$, we have $\phi(y) \in A \cap \phi(D)$.
    \end{proof}
  \end{lemma}
  \begin{lemma}
    (Period Points and Conjugacy)
    Suppose $G$ is semi-conjugate to $F$.
    If $p$ is a $n$-periodic point of $F$, then $\phi(p)$ is a $n$-periodic point of $G$.
    \label{thm:conj-per}
    \begin{proof}
      $\itr{F}{n}(p) = p$ implies that $\itr{G}{n}(\phi(p)) = \phi(\itr{F}{n}(p)) = \phi(p)$.
    \end{proof}
  \end{lemma}
\begin{theorem}
    Suppose that $G$ is semi-conjugate to $F$.
    If $\mathrm{P}(F)$ is dense in $X$, then $\mathrm{P}(G)$ is dense in $Y$.
    \label{cor:conj-dense-per}
    \begin{proof}
      It follows from Theorem~\ref{thm:conj-per} that $\phi(\mathrm{P}(F)) \subseteq \mathrm{P}(G)$.
      Then, by Theorem~\ref{thm:conj-dense}, $\mathrm{P}(G)$ is a dense subset of $Y$.
    \end{proof}
\end{theorem}
Finally, we obtain the main result of this section as a corollary.
\begin{corollary}
  Suppose that $G$ is semi-conjugate to $F$.
  If $F$ is sensitive to initial conditions, then $G$ is sensitive to initial conditions.
  \label{cor:conj-sdic}
  \begin{proof}
  This follows from Theorem~\ref{thm:silverman}, Theorem~\ref{thm:conj-trans}, and Theorem~\ref{thm:conj-dense}.
  \end{proof}
\end{corollary}

\section{Martelli's Definition}
In this final section of the chapter, we introduce a definition that is equivalent to Wiggins's definition in $\R^n$.
\citet{martellibook}

\begin{definition}
  (Limit Points of an Orbit/Limit Set)
  Let $F: X\to X$ be an continuous mapping.
  Let $O(x_0)$ be the orbit of $x_0 \in X$.
  $z \in X$ is called a \textit{limit point} of $O(x_0)$, if there exists a subsequence of $O(x_0)$ that converges to $z$.
  The \textit{limit set} of $O(x_0)$, denoted $L(x_0)$, is the set of limit points $O(x_0)$.
  \label{def:limset}
  \index{limit set}
\end{definition}
%With the notion of limit sets, we may speak of aympototical periodicity of an orbit.
%\begin{definition}
%  (Asymptotically periodic orbit)
%  An orbit $O(x_0)$ is said to be \textit{asymptotically periodic} if its limit set is a periodic orbit.
%  \label{def:asymporb}
%  \index{asymptotically periodic!orbit}
%\end{definition}
%
%\begin{definition}
%  (Aperiodic orbit)
%  The orbit $O_F(x_0)$ is said to be \textit{aperiodic} if its limit set $L_F(x_0)$ is not finite.
%  \label{def:aporbit}
%  \index{aperiodic!orbit}
%\end{definition}

Then, chaos can be defined in terms of stability.
\begin{definition}
  (Chaos in Martelli's sense)
  Let $U$ be open in $\R^n$, $F: U \to \R^n$, and $X\subset U$ be closed, bounded and invariant.
  Assume that $F$ is continuous in $X$.
  Then $F$ is said to be \textit{chaotic} in $X$ if there exists $x_0 \in X$ that meets the following conditions
  \begin{enumerate}
    \item $L(x_0)$ is dense in $X$.
    \item $O(x_0)$ is unstable in $X$.
  \end{enumerate}
  \label{defn:martelli}
  \index{definition of chaos!Martelli}
\end{definition}
This definition is equivalent to Wiggins's definition.
\begin{theorem}
  (Equivalence of Martelli's and Wiggins's definitions)
  Let $F: X \to X$.
  $F$ is chaotic in Martelli's sense if and only if $F$ is chaotic in Wiggins's sense.
  \label{thm:martelli-wiggins}
\end{theorem}
We prove this theorem in two parts.
The outline of proof is in \citet{martelli}.
\begin{lemma}
  \citep{silverman}
  Suppose $X$ is a metric space with no isolated point.
  If $F: X\to X$, a continuous map, has a dense orbit, then $F$ is topologically transitive.
  \label{lem:dob-transitivity}
\end{lemma}
\begin{proposition}
  Let $X$ be a metric space with no isolated point, and $F: X \to X$ be a continuous mapping.
  $F$ is topologically transitive in $X$ if and only if there exists $x_0 \in X$ such that $L(x_0) = X$.
  \begin{proof}
    We first show that the existence of an orbit whose limit set is $X$ implies topological transitivity.
    $O(x_0)$ is a dense orbit.
    By Lemma~\ref{lem:dob-transitivity}, $F$ is topologically transitive.

    Next, suppose that $F$ is topologically transitive.
    A finite open cover $\mathcal{C}$ of $X$ exists because $X$ is compact.
    Suppose the cardinality of $\mathcal{C}$ is $m$, and let $\mathcal{C} = \set{A_i}_{i = 1}^m$.
    For each $i$, choose a point $y_i \in A_i$.
    so that the set $\set{N_{1/m}(x_1), \ldots, N_{1/m}(x_m)}$ forms a finite open cover of $X$.
    (Note that choice of $x_1, \ldots, x_m$ depends on $m$.
    So, for example, $x_1$ should be written as $x_{1,m}$.
    However, we omit this $m$ to avoid cluttering up the argument.)
    Define sets $Y_1, \ldots, Y_m$ as
    \begin{align*}
      Y_i &:= F^{n(i)}(N_{1/m}(Y_{i-1})) \cap N_{1/m}(x_{i+1}) \quad (\mbox{for } i \geq 2) \\
      Y_1 &:= N_{1/m}(x_1),
    \end{align*}
    where $n(i)$ is an integer large enough to ensure each $Y_i$ is not empty, which is guaranteed to exist by topological transitivity.
    Finally, let $T = F^{-\sum\limits_{i=1}^{m}n(i)}(Y_m) \cap Y_1$.
    By construction, a point in $T$ visits every $N_{1/m}(x_i)$.
    Furthermore, as we take $m \to \infty$, we may assume that $\set{x_{1,m}}_m$ converges to a point, 
    since, if it doesn't, we can find a subsequence of $\set{x_{1,m}}_m$ that converges and replace the original sequence with the convergent subsequence.
    In the limit $m \to \infty$, any point in $T$, say $x_0$, satisfies $L(x_0) = X$.
  \end{proof}
\end{proposition}
\begin{proposition}
  \begin{proof}
    $F$ has sensitive dependence on initial conditions with respect to $X$ if and only if $O(x_0)$ is unstable with respect to $X$.
    The ``only if'' part is obvious, since sensitive dependence on initial conditions is a stronger condition.
    Of course, the ``if'' part is not true in general. We will show, however, 
    that the two conditions are equivalent if $F$ has an unstable orbit
    whose limit set fills up $X$.
    So assume that $O(x_0)$ is an unstable orbit such that $L(x_0) = X$,
    and let $r(x_0)$ be its separation constant.
    It follows that for each $y_0\in X$ and $d > 0$ there exists $p>0$ such that $\metric{y_0, F^p(x_0)} \leq d/2$.
    One case is that there exists $n$ such that $\metric{F^n(y_0), F^{p+n}(x_0)} > r(x_0)/2$.
    In the other case (i.e. for each $n$, $\metric{F^n(y_0), F^{p+n}(x_0)} \leq r(x_0))/2$,
    there must exist a different point $w_0$ in a neighborhood of $F^p(x_0)$ such that 
    \begin{equation*}
      \metric{w_0, F^p(x_0)} \leq \frac{d}{2} \quad \mbox{ and } \quad \metric{F^m(w_0), F^{m+p}(x_0)} > r(x_0).
    \end{equation*}
    Note that
    \begin{equation*}
      \metric{y_0, w_0} \leq \metric{y_0, F^p(x_0)} + \metric{F^p(x_0), w_0} \leq d
    \end{equation*}
    by the triangle inequality.
    Moreover,
    \begin{align*}
      r(x_0) &< \metric{F^m(w_0), F^{m+p}(x_0)} \\
      &\leq \metric{F^{m+p}(x_0), F^m(y_0)} + \metric{F^m(y_0), F^m(w_0)}  \\
      &\leq \frac{r(x_0)}{2} + \metric{F^m(y_0), F^m(w_0)}
    \end{align*}
    Hence in both cases we have
    \begin{equation*}
      \frac{r(x_0)}{2} < \metric{F^m(y_0), F^m(w_0)}.
    \end{equation*}
    Thus, $F$ has sensitive dependence on initial conditions with respect to $X$, and 
    $r(x_0)/2$ is the separation constant for any point in $X$.
  \end{proof}
\end{proposition}


\begin{definition}
  (Expansiveness) $F: J\rar J$ is expansive if there exists $\epsilon > 0$
  such that, for any $x,y\in J$, there exists $n$ such that
  $\norm{f^{n}(x)-f^{n}(y)} > \epsilon$.
  \index{defn:expansiveness}
\end{definition}
\begin{definition}
  (Structural Stability) $F: J \rar J$ is said to be $C^r$-structurally
  stable on $J$, if there exists $\epsilon > 0$ such that whenever
  $d_r(f,g) < \epsilon$ for $g: J\rar J$, it follows that $f$
  is topologically conjugate to $g$.
  \index{defn:structural stability}
\end{definition}


\bibliographystyle{../../bibliography/pjgsm}
\bibliography{../../bibliography/thesis}

\printindex
\end{document}
