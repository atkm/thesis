\documentclass[12pt,draft,twoside]{book}
\usepackage{../../thesis}

\makeindex
\begin{document}

\chapter{Devaney's Definition}
\citet{devaney} defines chaos in topological dynamics using three properties.

\begin{definition}
  (Topological Transitivity) 
  $f: J \rar J$ is said to be topologically transitive if for any pair of open sets $U$, $V \subseteq J$ there exists $k > 0$ such that $\itr{f}{k}(U) \cap V \neq \emptyset$.
  \label{defn:transitivity}
  \index{topological transitivity}
\end{definition}

Note that we do \textit{not} require that \textit{all} points near $x$ need eventually separate from $x$ under iteration.
The existence of one such point in every neighborhood of $x$ suffices in this definition of chaos.

\begin{definition}
  (Sensitive Dependence on Initial Conditions) 
  $f: J \rightarrow J$ is said to have sensitive initial conditions if there exists (for any???) $\delta > 0$
  such that, for any $x \in J$ and any neighborhood $N$ of $x$,
  there exists $y\in N$ and $n\leq 0$ such that $|\itr{f}{n}(x) - \itr{f}{n}(y)|>\delta$.
  \label{defn:sdic}
  \index{sensitive dependence on initial conditions}
\end{definition}

\begin{definition}
  (Chaos in the sense of Devaney) 
  Let $V$ be a set.
  $f: V\rar V$ is said to be chaotic on $V$ if
  \begin{enumerate}
    \item $f$ has sensitive dependence on initial conditions.
    \item $f$ is topologically transitive.
    \item Periodic points are dense in $V$.
  \end{enumerate}
  \index{definition of chaos!Devaney}
\end{definition}


We will show Sensitive dependence on initial conditions can be derived from transivity and dense property if the mapping is continuous and $X$ is a compact, invariant set later in this chapter. \citep{banks}
Furthermore, for an interval, \textbf{the only property necessary is
topological transitivity} (Vellekoop-Berglund, 1994).

A system is chaotic in $X$ in the sense of Devaney if and only if
for every pair of open sets in $X$ there exists a periodic orbit 
which visits both sets (Touhey, 1997).


\begin{proposition}
  (Conjugacy, Transitivity, and Homeomorphism) 
  Suppose that $g \compose h = h \compose f$. If $h$ is continuous and
  surjective and $f$ is transitive, then $g$ is transitive.
  If $h$ is a homeomorphism, then $f$ is transitive if and only if
  $g$ is transitive.
\end{proposition}

\bibliographystyle{../../bibliography/pjgsm}
\bibliography{../../bibliography/thesis}

\printindex
\end{document}
