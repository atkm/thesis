\documentclass[12pt,draft,twoside]{book}
\usepackage{../../thesis}

\makeindex
\begin{document}

\chapter{Devaney's Definition}
\label{chap:devaney}
\citet{devaney} defines chaos in topological dynamics using three properties.
We first state the definition, and 
Throughout this chapter, $X$ is a compact metric space, and $d$ is the metric associated with the space.
\begin{definition}
  (Chaos in Devaney's sense) 
  A continuous map $F: X\rar X$ is said to be chaotic on $X$ if
  \begin{enumerate}
    \item Periodic points are dense in $X$.
    \item $F$ is \textit{topologically transitive}.
    \item $F$ has \textit{sensitive dependence on initial conditions}.
  \end{enumerate}
  \index{definition of chaos!Devaney}
\end{definition}
%
\begin{definition}
  (Dense Periodic Points) 
  Let $F: X \to X$.
  Define $P(F)$ to be the set of periodic points of the map.
  We say that $F$ has dense periodic points if $P(F)$ is dense in $X$.
\end{definition}
%
\begin{definition}
  (Topological Transitivity) 
  $F: X \rar X$ is said to be \textit{topologically transitive} if for any pair of open sets $U$, $V \subseteq X$ there exists $k > 0$ such that $\itr{F}{k}(U) \cap V \neq \emptyset$.
  \label{defn:transitivity}
  \index{topological transitivity}
\end{definition}
%
\begin{definition}
  (Sensitive Dependence on Initial Conditions) 
  $F: X \rightarrow X$ is said to have \textit{sensitive dependence on initial conditions} if there exists $\epsilon > 0$ such that, for any $x \in X$ and any neighborhood $N$ of $x$, there exists $y\in N$ and $n\geq 0$ such that 
  \begin{equation*}
    \metric{\itr{F}{n}(x), \itr{F}{n}(y)} > \epsilon.
  \end{equation*}
  \label{defn:sdic}
  \index{sensitive dependence on initial conditions}
\end{definition}
%
\citet[p.50]{devaney}
\begin{quotation}
  [A] chaotic map possesses three ingredients:
  unpredictability, indecomposability, and an element of regularity.
  A chaotic system is unpredictable because of the sensitive dependence on initial conditions.
  It cannot be broken down or decomposed into two subsystems (two invariant open subsets) which do not interact under $f$ because of topological transitivity.
  And, in the midst of this random behavior, we nevertheless have an element of regularity, namely the periodic points hwich are dense.
\end{quotation}
\noindent Note that for sensitive dependence on initial conditions we do \textit{not} require \textit{all} points near $x$ need eventually separate from $x$ under iteration.
The existence of one such point in every neighborhood of $x$ suffices.
We give some examples to illustrate the definition.
\begin{example}
  (Devaney, 1987)
  An irrational rotation of the circle $S^1$ is topologically transitive, but not sensitive to initial conditions.
  All points remain the same distance apart after iteration.
\end{example}
\begin{example}
  (Devaney, 1987)
  The doubling map on a $S^1$ is chaotic.
\end{example}
\begin{example}
  The logistic map $L_\mu$ is chaotic on its attractor when $\mu > 2 + \sqrt{5}$.
\end{example}
\begin{example}
  (Rotation by magnitude is chaotic)
  Let $D = \set{x \in \R^2: \norm{x} \leq 2}$, and $C_r = \set{x \in \R^2: \norm{x} = r}$.
  Define $F: D \to D$, a rotation in $C_r$, by
  \begin{equation*}
    F(x) = F(r, \theta) = (r, \theta + r).
  \end{equation*}
  For every $r \in (0,2]$, $C_r$ is invariant.
  We claim that the system has in $C_r$ sensitive dependence on initial conditions with $r_0 = r$.
  Let $x_0 = (r_0, \theta_0)$ and $d > 0$.
  Choose $n$ large enough that $\frac{\pi}{n} < d$ and $r_0 - \frac{\pi}{n} > 0$.
  Let $y_0 = (r_0 - \frac{\pi}{n}, \theta_0)$.
  Then,
  \begin{equation*}
    x_n = (r_0, \theta_0 + nr_0),\quad
    y_n = (r_0 - \frac{\pi}{n}, \theta_0 + nr_0 - \pi),
  \end{equation*}
  so
  \begin{equation*}
    \norm{x_0 - y_0} < d \quad\mbox{and}\quad \norm{y_n - x_n} = \paren*{\frac{\pi}{n}}^2 + \pi^2 > 2 > r_0.
  \end{equation*}
\end{example}

\begin{example}
  Let $D \subseteq \R^2$ a closed disk centered at the origin with radius 1.
  Let $F: D \to D$ be defined as
  \begin{equation*}
    F: (r, \theta) \mapsto (4r(1 - r), \theta + 1).
  \end{equation*}
  The origin is the only fixed point.
  $F$ does not have a periodic orbit of period greater than 1.
  Hence, the periodic points of $F$ is not dense in $D$ so that $F$ is not chaotic in the sense of Devaney.
\end{example}

\section{Tidying Up the Definition}
\citet{banks} showed that sensitive dependence on initial conditions can be derived from transivity and dense periodic orbits property if the mapping is continuous and $X$ is a compact, invariant set. 

\begin{theorem}
  (Topological Transitivity and Dense Periodic Points Imply Sensitive Dependence on Initial Conditions)
  Let $F: X\rar X$ be a chaotic transformation. Then there is an $\epsilon > 0$ such that
  for any $x\in X$ and any open set $J$ containing at least $x$ and another point, there is
  a point $y\in J$ and $n\in \N$ with
  \begin{equation*}
    \metric{\itr{F}{n}(x),\itr{F}{n}(y)} > \epsilon.
  \end{equation*}
  \label{thm:banks}
\end{theorem}
First, we shall prove a lemma that is essential to the proof of the theorem.
Two distincy periodic orbits cannot intersect.
This observation results in the following lemma.
\begin{lemma}
  Let $F: X\rar X$ be a mapping with (at least) two distinct periodic orbits.
  Then there is a $c > 0$ such that for any $x\in X$ there exists a periodic
  point $p$ such that for each $k$
  \begin{equation*}
    \metric{x, \itr{F}{k}(p)} > c.
  \end{equation*}
  \label{lem:dev1}
\begin{proof}[Proof of the lemma]
  Let $r$ and $s$ be periodic points with distinct orbits. For each $k$ and $l$ we have
  \begin{equation*}
    \metric{\itr{F}{k}(r), \itr{F}{l}(s)} > 0.
  \end{equation*}
  Choose $c$ so that $2c < \min \metric{\itr{F}{k}(r),\itr{F}{l}(s)}$.
  Then by the triangle inequality
  \begin{equation*}
    2c < \metric{\itr{F}{k}(r),\itr{F}{l}(s)} \leq \metric{\itr{F}{k}(r),x} + \metric{x,\itr{F}{l}(s)}
  \end{equation*}
  for all $k$ and $l$.
  Now suppose that $\metric{\itr{F}{k}(r),x} \leq c$ and $\metric{x,\itr{F}{l}(s)} \leq c$. The supposition
  violates the inequality just derived. Thus 
  \begin{equation*}
    \metric{\itr{F}{k}(r),x} > c \mbox{ or } \metric{x,\itr{F}{l}(s)} > c
  \end{equation*}
  must hold.
\end{proof}
\end{lemma}
%
\begin{proof}[Proof of \ref{thm:banks}]
  Let $x$ be any point of $X$ and $J$ any open set containing $x$.
  Since the periodic points of $f$ are dense (in $X$), we may introduce a periodic point
  $q$ of $f$ in $U$ defined as
  \begin{equation*}
    U \ceq J\cap \oball{\epsilon}{x}.
  \end{equation*}
  Let $n$ be the period of $q$.
  Apply Lemma~\ref{lem:dev1} to obtain a periodic orbit $O(p)$ and a constant $c > 0$ such that $\metric{x, O(p)} > c$.
  Let $\epsilon \ceq c/4$ so that $\metric{x, O(p)} > 4\epsilon$.
  Define
  \begin{equation*}
    W_i := \oball{\epsilon}{\itr{F}{i}(p)} \quad\text{ and }\quad V := \bigcap\limits_{i = 1}^{n} \itr{F}{-i}(W_i).
  \end{equation*}
  Since $\itr{F}{i}(p) \in W_i$, we see that $p \in V$, and $V$ is not empty. 
  Also note that $V$ is open.
  By the transitivity property, we can find a $z\in U$ and a positive integer $k$ such that
  $\itr{F}{k}(z) \in V$. Let $j$ be the least integer such that $k < nj$. 
  That is,
  \begin{equation*}
    1 \leq nj - k \leq n.
  \end{equation*}
  Therefore,
  \begin{equation*}
    \itr{F}{nj} (z) = \itr{F}{nj - k} (\itr{F}k(z)) \in \itr{F}{nj-k}(V).
  \end{equation*}
  Also,
  \begin{equation*}
     \itr{F}{nj - k}(V) \subset \itr{F}{nj - k}(\itr{F}{-(nj - k)} W_{nj-k}) 
     = W_{nj-k}
     \equiv \oball{\epsilon}{\itr{F}{nj-k}(p)}.
  \end{equation*}
  The last two equalities imply
  \begin{equation*}
    \metric{\itr{F}{nj}(z), \itr{F}{nj - k}(p)} < \epsilon.
  \end{equation*}
  Furthermore, $\itr{F}{nj}(q) = q$, since $q$ is a $n$-periodic point 
  \begin{equation*}
    \metric{\itr{F}{nj}(q), \itr{F}{nj}(z)} = \metric{q, \itr{F}{nj}(z)}.
  \end{equation*}
  By the triangle inequality, we have
  \begin{equation*}
    \metric{x, \itr{F}{nj-k}(p)} \leq \metric{x,q} + \metric{q, \itr{F}{nj}(z)} + \metric{\itr{F}{nj}(z), \itr{F}{nj - k}(p)}.
  \end{equation*}
  Finally, 
  \begin{align*}
    \metric{\itr{F}{nj}(q), \itr{F}{nj}(z)} 
    &= \metric{q, \itr{F}{nj}(z)}  \\
    &\geq  \metric{x, \itr{F}{nj-k}(p)} - \metric{x,q} - \metric{\itr{F}{nj}(z), \itr{F}{nj - k}(p)}  \\
    &> 4\epsilon - \epsilon - \epsilon 
    = 2\epsilon,
  \end{align*}
  which implies
  \begin{equation*}
    \metric{\itr{F}{nj}(q), \itr{F}{nj}(x)} + \metric{\itr{F}{nj}(x), \itr{F}{nj}(z)} 
    \geq \metric{\itr{F}{nj}(q), \itr{F}{nj}(z)}
    > 2\epsilon.
  \end{equation*}
  It follows from the inequality that 
  \begin{equation*}
    \metric{\itr{F}{nj}(x), \itr{F}{nj}(z)} \geq \epsilon \quad\mbox{ or }\quad \metric{\itr{F}{nj}(x), \itr{F}{nj}(q)} \geq \epsilon 
  \end{equation*}
   must hold.

\end{proof}

Furthermore, in an interval, topological transitivity implies the other two (Vellekoop-Berglund, 1994).
A system is chaotic in $X$ in the sense of Devaney if and only if for every pair of open sets in $X$ there exists a periodic orbit which visits both sets (Touhey, 1997).

In particular, we mainly consider conjugation by continuous, surjective transformations.
In some cases, we will consider conjugation by homeomorphism.
\begin{theorem}
  (Transitivity and Conjugacy) 
  Let $F: X \to X$, $G: Y \to Y$, and $\phi: X \to Y$.
  Suppose that $G \compose \phi = \phi \compose F$.
  Suppose $H$ is continuous and surjective, and $F$ is transitive.
  Then $G$ is transitive.
  If $\phi$ is a homeomorphism, then $F$ is transitive if and only if $G$ is transitive.
\end{theorem}
A similar theorem for dense periodic orbits.
\begin{theorem}
  (Dense Periodic Orbits and Conjugacy) 
  Suppose that $G \compose \phi = \phi \compose F$.
  Suppose $\phi$ is continuous and surjective, and $P(F)$ is dense in $X$.
  Then $P(G)$ is dense in $Y$.
\end{theorem}
Note that in the proof, we proved the following
\begin{corollary}
  (Period is preserved under conjugacy)
\end{corollary}
We obtain the results above when $\phi$ is continuous and surjective.

\bibliographystyle{../../bibliography/pjgsm}
\bibliography{../../bibliography/thesis}

\printindex
\end{document}
