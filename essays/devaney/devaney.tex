\documentclass[10pt,draft,twoside]{book}
\usepackage{../../thesis}

\makeindex
\begin{document}

\chapter{Devaney's Definition}
\label{chap:devaney}
\section{Chaos in Devaney's Sense}
Throughout this chapter, $X$ is a metric space, and $d$ is the metric associated with the space.
\citet{devaney} defines chaos using three properties.
We first state his definition of chaos, then define the terms used in the definition.
\begin{definition}
  (Chaos in Devaney's sense) 
  Let $F: X\rar X$ be a continuous map.
  Suppose there exists a closed and invariant subset $Y \subseteq X$ such that
  \begin{enumerate}
    \item \textit{Periodic points are dense} in $Y$.
    \item $F$ is \textit{topologically transitive} on $Y$.
    \item $F$ has \textit{sensitive dependence on initial conditions} in $Y$.
  \end{enumerate}
  Then, $F$ is said to be chaotic on $Y \subseteq X$.
  We will often refer to these conditions as \dpp (dense periodic points), \tt, and \sdic.
  \index{definition of chaos!Devaney}
\end{definition}
%
\begin{definition}
  (Dense Periodic Points) 
  Let $F: X \to X$.
  Define $P(F)$ to be the set of periodic points of the map.
  We say that $F$ has dense periodic points if $P(F)$ is dense in $X$.
\end{definition}
%
\begin{definition}
  (Topological Transitivity) 
  $F: X \rar X$ is said to be \textit{topologically transitive} if for any pair of open sets $U$, $V \subseteq X$ there exists $k > 0$ such that $\itr{F}{k}(U) \cap V \neq \emptyset$.
  \label{defn:transitivity}
  \index{topological transitivity} \end{definition}
%
\begin{definition}
  (Sensitive Dependence on Initial Conditions) 
  $F: X \rightarrow X$ is said to have \textit{sensitive dependence on initial conditions} if there exists $\epsilon > 0$, which we refer to as the \textit{separation constant}, such that, for any $x \in X$ and any neighborhood $N$ of $x$, there exists $y\in N$ and $n > 0$ such that 
  \begin{equation*}
    \metric{\itr{F}{n}(x), \itr{F}{n}(y)} > \epsilon.
  \end{equation*}
  \label{defn:sdic}
  \index{sensitive dependence on initial conditions}
\end{definition}
%
\sdic does \textit{not} mean that \textit{all} points near $x$ eventually separate from $x$ under iteration.
The existence of one such point in every neighborhood of $x$ suffices.
However, since the mapping is continuous, points close to the point that separates from $x$ also move away from $x$ to some extent.
Also, if there were an isolated point in the space, the dynamical system cannot be sensitive to initial conditions.

We will frequently use the following property of a topologically transitive map.
\begin{proposition}
  Let $F: X \to X$ be a topologically transitive, continuous map.
  Also, let $\seq{A_i}{\infty}{i=1}$ be a sequence of open subsets of $X$.
  Then, there exist $x \in X$ and a subsequence $\set{\itr{F}{n(i)}(x)}$ of $O(x)$ such that, for each $i$,
  \begin{equation*}
    \itr{F}{n(i)}(x) \in A_i
  \end{equation*}
  holds.
  \label{prop:transitivity}
  \begin{proof}
    For each $m \geq 1$, define $\mathcal{B}_m \ceq \seq{B_i}{m}{i=1}$ as follows
    \begin{align*}
      B_i &\ceq A_{i} \cap \itr{F}{-n(i+1)} (B_{i+1}) \quad (i < m) \\
      B_m &\ceq A_m,
    \end{align*}
    where, for each $m \geq i > 1$, we define $n(i)$ to be a positive integer such that $\itr{F}{n(i)}(A_{i-1}) \cap B_i$ is nonempty.
    Such $n(i)$ exists because both $A_{i-1}$ and $B_i$ are open.
    Also, set $n(1) = 0$ so that $\itr{F}{n(1)}(A_1) = A_1$.
    Now, let $\mathcal{B} \ceq \lim\limits_{m \to \infty} \mathcal{B}_m$.
    Consider $x \in B_1 \in \mathcal{B}$.
    By construction, we have $\itr{F}{n(i)}(x) \in A_i$.
  \end{proof}
\end{proposition}

We informally defined in the introduction that a chaotic mapping is a mapping that has a sensitive dependence on initial conditions, but Devaney's definition consists of properties besides sensitivity.
Also, there is no reference to the geometry of the mapping in his definition.
Devaney motivates his definition as follows \citep[p.50]{devaney}:
\begin{quotation}
  [A] chaotic map possesses three ingredients:
  unpredictability, indecomposability, and an element of regularity.
  A chaotic system is unpredictable because of the sensitive dependence on initial conditions.
  It cannot be broken down or decomposed into two subsystems (two invariant open subsets) which do not interact under $f$ because of topological transitivity.
  And, in the midst of this random behavior, we nevertheless have an element of regularity, namely the periodic points which are dense.
\end{quotation}
While \tt and \sdic are natural requirements for chaotic dynamics, the last sentence about periodic orbits is noteworthy.
As we will see later in the examples, classical examples of chaotic systems possess dense periodic points.
Nevertheless, whether \dpp should be included in the definition of chaotic dynamical system should be considered with care.
\citet{glasner} suggest that \dpp is too strong a condition, and he replaces the condition with the existence of an invariant measure that is positive on each non-empty open set. 
Some authors simply omit this condition.
We attribute this version of definition to Wiggins, though some authors refer to definition as Auslander-Yorke's chaos \citep{blanchard}.
\begin{definition}
  (Chaos in Wiggins's sense)
  A continuous map $F: X\rar X$ is said to be chaotic on $X$ if
  \begin{enumerate}
    \item $F$ is \textit{topologically transitive}.
    \item $F$ has \textit{sensitive dependence on initial conditions}.
  \end{enumerate}
  \index{definition of chaos!Wiggins}
\end{definition}
%%%
We give some examples to illustrate the two definitions.
\begin{example}
  (Rotation on a circle \citep{devaney})
  Consider a unit circle $S^1$, and use the polar coordinates $e^{2\pi i \theta}$ to represent each point on the circle.
  Then, we can associate a point with an angle $\theta \in [0, 1)$, identify $S^1$ with $[0,1)$. % indent]]
  Define the rotation by $\alpha \in \R$ by
  \begin{equation*}
    F: \theta \mapsto \theta + \alpha \quad(\mod{1}).
  \end{equation*}
  We consider two cases: when $\alpha$ is rational and when it is not.
  First, assume that $\alpha$ is rational.
  We may suppose that $\alpha = p/q$ for some $p, q \in \Z$, $q \neq 0$.
  Then, for each $\theta \in [0, 1)$, we have %]
  \begin{equation*}
    \itr{F}{q}(\theta) = \theta + q \cdot \frac{p}{q} = \theta + p = \theta \quad(\mod{1}).
  \end{equation*}
Therefore, each point in $S^1$ is periodic. %]
  Hence, the periodic points are dense in $S^1$. %]
  However, the mapping is not sensitive to initial conditions, since the distance between any pair of points $\theta, \phi \in [0,1)$ remain the same under iterations by $F$.  %]
  Thus, $F_\alpha$ is not chaotic when $\alpha$ is rational.

  Next, assume that $\alpha$ is irrational.
  As in the previous case, $F_\alpha$ is not sensitive to initial conditions.
In this case, however, the mapping has no periodic point, the set of periodic points is not dense in $S^1$. %]
The argument for this is by contradiction: suppose that there exists a periodic point $\theta$ whose period is $p$.
Then,
\begin{equation*}
  \itr{F}{p}(\theta) = \theta 
  \Rar
  \theta + p\alpha \quad(\mod{1}) = \theta
  \Rar
  p\alpha \quad(\mod{1}) = 0
  \Rar
  p\alpha = k
\end{equation*}
for some $k \in \Z$.
The last equality implies that $\alpha = \frac{k}{p}$, but this is a contradiction because $\alpha$ is not a rational number.
Therefore, $F_\alpha$ is not chaotic for any $\alpha$.

On the other hand, an irrational rotation of the unit circle is topologically transitive.
To see this, consider two points $\theta, \phi \in S^1$, and a neighborhood $U = \oball{\epsilon}{\phi}$ for any $\epsilon$.
If there exists $n$ such that $\itr{F_\alpha}{n}(\theta) \in U$, then $F_\alpha$ is topologically transitive.
Let us begin by choosing a $N > 0$ such that $1/N < \epsilon$.
Then, pick $N$ points from the orbit $O(\theta)$, and call them $\theta_1, \ldots, \theta_N$.
Since $\theta$ is not a periodic point, $\theta_1, \ldots, \theta_N$ are distinct.
Then, by the pigeon hole principle, there exist $1 \leq m,k \leq N$ such that
\begin{equation*}
  \abs{ \theta_m - \theta_k } \leq \frac{1}{N} < \epsilon.
\end{equation*}
Without loss of generality, we may assume that $m > k$.
It follows that 
\begin{equation*}
  \itr{F}{m - k}: \theta \mapsto \theta + \beta \quad(\mod{1}),
\end{equation*}
and $\beta < \epsilon$.
Since $\beta < \epsilon < 2\epsilon$, there exists $n$ such that
\begin{equation*}
  \itr{F}{n(m-k)}(\theta) \in U.
\end{equation*}
This shows that $F_\alpha$ is topologically transitive when $\alpha$ is irrational.
\end{example}
%%%
\begin{example}
  (Doubling map \citep{devaney})
  Define the doubling map to be 
  \begin{align*}
    D: &S^1 \to S^1  \\
    &\theta \mapsto 2\theta \quad(\mod{1}).
  \end{align*}
  First, we show that $D$ is sensitive to initial conditions. 
  Consider an arbitrary point $\theta \in S^1$, and a nearby point $\theta - \epsilon$ ($\epsilon$ can be arbitrarily small).
  For each $n$, we have
  \begin{equation*}
    \abs{ \itr{D}{n}(\theta) - \itr{D}{n}(\theta - \epsilon) } 
    = 
    n\epsilon.
  \end{equation*}
  Therefore, for any separation constant $0 < \delta < 1$, there exists $n$ such that $\abs{ \itr{D}{n}(\theta) - \itr{D}{n}(\theta - \epsilon) } > \delta$.
  To show that the doubling map has a dense set of periodic points and is topologically transitive, we use a technique called \textit{symbolic dynamics}.
  Chapter~\ref{chap:symbolic} will include the proof that the doubling map is chaotic in Devaney's sense.
\end{example}
The reader might have raised a question on the appropriateness of labeling the doubling map as chaotic.
We think we understand such a simple mapping perfectly well; it's geometry is not complicated either.
Yet, do we want to call this ``chaotic?''
The question will be addressed again in Chapter~\ref{chap:symbolic}.
\begin{example}
  The logistic map $L_\mu: x \mapsto \mu x(1-x)$ (the mapping we saw in the introduction) is chaotic in Devaney's sense on a compact subset, which turns out to be a Cantor set, when $\mu > 2 + \sqrt{5}$.
  The chaotic behavior of $L_\mu$ is discussed in various sources; for example, see \citet{sternberg}.
  We can say that $L_\mu$ is chaotic on $I \equiv [0,1]$, because there exists a compact set on which $L_\mu$ satisfies the three conditions.
  Note that $L_\mu$ is not topologically transitive on $[0,1]$.
\end{example} 
\begin{example}
  (Rotation by magnitude \citep{martelli})
  Let $D$ be the unit disk centered at the origin, and $C_r = \set{v \in \R^2: \norm{v} = r}$.
  Define $F: D \to D$, a rotation, by 
  \begin{equation*}
    F(x) = F(r, \theta) = (r, \theta + r).
  \end{equation*}
For every $r \in (0,1]$, $C_r$ is invariant.
Clearly, $F$ does not have a sensitive dependence on initial conditions on $C_r$.
However, $F$ has sensitive dependence on initial conditions on $D$, which is also an invariant set. 
This example shows why we should include topological transitivity in the definition of chaos.
%To see this, let $x_0 = (r_0, \theta_0)$ and $d > 0$.
%Choose $n$ large enough that $\frac{\pi}{n} < d$ and $r_0 - \frac{\pi}{n} > 0$.
%Let $y_0 = (r_0 - \frac{\pi}{n}, \theta_0)$.
%Then,
%\begin{equation*}
%  x_n = (r_0, \theta_0 + nr_0),\quad
%  y_n = (r_0 - \frac{\pi}{n}, \theta_0 + nr_0 - \pi),
%\end{equation*}
%so
%\begin{equation*}
%  \norm{x_0 - y_0} < d \quad\mbox{and}\quad \norm{y_n - x_n} = \paren*{\frac{\pi}{n}}^2 + \pi^2 > 2 > r_0.
%\end{equation*}
\end{example}
  %%%
\begin{example}
  \citep{martelli}
  This example illustrates a dynamical system that is topologically transitive and sensitive to initial conditions, but does not have dense periodic points.
  Let $D \subseteq \R^2$ a closed unit disk centered at the origin.
  Let $F: D \to D$ be defined as
  \begin{equation*}
    F: (r, \theta) \mapsto (4r(1 - r), \theta + 1).
  \end{equation*}
  The origin is the only fixed point, and $F$ does not have a periodic orbit of period greater than 1.
  Hence, the periodic points of $F$ is not dense in $D$ so that $F$ is chaotic in the sense of Wiggins, but not in the sense of Devaney.
  This example motivates us to think whether \dpp is necessary for the definition of chaos.
  \label{eg:notdpp}
\end{example}
%  \begin{example}
%    (Twist Map \citep{wiggins})
%  \end{example}
%%%)

\section{Redundancy of the Definition}
\citet{silverman} (and also \citet{banks}) showed that \sdic is a necessary condition of \tt and \dpp.
%%%
\begin{theorem}
  (\tt and \dpp Imply \sdic)
  Suppose $F: X\rar X$ is transitive and has dense periodic points in $X$.
  Then there exists $\epsilon > 0$ such that, for any $x\in X$ and any open set $J$ containing at least $x$ and another point, there is a point $y\in J$ and $n\in \N$ with
  \begin{equation*}
    \metric{\itr{F}{n}(x),\itr{F}{n}(y)} > \epsilon.
  \end{equation*}
  \label{thm:silverman}
\end{theorem}

First, we prove a lemma, which states that two distinct periodic orbits cannot intersect.
\begin{lemma}
  Let $F: X\rar X$ be a mapping with (at least) two distinct periodic orbits.
  Then there is a $c > 0$ such that for any $x\in X$ there exists a periodic
  point $p$ such that for each $k$
  \begin{equation*}
    \metric{x, \itr{F}{k}(p)} > c.
  \end{equation*}
  \label{lem:dev1}
  \begin{proof}[Proof of the lemma]
    Let $r$ and $s$ be periodic points with distinct orbits. For each $k$ and $l$ we have
    \begin{equation*}
      \metric{\itr{F}{k}(r), \itr{F}{l}(s)} > 0.
    \end{equation*}
    Choose $c$ so that $2c < \min \metric{\itr{F}{k}(r),\itr{F}{l}(s)}$.
    Then by the triangle inequality
    \begin{equation*}
      2c < \metric{\itr{F}{k}(r),\itr{F}{l}(s)} \leq \metric{\itr{F}{k}(r),x} + \metric{x,\itr{F}{l}(s)}
    \end{equation*}
    for all $k$ and $l$.
    We cannot have $\metric{\itr{F}{k}(r),x} \leq c$ and $\metric{x,\itr{F}{l}(s)} \leq c$, since it violates the inequality just derived.
    Thus,
    \begin{equation*}
      \metric{\itr{F}{k}(r),x} > c \mbox{ or } \metric{x,\itr{F}{l}(s)} > c
    \end{equation*}
    must hold.
  \end{proof}
\end{lemma}
  %
\begin{proof}[Proof of Theorem~\ref{thm:silverman}]
  Let $x$ be any point of $X$ and $J$ any open set containing $x$.
  Since the periodic points of $f$ are dense (in $X$), we may introduce a periodic point
  $q$ of $f$ in $U$ defined as
  \begin{equation*}
    U \ceq J\cap \oball{\epsilon}{x}.
  \end{equation*}
  Let $n$ be the period of $q$.
  Apply Proposition~\ref{lem:dev1} to obtain a periodic orbit $O(p)$ and a constant $c > 0$ such that $\metric{x, O(p)} > c$.
  Let $\epsilon \ceq c/4$ so that $\metric{x, O(p)} > 4\epsilon$.
  Define
  \begin{equation*}
    W_i := \oball{\epsilon}{\itr{F}{i}(p)} \quad\text{ and }\quad V := \bigcap\limits_{i = 1}^{n} \itr{F}{-i}(W_i).
  \end{equation*}
  Since $\itr{F}{i}(p) \in W_i$, we see that $p \in V$, and $V$ is not empty. 
  Also note that $V$ is open.
  By the transitivity property, we can find a $z\in U$ and a positive integer $k$ such that
  $\itr{F}{k}(z) \in V$. Let $j$ be the least integer such that $k < nj$. 
  That is,
  \begin{equation*}
    1 \leq nj - k \leq n.
  \end{equation*}
  Therefore,
  \begin{equation*}
    \itr{F}{nj} (z) = \itr{F}{nj - k} (\itr{F}k(z)) \in \itr{F}{nj-k}(V).
  \end{equation*}
  Also,
  \begin{equation*}
    \itr{F}{nj - k}(V) \subset \itr{F}{nj - k}(\itr{F}{-(nj - k)} W_{nj-k}) 
    = W_{nj-k}
    \equiv \oball{\epsilon}{\itr{F}{nj-k}(p)}.
  \end{equation*}
  The last two equalities imply
  \begin{equation*}
    \metric{\itr{F}{nj}(z), \itr{F}{nj - k}(p)} < \epsilon.
  \end{equation*}
  Furthermore, $\itr{F}{nj}(q) = q$, since $q$ is a $n$-periodic point 
  \begin{equation*}
    \metric{\itr{F}{nj}(q), \itr{F}{nj}(z)} = \metric{q, \itr{F}{nj}(z)}.
  \end{equation*}
  By the triangle inequality, we have
  \begin{equation*}
    \metric{x, \itr{F}{nj-k}(p)} \leq \metric{x,q} + \metric{q, \itr{F}{nj}(z)} + \metric{\itr{F}{nj}(z), \itr{F}{nj - k}(p)}.
  \end{equation*}
  Finally, 
  \begin{align*}
    \metric{\itr{F}{nj}(q), \itr{F}{nj}(z)} 
    &= \metric{q, \itr{F}{nj}(z)}  \\
    &\geq  \metric{x, \itr{F}{nj-k}(p)} - \metric{x,q} - \metric{\itr{F}{nj}(z), \itr{F}{nj - k}(p)}  \\
    &> 4\epsilon - \epsilon - \epsilon 
    = 2\epsilon,
  \end{align*}
  which implies
  \begin{equation*}
    \metric{\itr{F}{nj}(q), \itr{F}{nj}(x)} + \metric{\itr{F}{nj}(x), \itr{F}{nj}(z)} 
    \geq \metric{\itr{F}{nj}(q), \itr{F}{nj}(z)}
    > 2\epsilon.
  \end{equation*}
  It follows from the inequality that 
  \begin{equation*}
    \metric{\itr{F}{nj}(x), \itr{F}{nj}(z)} \geq \epsilon \quad\mbox{ or }\quad \metric{\itr{F}{nj}(x), \itr{F}{nj}(q)} \geq \epsilon 
  \end{equation*}
  must hold.
\end{proof}

Thus, we can regard \dpp and \tt as sufficient conditions for \sdic.
Using this result, we can restate Devaney's definition in a more concise form, since the third condition, sensitive dependence on initial condition, is redundant.
\begin{definition}
  (Chaos in Devaney's sense, redefined) 
  A continuous map $F: X\rar X$ is said to be chaotic on $X$ if
  \begin{enumerate}
    \item $P(F) = X$.
    \item $F$ is \textit{topologically transitive}.
  \end{enumerate}
\end{definition}
%%%
A natural question to raise at this point is whether we can further reduce the definition to one condition, or whether other combinations of two conditions imply the other.
The answers to both of these question are negative; \citet{assaf} showed that, in general, \dpp and \sdic do not imply \tt.
Also, Example~\ref{eg:notdpp} shows that \sdic and \tt do not necessarily imply \dpp.
However, when we restrict our space to a compact interval, \tt implies the other two \citep{silverman}.
%A system is chaotic in $X$ in the sense of Devaney if and only if for every pair of open sets in $X$ there exists a periodic orbit which visits both sets (Touhey, 1997).

\section{Chaos is a Topological Property}
All conditions employed in Devaney's definition are topological invariants, that is, the properties are preserved under conjugacy by a continuous mapping.
We show that \dpp and \tt are topological invariants.
It follows that sensitive dependence on initial conditions is also a topological invariant.
\begin{theorem}
  (\tt is a topological invariant) 
  Let $F: X \to X$ and $G: Y \to Y$ be continuous functions.
  Suppose that $G$ is semi-conjugate to $F$.
  If $F$ is transitive, then $G$ is transitive.
  \label{thm:conj-trans}
  \begin{proof}
    Let $A$ and $B$ be open sets of $Y$.
    Since $\phi^{-1}$ is continuous, $\phi^{-1}(A)$ is a union of open sets in $X$.
    Let $A'$ be one of the open sets constituting $\phi^{-1}(A)$.
    Similarly, let $B'$ be one of the open sets constituting $\phi^{-1}(B)$.
    By the transitivity of $F$, there exists a positive integer $n$ and $x \in B'$ such that $\itr{F}{n}(x)\in A'$.
    Let $y \ceq \phi(x)$.
    Then, $y \in B$, and 
    \begin{equation*}
      \itr{G}{n}(y) 
      = \itr{G}{n}(\phi(x))
      = \phi(\itr{F}{n}(x)) \in \phi(A') \subseteq A.
    \end{equation*}
  \end{proof}
\end{theorem}
%%%
We prove two lemmas, from which the theorem of our interest immediately follows.
\begin{lemma}
  Let $\phi$ be a continuous and surjective mapping.
  If $D$ is dense in $X$, then $\phi(D)$ is dense in $Y$.
  \label{thm:conj-dense}
  \begin{proof}
    Let $A \subseteq Y$ be an open set.
    Since $\phi$ is continuous, $\phi^{-1} (A)$ is a union of open sets.
    Let $B$ be one of the open sets that constitutes $\phi^{-1}(A)$.
    The intersection $B \cap D$ is nonempty, because $D$ is dense in $Y$.
    For each $y \in B \cap D$, we have $\phi(y) \in A \cap \phi(D)$.
  \end{proof}
\end{lemma}
\begin{lemma}
  (Periods are preserved under conjugacy)
  Suppose $G$ is semi-conjugate to $F$.
  If $p$ is a $n$-periodic point of $F$, then $\phi(p)$ is a $n$-periodic point of $G$.
  \label{thm:conj-per}
  \begin{proof}
    $\itr{F}{n}(p) = p$ implies that $\itr{G}{n}(\phi(p)) = \phi(\itr{F}{n}(p)) = \phi(p)$.
  \end{proof}
\end{lemma}
\begin{theorem}
  (\dpp is a topological invariant)
  Suppose that $G$ is semi-conjugate to $F$.
  If $\mathrm{P}(F)$ is dense in $X$, then $\mathrm{P}(G)$ is dense in $Y$.
  \label{cor:conj-dense-per}
  \begin{proof}
    It follows from Theorem~\ref{thm:conj-per} that $\phi(\mathrm{P}(F)) \subseteq \mathrm{P}(G)$.
    Then, by Theorem~\ref{thm:conj-dense}, $\mathrm{P}(G)$ is a dense subset of $Y$.
  \end{proof}
\end{theorem}
Finally, we obtain the main result of this section as a corollary.
\begin{corollary}
  (\sdic is a topological invariant)
  Suppose that $G$ is semi-conjugate to $F$.
  If $F$ is sensitive to initial conditions, then $G$ is sensitive to initial conditions.
  \label{cor:conj-sdic}
  \begin{proof}
    This follows from Theorem~\ref{thm:silverman}, Theorem~\ref{thm:conj-trans}, and Theorem~\ref{thm:conj-dense}.
  \end{proof}
\end{corollary}
Hence, chaos in Devaney's sense is a topological invariant.
%%%

\section{Martelli's Definition}
In this final section of the chapter, we introduce a definition that is equivalent to Wiggins's definition.
\citet{martelli} prove the equivalence when $X$ is a compact subset of $\R^n$, but the proof sketched in their paper also works for a more general setting, namely when $X$ is a compact metric space.
We first state the definition, then define the terms used in it.
%%%
\begin{definition}
  (Chaos in Martelli's sense)
  Suppose $F: X \to X$ is a continuous map, where $X$ is a compact metric space.
  Then $F$ is said to be \textit{chaotic} in $X$ if there exists $x \in X$ that meets the following conditions
  \begin{enumerate}
    \item $L(x) = X$.
    \item $O(x)$ is unstable in $X$.
  \end{enumerate}
  \label{defn:martelli}
  \index{definition of chaos!Martelli}
\end{definition}
\begin{definition}
  (Limit Points of an Orbit/Limit Set)
  Let $F: X\to X$ be an continuous mapping.
  Let $O(x)$ be the orbit of $x \in X$.
  $z \in X$ is called a \textit{limit point} of $O(x)$, if there exists a subsequence of $O(x)$ that converges to $z$.
  The \textit{limit set} of $O(x)$, denoted $L(x)$, is the set of all limit points $O(x)$.
  \label{def:limset}
  \index{limit set}
\end{definition}
  %%%
\begin{definition}
  (Unstable Point)
  Let $F: X\to X$ be an continuous mapping, and $x$ be a member of $X$.
  $x$ is called an \textit{unstable point}, if there exists $\epsilon > 0$, which we refer to as the \textit{separation constant} (as in the definition of \sdic), such that, for any neighborhood $N$ of $x$, there exists $y \in N$ and a positive integer $n$ such that 
  \begin{equation*}
    \metric{\itr{F}{n}(x), \itr{F}{n}(y)} > \epsilon.
  \end{equation*}
  \label{defn:unstable-orbit}
  \index{unstable orbit}
\end{definition}
Note that if $F$ has a sensitive dependence on initial conditions on $X$, then all points in $X$ are unstable points, and there exists a globally defined separation constant.
Therefore, sensitivity is a stronger condition than instability.

As I mentioned in the beginning of this section, Martelli's definition is equivalent to Wiggins's.
\begin{theorem}
  (Equivalence of Martelli's and Wiggins's definitions)
  Let $F: X \to X$, where $X$ is a compact metric space.
  $F$ is chaotic in Martelli's sense if and only if $F$ is chaotic in Wiggins's sense.
  \label{thm:martelli-wiggins}
  \begin{proof}
    We prove this theorem in two parts, Proposition~\ref{prop:martelli-wiggins1} and Proposition~\ref{prop:martelli-wiggins2}.
  \end{proof}
\end{theorem}
We use the following result in the proof of Proposition~\ref{prop:martelli-wiggins1}.
\begin{proposition}
  \citep{silverman}
  Suppose $X$ is a metric space with no isolated point.
  If $F: X\to X$, a continuous map, has a dense orbit, then $F$ is topologically transitive.
  \begin{proof}
    See \citep{silverman}.
  \end{proof}
  \label{lem:dob-transitivity}
\end{proposition}
\begin{proposition}
  Let $X$ be a compact metric space with no isolated point, and $F: X \to X$ be a continuous mapping.
  $F$ is topologically transitive in $X$ if and only if there exists $x \in X$ such that $L(x) = X$.
  \label{prop:martelli-wiggins1}
  \begin{proof}
    We first show that the existence of an orbit whose limit set is $X$ implies topological transitivity.
    $O(x)$ is a dense orbit.
    By Proposition~\ref{lem:dob-transitivity}, $F$ is topologically transitive.

    Next, suppose that $F$ is topologically transitive.
    Let $\seq{\epsilon_i}{\infty}{i=1}$ be a monotonically decreasing sequence of positive real numbers whose limit is 0.
    Since $X$ is totally bounded, for each $\epsilon_i$, there exists a finite open cover $\mathcal{C}_i$ of $X$ whose cardinality is $k_i$, say $\mathcal{C}_i \ceq \set{A_{i,1}, \ldots, A_{i, k_i}}$, such that for each $1 \leq j \leq k_i$, $\diam(A_{i,j}) < \epsilon_i$.
    Define an ordering of $A_{i,j}$ as follows
    \begin{equation*}
      A_{1,1}, A_{1,2}, \ldots, A_{1,k_1}, A_{2,1}, \ldots, A_{2,k_2}, A_{3,1}, \ldots.
    \end{equation*}
    Since $F$ is topologically transitive, by Proposition~\ref{prop:transitivity}, there exists a point $x \in X$ such that some subsequence of its orbit visits each $A_{i,j}$ in this order.
    Now, choose any point $y \in X$.
    Since $\mathcal{C}_i$ is an open cover, for each $i$, there exists $j_i$ such that $y \in A_{i,j_i}$.
    We may introduce a subsequence $\set{\itr{F}{n(i)}(x)}$ of $O(x)$ such that, for each $i$, 
    \begin{equation*}
      \itr{F}{n(i)}(x) \in A_{i,j_i}.
    \end{equation*}
    By construction, $A_{i,j_i}$ is a sequence of sets whose diameters monotonically decrease. 
    It follows that $\set{\itr{F}{n(i)}(x)}$ converges to $y$.
    Since the choice of $y$ was arbitrary, we have $L(x) = X$.
  \end{proof}
\end{proposition}
%%%

Now we prove the second part of the theorem.
Trivially, if $F$ has a sensitive dependence on initial conditions on $X$, then for any $x \in X$, $O(x)$ is unstable.
On the other hand, the existence of an unstable orbit does not, in general, imply sensitivity of $F$.
However, by requiring an additional condition that the limit set of the same orbit equals the entire space, we obtain the desired result.
We first show that if $x$ is an unstable point, then any point in its orbit is also unstable.
Moreover, we show that the separation constant of $x$ is a property belonging to all points in the orbit of $x$.
\begin{proposition}
  Suppose there exists an unstable point $x \in X$ such that $L(x) = X$.
  Then, for each positive integer $n$, $\itr{F}{n}(x)$ is also an unstable point, and its separation constant is the same as that of $x$.
  \label{prop:unstable-orbit}
  \begin{proof}
    Let the separation constant of $x$ be $\epsilon$, and suppose that for some $y$ in any neighborhood of $x$, and for some positive integer $p$,
    \begin{equation*}
      \metric{\itr{F}{p}(x), \itr{F}{p}(y)} > \epsilon.
    \end{equation*}
    For any neighborhood $M$ of $\itr{F}{n}(x)$, choose some point $z \in M$.
    By Proposition~\ref{prop:martelli-wiggins1}, $F$ is topologically transitive.
    It follows that, for any $\delta > 0$, there exist positive integers $m$ and $k$ such that
    \begin{equation*}
      \itr{F}{n+m}(x) \in \oball{\delta}{x}
      \quad \mbox{ and } \quad
      \itr{F}{k}(z) \in \oball{\delta}{y}.
    \end{equation*}
    Then, by continuity, we may have $\metric{\itr{F}{n+m+p}(x), \itr{F}{p}(x)}$ and $\metric{\itr{F}{k+p}(z), \itr{F}{p}(y)}$ as small as we wish (*elaborate).
    Hence, $\itr{F}{n}(x)$ is also an unstable point, and its separation constant is $\epsilon$.
  \end{proof}
\end{proposition}
\begin{proposition}
  Suppose there exists an unstable point $x \in X$ such that $L(x) = X$.
  Then, $F$ has a sensitive dependence on initial conditions.
  \label{prop:martelli-wiggins2}
  \begin{proof}
    Suppose that $x$ is an unstable point such that $L(x) = X$.
    Then, for each $y\in X$ and $\delta > 0$ there exists $n>0$ such that 
    \begin{equation*}
      \metric{\itr{F}{n}(x), y} \leq \frac{\delta}{2}.
    \end{equation*}
    Let $\epsilon$ be the separation constant of $x$.
    We prove that $F$ has a sensitive dependence on initial conditions by showing that $\itr{F}{n}(x)$ eventually separates from $y$ by $\epsilon/2$.
    If there exists $p$ such that 
    \begin{equation*}
      \metric{\itr{F}{n+p}(x), \itr{F}{p}(y)} > \frac{\epsilon}{2},
    \end{equation*}
    then we are done.
    So suppose otherwise.
    By Proposition~\ref{prop:unstable-orbit}, there exists a point $z \in \oball{\delta/2}{\itr{F}{n}(x)} (z \neq y)$ such that 
    \begin{equation*}
      \metric{\itr{F}{n+q}(x), \itr{F}{q}(z)} > \epsilon,
    \end{equation*}
    for some $q$.
    Note that
    \begin{equation*}
      \metric{y, z} \leq \metric{y, \itr{F}{n}(x)} + \metric{\itr{F}{n}(x), z} \leq \delta,
    \end{equation*}
    by the triangle inequality.
    Moreover,
    \begin{align*}
      \epsilon &< \metric{\itr{F}{n+q}(x), \itr{F}{q}(z)} \\
      &\leq \metric{\itr{F}{n+q}(x), \itr{F}{q}(y)} + \metric{\itr{F}{q}(y), \itr{F}{q}(z)}  \\
      &\leq \frac{\epsilon}{2} + \metric{\itr{F}{q}(y), \itr{F}{q}(z)}
    \end{align*}
    Therefore,
    \begin{equation*}
      \metric{\itr{F}{q}(y), \itr{F}{q}(z)} > \frac{\epsilon}{2}.
    \end{equation*}
    Since the choice of $y$ and $\delta$ were arbitrary, we conclude that $F$ has sensitive dependence on initial conditions, and $\epsilon/2$ is the separation constant for any point in $X$.
  \end{proof}
\end{proposition}
This completes the proof of Theorem~\ref{thm:martelli-wiggins}.


%\begin{definition}
%  (Expansiveness) $F: J\rar J$ is expansive if there exists $\epsilon > 0$
%  such that, for any $x,y\in J$, there exists $n$ such that
%  $\norm{f^{n}(x)-f^{n}(y)} > \epsilon$.
%  \index{defn:expansiveness}
%\end{definition}
%\begin{definition}
%  (Structural Stability) $F: J \rar J$ is said to be $C^r$-structurally
%  stable on $J$, if there exists $\epsilon > 0$ such that whenever
%  $d_r(f,g) < \epsilon$ for $g: J\rar J$, it follows that $f$
%  is topologically conjugate to $g$.
%  \index{defn:structural stability}
%\end{definition}

%With the notion of limit sets, we may speak of aympototical periodicity of an orbit.
%\begin{definition}
%  (Asymptotically periodic orbit)
%  An orbit $O(x)$ is said to be \textit{asymptotically periodic} if its limit set is a periodic orbit.
%  \label{def:asymporb}
%  \index{asymptotically periodic!orbit}
%\end{definition}
%
%\begin{definition}
%  (Aperiodic orbit)
%  The orbit $O_F(x)$ is said to be \textit{aperiodic} if its limit set $L_F(x)$ is not finite.
%  \label{def:aporbit}
%  \index{aperiodic!orbit}
%\end{definition}

\bibliographystyle{../../bibliography/pjgsm}
\bibliography{../../bibliography/thesis}

\printindex
\end{document}
