\documentclass[10pt,twoside,draft]{book}
\usepackage{../../thesis}
\graphicspath{ {../../images/} }

\begin{document}
Lyapunov exponent, however, gives some mathematical content to the word sensitive dependence on initial conditions, and also illustrates the idea of invariants.
%Lyapunov exponents are sometimes called characteristic exponents.
%Also, \citet{young} gave a lower bound of the Hausdorff dimension in terms of the Lyapunov numbers, so the geometry of a dynamical system is not unrelated to sensitive dependence on initial conditions.
\begin{definition}
  (Lyapunov exponent)
  Let $F: \R \to \R$ be a differentiable mapping, and $x_0 \in \R$.
  We define the Lyapunov exponent of $O(x_0)$.
  We use the following notation $x_i \ceq \itr{F}{i}(x_0)$.
  Define Jacobian matrices
  \begin{equation*}
    A_0 \ceq F'(x_0),
    \; A_1 \ceq F'(x_1),
    \; \ldots,
    \; A_n \ceq F'(x_n).
  \end{equation*}
  Then, denote the spectral radii (i.e. the maximum of the absolute values of eigenvalues) of $A_i$ by $\rho_i(x_0)$.
  Then the Lyapunov number $N(x_0)$ of $O(x_0)$ is defined as 
  \begin{equation*}
    N(x_0) = \lim\limits_{n \to \infty} (\rho_n(x_0))^{1/n}.
  \end{equation*}
  Then, we define the Lyapunov exponent to be the log of the Lyapunov number.
  \begin{equation*}
    \Lambda(x_1) = \log N(x_1).
  \end{equation*}
  In particular,
  \begin{equation*}
    \Lambda(x_1) = \lim\limits_{n \to \infty} \paren{\log\abs{F'(x_n)} + \log\abs{F'(x_{n-1})} + \cdots + \log\abs{F'(x_1)}}
  \end{equation*}
  \index{Lyapunov number}
\end{definition}
%(Lyapunov exponent is computationally preferable to the Lyapunov number.)
%It can be shown that the Lyapunov exponent of a map is (almost everywhere) equivalent to the Lyapunov exponent of any orbit of a map (Lyapunov exponent is independent of orbit).
\begin{proposition}
  Suppose $h'(x) \neq 0$ and
  \begin{equation*}
    \lim\limits_{n\to \infty} \frac{\log\abs{h'(x_n)}}{n} = 0.
  \end{equation*}
  Then, the Lyapunov numbers of one-dimensional dynamical systems that are conjugate is the same 
  \label{prop:lyap-conj}
  \begin{proof}
    \begin{equation*}
      h'(x_{n+1})F'(x_n)\ldots F'(x_1) = G'(y_n)\ldots G'(y_1)h'(x_1).
    \end{equation*}
    Then 
    \begin{align*}
      \Lambda(x_1) &= \lim\limits_{n\to \infty} \frac{1}{n}\paren{\log\abs{F'(x_n)} + \cdots + \log\abs{F'(x_n)} } \\
      &= \lim\limits_{n\to \infty} \frac{1}{n}\left(\log\abs{F'(x_n)} + \cdots + \log\abs{F'(x_n)} + \log\abs{h'(x_{n+1})}\right) \\
      &= \lim\limits_{n\to \infty} \frac{1}{n} \paren{ \log\abs{G'(x_n)} + \cdots + \log\abs{G'(x_n)} + \log\abs{h'(x_1)} } \\
      &= \lim\limits_{n\to \infty} \frac{1}{n} \paren{ \log\abs{G'(x_n)} + \cdots + \log\abs{G'(x_n)} } \\
      &= \Lambda(y_1),
    \end{align*} 
    as desired.
  \end{proof}
\end{proposition}
%%%
Lyapunov exponent is a convenient method of seeing if a system is sensitive to initial conditions, and it is widely used in experimental settings.
Even so, Lyapunov exponent should be treated with some care, as the next example suggests.
\begin{example}
  \citep{wiggins}
  Consider the following vector field:
  \begin{equation*}
    \dot{x} = ax,
  \end{equation*}
  where $x \in \R$, and $a \in \R$ is a constant.
  The Lyapunov exponent of each orbit is $a > 0$.
\end{example}
\bibliographystyle{../../bibliography/pjgsm}
\bibliography{../../bibliography/thesis}

\end{document}
%\begin{definition}
%  (Lyapunov number, alternative)
%  Given a map $F$ and a $p-$periodic point $x_p$, the Lyapunov number of the point $N(x_p)$ is
%  defined to be the spectral radius of the Jacobian matrix of the map.
%  \begin{equation*}
%    N(x_p) = \brac{ \rho \paren{ F^p(x_p)}}^{1/p}
%  \end{equation*}
%  \label{def:lyapnum}
%  \index{Lyapunov number}
%\end{definition}
%The alternative definition can be shown to be equivalent to the first definition.
%
%In some cases, this yet another alternate definition is more convenient.
%This version ignores transient states.
%\begin{definition}
%  (Lyapunov Number, yet another alternative)
%  \begin{equation*}
%    N(x_1, k) = N(x_k)
%  \end{equation*}
%  Then the Lyapunov number is defined to be
%  \begin{equation*}
%    N(x_1) = \lim\limits_{k \to \infty} N(x_1,k).
%  \end{equation*}
%\end{definition}

