\documentclass[10pt,twoside]{book}
\usepackage{../../thesis}
\graphicspath{ {../../images/} }

\makeindex
\begin{document}
\chapter{Symbolic Dynamics}
\label{chap:symbolic}
Recall that in Chapter~\ref{chap:devaney}, we postponed the proof that the doubling map is chaotic.
In the present chapter, we complete the proof, and we introduce the technique called the \textit{symbolic dynamics} as we proceed through the proof.
Then, we introduce a definition of chaos, which is motivated by symbolic dynamics, by Block and Coppel.

%%%

\section{The Shift Map}
We prove that $(S^1, D)$ is conjugate to the (full) \textit{one-sided shift} ($\Sigma$, $\sigma$), defined as follows.

\begin{definition}
  (One-sided shift)
  Let $S$ be a set of finite number of symbols, and let $\Sigma$ be a set of bi-infinite sequences of the form
  \begin{equation*}
    s = \set{ s_0 s_1 s_2 \cdots s_n \cdots },
  \end{equation*}
  where $s_i \in S$.
  We take a finite number of integers, $\set{1, \ldots, n}$ as our symbols.
  Also, we put the absolute value metric on $S$.
  Define the metric on $\Sigma$ as follows:
  \begin{equation*}
    d(a,b) \ceq \sum\limits_{i = 0}^{\infty} \frac{\abs{a_i - b_i}}{2^i}.
  \end{equation*}
  The metric induces the topology on $\Sigma$.
  Let $\sigma: \Sigma \to \Sigma$ defined as:
  \begin{equation*}
    \sigma: \set{ s_0 s_1 s_2 \cdots s_n \cdots } 
    \mapsto 
    \set{ s_1 s_2 \cdots s_n \cdots }.
  \end{equation*}
  We call the dynamical system ($\Sigma$, $\sigma$) an \textit{one-sided shift}.
  \index{one-sided shift}
\end{definition}
%%%
Before we proceed further, we verify the validity of the definition.
\begin{proposition}
  $d$ is a metric.
  \label{prop:symb-metric}
  \begin{proof}
    Clearly, $d(a,b) \geq 0$ for each $a,b \in \Sigma$, and if $a = b$ then $d(a,b) = 0$.
    $d(a,b) = 0$ implies that $a_i = b_i$ for each $i$, so $a = b$.
    We are left to prove the triangle inequality.
    Suppose $a,b,c \in \Sigma$.
    We have
    \begin{align*}
      d(a,b) + d(b,c)
      &\equiv \sum\limits_{i = 0}^{\infty} \frac{\abs{a_i - b_i}}{2^i}  +  \sum\limits_{i = 0}^{\infty} \frac{\abs{b_i - c_i}}{2^i}  \\
      %
      &= \sum\limits_{i = 0}^{\infty} \frac{\abs{a_i - b_i} - \abs{b_i - c_i}}{2^i}  \\
      %
      &\geq \sum\limits_{i = 0}^{\infty} \frac{\abs{a_i - c_i}}{2^i}  \\
      &= d(a,c)
    \end{align*}
    by the triangle equality for real numbers.
  \end{proof}
\end{proposition}
%%%
Next we note the following properties of $S$, the set of finite symbols equipped with the absolute value metric.
\begin{proposition}
  $S$ equipped with the absolute value metric is 
  \begin{enumerate}[(i)]
    \item compact
    \item totally disconnected
  \end{enumerate}
  \begin{proof}
    \begin{enumerate}
      \item
        $S$ is finite.
        %%%
      \item 
        $S$ is a discrete space.
        Hence, it is totally disconnected.
        %%%
    \end{enumerate}
  \end{proof}
\end{proposition}
%%%
These properties of the underlying space are inherited to the space of infinite sequences of symbols, $\Sigma$, which is an infinite product of $S$.
\begin{proposition}
  $\Sigma$ equipped with the metric $d$ is 
  \begin{enumerate}[(i)]
    \item compact
    \item totally disconnected
    \item perfect.
  \end{enumerate}
  \begin{proof}
    \begin{enumerate}
      \item 
        It follows from Tychonov's theorem.
        %%%
      \item
        A product of totally disconnected space is also totally disconnected.
        See \citet{dugundji}.
        %%%
      \item
        We prove that $\Sigma \equiv S^\N$ is perfect when $S$ only consists of two symbols, i.e. $S \ceq = \set{0,1}$ to simplify the proof.
        The same result can be proved using the same method when $S$ contains more than two symbols.
        Let $s$ an arbitrary member of $\Sigma$.
        We show that, for each $\epsilon > 0$, $\oball{\epsilon}{s}$ contains a point $t \neq s$.
        We may suppose that $2^{-k} < \epsilon$ for some $k \in \N$.
        Also suppose that
        \begin{align*}
          s = s_0 s_1 s_2 \cdots.
        \end{align*}
        Define $t$ by
        \begin{equation*}
          t_i = \begin{cases}
            &s_i \mbox{ if } i \leq k  \\
            &1 - s_i \mbox{ otherwise.}
          \end{cases}
        \end{equation*}
        Thus, $t$ agrees with $s$ on the first $k+1$ symbols, and differs after the $k+1$-th symbol.
        $t$ is clearly in $\Sigma$.
        Note that
        \begin{equation*}
          d(s,t) = \sum\limits_{i = k+1}^{\infty} \frac{1}{2^i}
          = \sum\limits_{i = 0}^{\infty} \frac{1}{2^i} - \sum\limits_{i = 0}^{k} \frac{1}{2^i},
          = 2 - (2 - \frac{1}{2^{k}})
          = 2^{-k} < \epsilon.
        \end{equation*}
        so $t \in \oball{\epsilon}{s}$.
        Hence, $\Sigma$ is perfect.
        %%%
    \end{enumerate}
  \end{proof}
\end{proposition}
%%%
Thus, $\Sigma$ is a \textit{Cantor set}.
Knowing the structure of $\Sigma$ will be useful when we construct a conjugacy between the doubling map and the full one-sided shift.
Since the properties of $\Sigma$ that we mentioned in the proposition are topological invariants \citep{dugundji}, any dynamical system that is conjugate to the full one-sided shift also possesses these properties.
%%%
Now we show that $\sigma$ is continuous and surjective.
\begin{proposition}
  ($\sigma$ is a homeomorphism)
  \begin{enumerate}[(i)]
    \item $\sigma$ is continuous.
    \item  $\sigma$ is surjective and two-to-one..
  \end{enumerate}
  \begin{proof}
    \begin{enumerate}[(i)]
      \item 
        Fix $\epsilon > 0$.
        We may assume that $2^{-k} < \epsilon$ for some positive integer $k$.
        For any $a \in \Sigma$, if $b \in \Sigma$ satisfies
        \begin{equation*}
          d(a,b) = \sum\limits_{i=0}^{\infty} \frac{\abs{a_i - b_i}}{2^i} < 2^{-(k+1)},
        \end{equation*}
        then 
        \begin{align*}
          d(\sigma(a), \sigma(b)) 
          &= \sum\limits_{i=1}^{\infty} \frac{\abs{a_i - b_i}}{2^i} 
          = \sum\limits_{i=0}^{\infty} \frac{\abs{a_i - b_i}}{2^i} - \abs{a_0 - b_0}  \\
          &< 2^{-(k+1)} - \abs{a_0 - b_0}
          < 2^{-k}
          < \epsilon.
        \end{align*}
        This shows that $\sigma$ is continuous.
        %%%
      \item  
        Each $s \in \Sigma$ takes the form
        \begin{equation*}
          s = s_0s_1s_2 \cdots.
        \end{equation*}
        $\sigma$ maps
        \begin{equation*}
          0 s_0s_1s_2 \cdots \mbox{ and } 1 s_0s_1s_2 \cdots
        \end{equation*}
        to $s$.
        Hence, $\sigma$ is surjective and two-to-one.
        %%%
    \end{enumerate}
  \end{proof}
  \label{prop:sigma-cont}
\end{proposition}
%%%

\begin{theorem}
  The full one-sided shift is chaotic in Devaney's sense.
  \begin{proof}
    \citep{sternberg}
  \end{proof}
\end{theorem}

%%%

\section{The Doubling Map is Chaotic}
In this section, we use $S = \set{0,1}$, so that an element of $\Sigma$ is an infinite sequence of 0 and 1.
To gain some insights of the dynamics of the one-sided shift, we shall spend some time on discussing what $d$ and $\sigma$ are.
If $S = \set{0,1}$, then the metric on $d$ can be written as
\begin{equation*}
  d(a,b) = \sum\limits_{i = 0}^{\infty} \frac{\delta_{i}}{2^i},
\end{equation*}
where
\begin{equation*}
  \delta_i = 
  \begin{cases}
    &0 \quad \mbox{ if } a_i = b_i  \\
    &1 \quad \mbox{ otherwise.}
  \end{cases}
\end{equation*}
Thus, two infinite sequences are ``close'' to each other if they agree on a long frontal block.
We think of an infinite sequence in $\Sigma$ as a binary expansion of some real number.
In this scheme, $\sigma$ corresponds to multiplication of a binary number by $2$.

As a consequence, ($S^1$, $D$) possesses any topological property of ($\Sigma$, $\sigma$), and vice versa.

Let us make a few general remarks on symbolic dynamics.
In this chapter, we only use an one-sided shift, but we can also define the \textit{two-sided shift} by replacing $\Sigma$ with a space of infinite \textit{bi-infinite} sequences (i.e. sequences of the form $s = \set{ \cdots s_{-n} \cdots s_{-1} * s_0 \cdots s_n \cdots }$).
An one-sided shift is used to study a non-invertible maps, while two-sided shift is used to study an invertible map.
In the two-sided shift, the shift map is also injective.
It is easy to see that the full two-sided shift $(\Sigma, \sigma)$ is also chaotic in Devaney's sense.
\citet{wiggins} has a exposition on the dynamics of the horseshoe map using symbolic dynamics.
The horseshoe map has a geometrically complicated dynamics.
However, we can understand the dynamics of map on its attractor by constructing a conjugacy between the horseshoe map on its attractor and the full two-sided shift.
An one-sided shift can be similarly used to study the dynamics of a mapping.
\citet{sternberg} has an exposition on the study of the attractor of the logistic map.

%%%

\section{Defining Chaos Using Symbolic Dynamics}
The two-sided shift, which we introduced in the previous section, is used for an invertible map (i.e. a function with an inverse).
\citet{blockcoppel} defines chaos using the one-sided shift, a symbolic dynamics used to encode a non-invertible mapping.
\begin{definition}
  One-sided shift.
\end{definition}
On a related note, the logistic map, tent map, and doubling maps are all conjugate to the \textit{one-sided shift}, an analogue of the two-sided shift for non-invertible maps (the logistic map is not one-to-one).
%%%
\begin{definition}
  (Chaos in the sense of Block-Coppel \citep{blockcoppel})
  \label{defn:blockcoppel}
  \index{definition of chaos!Block-Coppel}
\end{definition}


\bibliographystyle{../../bibliography/pjgsm}
\bibliography{../../bibliography/thesis}

\printindex
\end{document}

