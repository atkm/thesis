\documentclass[10pt,twoside]{book}
\usepackage{../../thesis}
\graphicspath{ {../../images/} }

\makeindex
\begin{document}
\chapter{Symbolic Dynamics}
\label{chap:symbolic}
Recall that in Chapter~\ref{chap:devaney}, we postponed the proof that the doubling map is chaotic.
In the present chapter, we complete the proof, and we introduce the technique called the \textit{symbolic dynamics} as we proceed through the proof.
Then, we introduce a definition of chaos, which is motivated by symbolic dynamics, by Block and Coppel.
%%%

\section{The Shift Map}
We prove that $(S^1, D)$ is conjugate to the (full) \textit{one-sided shift} ($\Sigma$, $\sigma$), defined as follows.

\begin{definition}
  (One-sided shift)
  Let $S$ be a set of finite number of symbols, and let $\Sigma$ be a set of bi-infinite sequences of the form
  \begin{equation*}
    s = \set{ s_0 s_1 s_2 \cdots s_n \cdots },
  \end{equation*}
  where $s_i \in S$.
  We take a finite number of integers, $\set{1, \ldots, n}$ as our symbols.
  Also, we put the absolute value metric on $S$.
  Define the metric on $\Sigma$ as follows:
  \begin{equation*}
    d(a,b) \ceq \sum\limits_{i = 0}^{\infty} \frac{\abs{a_i - b_i}}{2^i}.
  \end{equation*}
  Thus, two infinite sequences are close to each other if they agree on a long frontal block.
  The metric induces the topology on $\Sigma$.
  Let $\sigma: \Sigma \to \Sigma$ defined as:
  \begin{equation*}
    \sigma: \set{ s_0 s_1 s_2 \cdots s_n \cdots } 
    \mapsto 
    \set{ s_1 s_2 \cdots s_n \cdots }.
  \end{equation*}
  We call the dynamical system ($\Sigma$, $\sigma$) the \textit{full one-sided shift}.
  If $X$ is a (appropriately chosen) subset of $\Sigma$, then $(X, \sigma)$ is called a \textit{subshift}.
  \index{one-sided shift}
\end{definition}
For the definition of a subshift to make sense, a $X$ must be an invarient set of $\sigma$, but it is not the only requirement.
Subshifts provide rich examples of dynamical systems, but we do not discuss them here.
For discussions on subshifts, refer to \citet{kitchens} and \citet{lind}.
%%%

Before we proceed further, we verify the validity of the definition of an one-sided shift.
\begin{proposition}
  $d$ is a metric.
  \label{prop:symb-metric}
  \begin{proof}
    Clearly, $d(a,b) \geq 0$ for each $a,b \in \Sigma$, and if $a = b$ then $d(a,b) = 0$, which implies that $a_i = b_i$ for each $i$.
    Hence $a = b$.
    We are left to prove the triangle inequality.
    Suppose $a,b,c \in \Sigma$.
    We have
    \begin{align*}
      d(a,b) + d(b,c)
      &\equiv \sum\limits_{i = 0}^{\infty} \frac{\abs{a_i - b_i}}{2^i}  +  \sum\limits_{i = 0}^{\infty} \frac{\abs{b_i - c_i}}{2^i}  \\
      %
      &= \sum\limits_{i = 0}^{\infty} \frac{\abs{a_i - b_i} - \abs{b_i - c_i}}{2^i}  \\
      %
      &\geq \sum\limits_{i = 0}^{\infty} \frac{\abs{a_i - c_i}}{2^i}  \\
      &= d(a,c)
    \end{align*}
    by the triangle equality.
  \end{proof}
\end{proposition}
%%%
It suffices for our purpose to use $S \ceq \set{0,1}$.
All properties of $(\Sigma, \sigma)$ that we prove in this section also hold when $S$ is any finite set of symbols.
However, letting $S$ consist of only two symbols greatly simplify proofs, and also we will not require more than two symbols for studying systems in this exposition.
Under our assumption that $S = \set{0,1}$, we can bound the distance between two symbolic sequences if we know that the two sequences agree in the beginning.
\begin{proposition}
  Let $s, t \in \Sigma$.
  If $s_i = t_i$ for $0 \leq i \leq k$, then $d(s,t) \leq 2^{-k}$.
  \label{prop:symbol-bound1}
  \begin{proof}
    $d(s,t)$ achieves its maximum when $s_i \neq t_i$ for all $i \geq k+1$.
    \begin{equation*}
      d(s,t) 
      \leq \sum\limits_{i = k+1}^{\infty} \frac{1}{2^i}
      = \sum\limits_{i = 0}^{\infty} \frac{1}{2^i} - \sum\limits_{i = 0}^{k} \frac{1}{2^i}
      = 2 - (2 - \frac{1}{2^{k}})
      = 2^{-k}.
    \end{equation*}
  \end{proof}
\end{proposition}
Conversely, we can infer from the distance between the two the length of a block on which two sequences agree.
\begin{proposition}
  Let $s, t \in \Sigma$.
  If $d(s,t) < 2^{-k}$, then $s_i = t_i$ for $0 \leq i \leq k$.
  \label{prop:symbol-bound2}
  \begin{proof}
    If $s_i \neq t_i$ for any $0 \leq i \leq k$, then we would have
    \begin{equation*}
      d(s,t) \geq \frac{1}{2^i} \geq 2^{-k}.
    \end{equation*}
    Hence, it must hold that $s_i = t_i$ for each $0 \leq i \leq k$.
  \end{proof}
\end{proposition}
%%%
Next, we note the following properties of $S$, the set of finite symbols equipped with the absolute value metric.
\begin{proposition}
  $S$ equipped with the absolute value metric is 
  \begin{enumerate}[(i)]
    \item compact
    \item totally disconnected
  \end{enumerate}
  \begin{proof}
    \begin{enumerate}
      \item
        $S$ is finite.
        %%%
      \item 
        $S$ is a discrete space.
        Hence, it is totally disconnected.
        %%%
    \end{enumerate}
  \end{proof}
\end{proposition}
%%%
These properties of the underlying space are inherited to the space of infinite sequences of symbols, $\Sigma$, which is an infinite product of $S$.
\begin{proposition}
  $\Sigma$ equipped with the metric $d$ is 
  \begin{enumerate}[(i)]
    \item compact
    \item totally disconnected
    \item perfect.
  \end{enumerate}
  \begin{proof}
    \begin{enumerate}
      \item 
        It follows from Tychonov's theorem.
        %%%
      \item
        A product of totally disconnected space is also totally disconnected.
        See \citet{dugundji}.
        %%%
      \item
        Let $s$ an arbitrary member of $\Sigma$.
        We show that, for each $\epsilon > 0$, $\oball{\epsilon}{s}$ contains a point $t \neq s$.
        We may suppose that $2^{-k} < \epsilon$ for some $k \in \N$.
        Also suppose that
        \begin{align*}
          s = s_0 s_1 s_2 \cdots.
        \end{align*}
        Define $t$ by
        \begin{equation*}
          t_i = \begin{cases}
            &s_i \mbox{ if } 0 \leq i \leq k  \\
            &1 - s_i \mbox{ otherwise.}
          \end{cases}
        \end{equation*}
        Thus, $t$ agrees with $s$ on the first $k+1$ symbols, and differs after the $k+1$-th symbol.
        $t$ is clearly in $\Sigma$.
        Then, by Proposition~\ref{prop:symbol-bound1}, $d(s,t) \leq 2^{-k} < \epsilon$, so that $t \in \oball{\epsilon}{s}$.
        Hence, $\Sigma$ is perfect.
        %%%
    \end{enumerate}
  \end{proof}
\end{proposition}
%%%
Thus, $\Sigma$ is a \textit{Cantor set}.
Knowing the structure of $\Sigma$ will prove useful when we construct a conjugacy between the doubling map and the full one-sided shift.
Also, note that any dynamical system that is conjugate to the full one-sided shift possesses these properties of $\Sigma$ \citep{dugundji}.
%%%
Now we show that $\sigma$ is continuous and surjective.
\begin{proposition}
  ($\sigma$ is a homeomorphism)
  \begin{enumerate}[(i)]
    \item $\sigma$ is continuous.
    \item  $\sigma$ is surjective and two-to-one..
  \end{enumerate}
  \begin{proof}
    \begin{enumerate}[(i)]
      \item 
        Fix $\epsilon > 0$.
        We may assume that $2^{-k} < \epsilon$ for some positive integer $k$.
        For any $a \in \Sigma$, if $b \in \Sigma$ satisfies
        \begin{equation*}
          d(a,b) = \sum\limits_{i=0}^{\infty} \frac{\abs{a_i - b_i}}{2^i} < 2^{-(k+1)},
        \end{equation*}
        then 
        \begin{align*}
          d(\sigma(a), \sigma(b)) 
          &= \sum\limits_{i=1}^{\infty} \frac{\abs{a_i - b_i}}{2^i} 
          = \sum\limits_{i=0}^{\infty} \frac{\abs{a_i - b_i}}{2^i} - \abs{a_0 - b_0}  \\
          &< 2^{-(k+1)} - \abs{a_0 - b_0}
          < 2^{-k}
          < \epsilon.
        \end{align*}
        This shows that $\sigma$ is continuous.
        %%%
      \item  
        Each $s \in \Sigma$ takes the form
        \begin{equation*}
          s = s_0s_1s_2 \cdots.
        \end{equation*}
        $\sigma$ maps
        \begin{equation*}
          0 s_0s_1s_2 \cdots \mbox{ and } 1 s_0s_1s_2 \cdots
        \end{equation*}
        to $s$.
        Hence, $\sigma$ is surjective and two-to-one.
        %%%
    \end{enumerate}
  \end{proof}
  \label{prop:sigma-cont}
\end{proposition}
%%%
Finally, we show that $(X,\sigma)$ is chaotic.
\begin{theorem}
  \citep{sternberg}
  The full one-sided shift is chaotic in Devaney's sense.
  \begin{proof}
    First, we prove that the dense periodic points of $\sigma$ is dense in $\Sigma$.
    Let $s$ be a member of $\Sigma$.
    Fix $\epsilon > 0$, and assume that $2^{-k} < \epsilon$ for some $k$.
    Then, construct $t \in \Sigma$ by letting $t_i = s_i$ for $0 \leq i \leq k$, and repeat $t_0 \cdots t_k$ afterwards, i.e.
    \begin{equation*}
      t = s_0 \cdots s_k s_0 \cdots s_k s_0 \cdots.
    \end{equation*}
    By Proposition~\ref{prop:symbol-bound1}, $d(s,t) \leq 2^{-k} < \epsilon$.
    Clearly, $t$ is periodic.
    Hence, each neighborhood of $s$ contains a periodic point.
    Since the choice of $s$ was arbitrary, we conclude that $P(\sigma)$ is dense in $\Sigma$.

    Next, we show that $\sigma$ is transitive.
    Fix $s,t \in \Sigma$.
    Also fix $\epsilon > 0$ and assume that $2^{-k} < \epsilon$ for some $k$.
    Consider a sequence
    \begin{equation*}
      s' = s_0 \cdots s_k t_0 t_1 t_2 \cdots.
    \end{equation*}
    Since $s'$ agrees $s$ for the first $k+1$ symbols, $s' \in \oball{\epsilon}{s}$.
    Note that we have
    \begin{equation*}
      \itr{\sigma}{k+1}(s') = t.
    \end{equation*}
    Hence, $\sigma$ is transitive.
  \end{proof}
\end{theorem}
%%%
It follows from Theorem~\ref{thm:silverman} that $(X, \sigma)$ is also sensitive to initial conditions.
However, we present a direct proof below, as symbolic dynamics provides a new perspective to our understanding of sensitive dependence on initial conditions.
\begin{proposition}
  The full one-sided shift is sensitive.
  \begin{proof}
    Fix $0 < \delta < 1$, the separation constant.
    Let $s$ be an arbitrary member of $\Sigma$, and for any $\epsilon > 0$, let $\oball{\epsilon}{s}$ be a neighborhood.
    Let $t \neq s$ be any point in $\oball{\epsilon}{s}$.
    We may suppose that $s_i \neq t_i$ for some $n \geq 0$.
    Then,
    \begin{equation*}
      \metric{\itr{\sigma}{n}(s), \itr{\sigma}{n}(t)} \geq 1 > \delta.
    \end{equation*}
  \end{proof}
\end{proposition}
%%%

\section{The Doubling Map is Chaotic}
The key to construct a semi-conjugacy between $(S^1,D)$ and $(\Sigma, \sigma)$ is to think of an infinite sequence in $\Sigma$ as the binary expansion of some real number.
In this scheme, $\sigma$ corresponds to multiplication of a binary number by $2$ (Figure~\ref{fig:doubling}).
\begin{figure}[ht]
  \centering
  \begin{tikzpicture}[scale=3.2]
    \draw[->] (-0.2,0) -- (1.2,0) node[right] {$x$};
    \draw[->] (0,-0.2) -- (0,1.2) node[above] {$D(x)$};
    \foreach \x/\xtext in {0.5/\frac{1}{2}, 1/1}
    \draw[shift={(\x,0)}] (0pt,1pt) -- (0pt,-1pt) node[below] {$\xtext$};
    \foreach \y/\ytext in {0.5/\frac{1}{2}, 1/1}
    \draw[shift={(0,\y)}] (1pt,0pt) -- (-1pt,0pt) node[left] {$\ytext$};

    \draw[domain=0:0.5] plot (\x,{2*\x}) node[below right] {};
    \draw[domain=0.5:1] plot (\x,{2*\x - 1}) node[below right] {};
  \end{tikzpicture}
  \label{fig:doubling}
  \caption{The doubling map, $D(x)$.}
\end{figure}

Based on this observation, we define the homeomorphism
\begin{proposition}
  The full one-sided shift is semi-conjugate to the doubling map on the unit circle.
  \begin{proof}
  \end{proof}
\end{proposition}
%%%
As a consequence, ($S^1$, $D$) possesses any topological property of ($\Sigma$, $\sigma$), and vice versa.

On a related note, the logistic map, tent map, and doubling maps are all conjugate to the \textit{one-sided shift}, an analogue of the two-sided shift for non-invertible maps (the logistic map is not one-to-one).

Let us make a few general remarks on symbolic dynamics.
In this chapter, we only use an one-sided shift, but we can also define the \textit{two-sided shift} by replacing $\Sigma$ with a space of infinite \textit{bi-infinite} sequences (i.e. sequences of the form $s = \set{ \cdots s_{-n} \cdots s_{-1} * s_0 \cdots s_n \cdots }$).
An one-sided shift is used to study a non-invertible maps, while two-sided shift is used to study an invertible map.
For the two-sided shift, the shift map is also injective.
It is easy to see that the full two-sided shift $(\Sigma, \sigma)$ is also chaotic in Devaney's sense.
\citet{wiggins} has a exposition on the dynamics of the horseshoe map using symbolic dynamics.
The horseshoe map has a geometrically complicated dynamics.
However, we can understand the dynamics of map on its attractor by constructing a conjugacy between the horseshoe map on its attractor and the full two-sided shift.
An one-sided shift can be similarly used to study the dynamics of a mapping.
\citet{sternberg} has an exposition on the study of the attractor of the logistic map.

%%%

\section{Defining Chaos Using Symbolic Dynamics}
\citet{blockcoppel} defines chaos using the one-sided shift, a symbolic dynamics used to encode a non-invertible mapping.
%%%
\begin{definition}
  (Chaos in the sense of Block-Coppel \citep{blockcoppel})
  Let $X$ be a metric space, and $F$ be a continuous map.
  Suppose there exists an invariant compact subset $Y \subseteq X$ and a positive integer $n$ such that $\itr{F}{n}_{|Y}$ is semi-conjugate to the full one-sided shift $(\Sigma, \sigma)$.
  Then, we say that $(X,F)$ is \textit{chaotic in the sense of Block-Coppel}.
  \begin{center}
    \begin{tikzpicture}[node distance=2cm, auto]
      \node (x1) {$Y$};
      \node (x2) [right of=x1] {$Y$};
      \node (y1) [below of=x1] {$\Sigma$};
      \node (y2) [below of=x2] {$\Sigma$};
      \draw[->] (x1) to node {$F$} (x2);
      \draw[->] (x1) to node[swap] {$\phi$} (y1);
      \draw[->] (y1) to node[swap] {$\sigma$} (y2);
      \draw[->] (x2) to node {$\phi$} (y2);
    \end{tikzpicture}
  \end{center}
  \label{defn:blockcoppel}
  \index{definition of chaos!Block-Coppel}
\end{definition}


\bibliographystyle{../../bibliography/pjgsm}
\bibliography{../../bibliography/thesis}

\printindex
\end{document}

