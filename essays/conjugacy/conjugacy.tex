\documentclass[12pt]{book}
\usepackage{../../thesis}

\begin{document}
\section{Conjugacy}
From Sternberg chap4
\subsection{Conjugacy of maps}
These happen to be all conjugate to each other.
\begin{itemize}
  \item Logistic map ($L_\mu(x) = \mu x(1-x)$)
  \item Quadratic map ($Q_c(x) = x^2 + c$)
  \item Tent map 
    \begin{equation*}
      T(x) = 
      \begin{cases}
        2x & 0 \leq x \leq \frac{1}{2}      \\
        2 - 2x & \frac{1}{2} \leq x \leq 1
      \end{cases}
    \end{equation*}
  \item Reverse tent map ($V(x) = 2|x| - 2$)
  \item all quadratic maps ($p(x) = ax^2 + bx + d$)
  \item Sawtooth transformation 
    \begin{equation*}
      S(x) = 
      \begin{cases}
        2x     & 0 \leq x \leq \frac{1}{2}      \\
        2x - 1 & \frac{1}{2} \leq x \leq 1
      \end{cases}
    \end{equation*}
  \item Shift map ($Sh: a_1a_2a_3\ldots \mapsto a_2a_3a_4\ldots$)
  \item Doubling map ()
\end{itemize}

\begin{proposition}
  (Logistic Map and a quadratic map)
  Any quadratic map $f = ax^2 + bx + d$ is conjugate to 
\end{proposition}

\begin{proposition}
  (The doubling map and $Q_{-2}$)
  Let $h: S \rar [-2,2]$
  \begin{equation*}
    h(\theta) = 2\cos\theta.
  \end{equation*}
  $h$ is clearly surjective and continuous.
  The following equality establishes the conjugacy.
  \begin{equation*}
    h(D(\theta)) = 2\cos2\theta = 2(2\cos^2\theta - 1) = (2\cos\theta)^2 - 2 = Q_{-2}(h(\theta)).
  \end{equation*}
\end{proposition}

\begin{proposition}
  (The shift map and the sawtooth transformation)
  First we need to reformulate the sawtooth transformation in terms of the binary representation.
  It turns out to be easy. It's just a multiplication of a sequence of bits by 2, just like any computer scientist knows how to do:
  \begin{equation*}
    S: .a_1a_2a_3\ldots \mapsto .a_2a_3a_4\ldots
  \end{equation*}
  Then, let $h: X \rar [0,1)$ be a trivial transformation from a binary sequence into a binary representation of a number
    \begin{equation*}
      h: a_1a_2a_3\ldots \mapsto .a_1a_2a_3\ldots
    \end{equation*}
    (note the period to indicate fraction).
  Thus conjugacy is established
  \begin{equation*}
    S \circ h = h \circ Sh
  \end{equation*}
  \end{proposition}

  \begin{proposition}
    (The sawtooth transformation and the shift map)
    Let $h = T$, the tent map itself. We have four cases to verify:
    \begin{enumerate}[$(i)$]
      \item $[0,1/4)$
      \item $[1/4, 1/2)$
      \item $[1/2, 3/4)$
      \item $[3/4, 1]$
    \end{enumerate}
    Computation is straightforward but tedious. One can easily verify that
    for each case $T\circ T = T\circ S$ holds.
  \end{proposition}

\bibliographystyle{../../pjgsm}
\bibliography{../../bibliography/thesis}

\end{document}
