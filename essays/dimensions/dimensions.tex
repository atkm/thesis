\documentclass[11pt]{book}
\usepackage{../../thesis}

\begin{document}

\section{Dimensions}

An example where the Hausdorff dimension has an advantage over the box-counting dimension.
Consider the following infinite sequence:
\begin{equation*}
  1, \frac{1}{2}, \frac{1}{3}, \frac{1}{4}, \cdots
\end{equation*}
The sequence, when regarded as a set on the real line, has a non-zero box-counting dimension.
It may be viewed as a dificiency of the box-counting dimension, as one expects that
a set of discrete points is zero-dimensional.
The Hausdorff dimension, however, yields zero for this set.

 \bibliographystyle{../../pjgsm}
\bibliography{../../bibliography/thesis}


\nocite{*}
\end{document}

