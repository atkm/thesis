\documentclass[11pt]{book}
\usepackage{thesis}

\begin{document}

\section{A list of chaotic maps}
\begin{itemize}
  \item Logistic map ($L_\mu(x) = \mu x(1-x)$)
  \item Quadratic map ($Q_c(x) = x^2 + c$)
  \item Tent map 
    \begin{equation*}
      T(x) = 
      \begin{cases}
        2x & 0 \leq x \leq \frac{1}{2}      \\
        2 - 2x & \frac{1}{2} \leq x \leq 1
      \end{cases}
    \end{equation*}
  \item Reverse tent map ($V(x) = 2|x| - 2$)
  \item all quadratic maps ($p(x) = ax^2 + bx + d$)
  \item Sawtooth transformation (Baker's transformation)
    \begin{equation*}
      S(x) = 
      \begin{cases}
        2x     & 0 \leq x \leq \frac{1}{2}      \\
        2x - 1 & \frac{1}{2} \leq x \leq 1      \\
        (0 & x = 1 \mbox{ in Martelli.})
      \end{cases}
    \end{equation*}
  \item Shift map ($Sh: a_1a_2a_3\ldots \mapsto a_2a_3a_4\ldots$)
  \item Doubling map ($D: \C \to \C$. $C: z \mapsto z^2$)
  \item $F: I [-1,1] \to I$; $F: x \mapsto 2|x| - 1$ (period 3) (Martelli p95)
  \item $F: I [0,1] \to I$; $F: x \mapsto 2x - \ceil{2x}$ (period 3, but not continuous) (Martelli p95)
  \item Henon map
    \begin{align*}
      x_{n+1} &= y_n + 1 - ax_n^2 \\
      y_{n+1} &= bx_n
    \end{align*}
  \item Kaplan-Yorke map
    \begin{align*}
      x_{n+1} &= 2x_n \mod 1    \\
      y_{n+1} &= \alpha y_n + \cos(4\pi x_n)
    \end{align*}
  \item Arnold's cat map
  \item (Martelli p215; Not chaotic in the sense of Devaney nor Marotto (generalized Li-Yorke chaos), 
    but meets Martelli's definition)
    \begin{equation*}
      T(\rho, \theta) =
      \begin{cases}
        (2\rho, \theta + 1) \quad 0\leq \rho < 1/2 \\
        (2 - 2\rho, \theta + 1) \quad 1/2 \leq \rho \leq 1
      \end{cases}
    \end{equation*}

\end{itemize}

\section{A list of chaotic (differential) equations}
\begin{itemize}
  \item Lorenz Model of Atmospheric Behavior
    \begin{equation*}
     \begin{cases}
      \dot{x} = -ax + ay \\
      \dot{y} = -xz + rx -y \\
      \dot{z} = xy - bz.
    \end{cases}
    \end{equation*}
    \end{itemize}
\end{document}
