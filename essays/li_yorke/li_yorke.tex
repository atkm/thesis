\documentclass[10pt,draft,twoside]{book}
\usepackage{../../thesis}

\makeindex
\begin{document}

\chapter{Chaos in Li-Yorke sense}
\label{chap:liyorke}

\section{Period Three Implies Chaos}
This section is an exposition of a seminal paper, ``Period Three Implies Chaos'' by \citet{li-yorke}.
As it has been mentioned in the introduction, Li and Yorke were the first to use the term ``chaos'' to describe complex behaviors seen in maps such as those of Lorentz, Henon, and Ueda.
Li-Yorke's definition inspired many other definitions of chaos, such as $\omega$-chaos~\citep{omegachaos}, distributional chaos~\citep{dchaos1}, and others \citep{genericchaos,densechaos,extremechaos}, which incorporate analogous conditions (equations \ref{eqn:liyorke1}, \ref{eqn:liyorke2}).
At the end of this chapter, we introduce a definition by Marotto as an example of a definition of chaos influenced by Li-Yorke's.
As we will see, determining whether a map is chaotic in Li-Yorke's sense for $\R^n$ for $n \geq 2$ using the definition is not a simple task.
Chaos in Marotto's sense, defined for $\R^n$, implies chaos in Li-Yorke's sense, and therefore, Marotto's definition may serve as a criterion for chaos in Li-Yorke's sense that is easier to verify, though it requires an additional structure, namely differentiability.

% F: I \to I a continuous map on a compact interval.
% Then, F is chaotic in the sense of devaney if and only if positive topological entropy.
% Furthermore, F has positive topological entropy if and only if omega-chaotic \citep{omegachaos}.

\begin{definition}
  (Chaos in Li-Yorke's sense)
  Let $X$ be a compact metric space and $F: X\to X$.
  $F$ is said to be \textit{Li-Yorke chaotic}, or 
  \textit{chaotic in Li-Yorke's sense} if there is an uncountable set $S \subseteq X$ containing no
  periodic points such that each pair $p,q \in S$ satisfies
  \begin{equation}
    \limsup\limits_{n \to \infty}\; \metric{\itr{F}{n}(p) , \itr{F}{n}(q)} > 0
    \label{eqn:liyorke1}
  \end{equation}
  and
  \begin{equation}
    \liminf\limits_{n \to \infty}\; \metric{\itr{F}{n}(p) , \itr{F}{n}(q)} = 0.
    \label{eqn:liyorke2}
  \end{equation}
  \index{chaos!definition of!Li-Yorke}
\end{definition}
%%%
Equation~\eqref{eqn:liyorke2} says that the orbits of two points come arbitrarily close to each other infinitely many times.
However, as Equation~\eqref{eqn:liyorke1} states, the same two orbits get separated by some positive constant infinitely often.
Thus, even if two initial points $p$ and $q$ has come very close to each other at some time $t$, after some time, say $t + t_0$, the two points can be far apart, and vice versa.
Li and Yorke defined chaos as the existence of \textit{uncountably many} pairs of such points.
At an intuitive level, we can expect that the uncountable set would cause complex behaviors of a system.

\citet{li-yorke} proved that any continuous map with a 3-periodic orbit is chaotic in their sense.

\begin{theorem}
  (Period three implies all periods \citep{li-yorke})
  Let $I$ be an interval and $F: I\to I$ be continuous. Suppose there is a point $a \in I$ for which
  the points
  \begin{equation*}
  b = F(a), \quad c = F(b), \quad d = F(c)
  \end{equation*}
  satisfy
  \begin{equation*}
    d \leq a < b < c \quad(\mbox{or } d \geq a > b > c).
  \end{equation*}
  Then $F$ has a periodic orbit of every period.
  Note that the hypothesis is satisfied if $F$ has a periodic orbit of period 3.
  \label{thm:liyorke1}
\end{theorem}
%%%
\begin{theorem}
  (Period three implies Li-Yorke chaos \citep{li-yorke})
  Suppose the same hypothesis as Theorem~\ref{thm:liyorke1}.
  Then $F$ is chaotic in Li-Yorke's sense.
  \label{thm:liyorke2}
\end{theorem}
%%%
\begin{example}
  To illustrate Li-Yorke's definition and their theorem, we show that both the logistic map and tent map are Li-Yorke chaotic by finding 3-periodic points.
  \citet{saha} proved that if $\mu \geq 1 + 2\sqrt{2}$, then $L_\mu: x \mapsto \mu x(1-x)$ has a 3-periodic orbit.
  Hence, $L_\mu$ is chaotic for $\mu \geq 1 + 2\sqrt{2}$.
  Next, the tent map is defined as
  \begin{equation*}
    T(x) = 
    \begin{cases}
      2x     &\mbox{ if } x \in [0,1/2] \\
      2 - 2x &\mbox{ if } x \in [1/2,1].
    \end{cases}
  \end{equation*}

\begin{figure}[th]
  \centering
  \begin{tikzpicture}[scale=3.2]
    \draw[->] (-0.2,0) -- (1.2,0) node[right] {$x$};
    \draw[->] (0,-0.2) -- (0,1.2) node[above] {$T(x)$};
    \foreach \x/\xtext in {0.5/\frac{1}{2}, 1/1}
    \draw[shift={(\x,0)}] (0pt,1pt) -- (0pt,-1pt) node[below] {$\xtext$};
    \foreach \y/\ytext in {0.5/\frac{1}{2}, 1/1}
    \draw[shift={(0,\y)}] (1pt,0pt) -- (-1pt,0pt) node[left] {$\ytext$};

    \draw[domain=0:0.5] plot (\x,{2*\x}) node[below right] {};
    \draw[domain=0.5:1] plot (\x,{2 - 2*\x}) node[below right] {};
  \end{tikzpicture}
  \caption{The tent map, $T(x)$.}
\end{figure}

  It is easy to see that $2/7$ and $2/9$ are 3-periodic points.
  This result could have been gleaned from the fact that the logistic map is conjugate to the tent map.
  We will present the conjugacy in Chapter~\ref{chap:entropy}.
\end{example}

One may be surprised that such an innocent condition gives rise to a complex behaviour.
\citet{sarkovskii}, a Soviet mathematician, proved a theorem stronger than Theorem~\ref{thm:liyorke1} ten years before the publication of ``Period Three Implies Chaos.''
Sarkovskii's result extends Li-Yorke's Theorem~\ref{thm:liyorke1} by proving analogous results for period 5, 7, 9, \ldots, and so on.
To be more specific, Sarkovskii's theorem states that there exists an ordering of $\mathbb{N}$---called \textit{Sarkovskii's ordering}---such that if a mapping has a $k$-periodic orbit, then it has periodic orbits of period $n$ for \textit{each} integer $n$ that come after $k$ in the ordering.
Three comes the first in Sarkovskii's ordering, and in this sense, the number three is at the top of the Sarkovskii's hierarchy of the natural numbers.
(See the Appendix for Sarkovskii's ordering.)

Now we proceed to prove Theorem~\ref{thm:liyorke1}. 
(The proof for Theorem~\ref{thm:liyorke2} uses ideas similar to those in Theorem~\ref{thm:liyorke1}.
However, the proof is longer and tedious.
For those who are interested in the proof, it is contained in the appendix.
)
We start with proving three results that we require for proving the theorems.

\begin{proposition}
  Let $I$ be an interval, $F: I \to \R$ a continuous map.
  For each compact interval $I_1 \subseteq F(I)$, there exists a compact interval $Q \subseteq I$ such that $G(Q) = I_1$.
  \label{prop:liyorke1}
  \begin{proof}
    Suppose $I_1 = [F(p),F(q)]$ for some $p,q \in I$.
    Without loss of generality, we may assume $p < q$, for otherwise, we merely switch the roles of $p$ and $q$ in the proof.
    Let $Q = [r,s]$, where $r$ is the greatest point in $[p,q]$ such that $F(r) = F(p)$, and $s$ is the least point in $(r,q]$ such that $F(s) = F(q)$.
    Since $F$ is continuous, $F(Q) = I_1$.
  \end{proof}
\end{proposition}

\begin{proposition}
  Let $I$ be an interval, and $F: I\to I$ be a continuous map.
  Suppose that for each $n$, $\seq{I_n}{\infty}{n=0}$ is a sequence of compact intervals with $I_n \subseteq I$ and $I_{n+1} \subseteq F(I_n)$.
  Then, there exists a sequence $\seq{Q_n}{\infty}{n=0}$ of compact intervals such that 
\begin{equation*}
    Q_{n+1} \subseteq Q_{n} \subseteq I_0,
  \end{equation*}
and, for each $n$,
  \begin{equation*}
    \itr{F}{n}(Q_n) = I_n.
  \end{equation*}
  Furthermore, for any $x \in Q \equiv \bigcap\limits_{n=0}^{\infty}Q_n$, we have for each $n$
  \begin{equation*}
    \itr{F}{n}(x) \in I_n.
  \end{equation*}
  \label{prop:liyorke2}
  %%%
  \begin{proof}
    Proof by induction.
    Let $Q_0 = I_0$.
    Clearly, we have $\itr{F}{0}(Q_0) = I_0$.
    For each $n$, define $Q_{n-1}$ as a compact interval such that $\itr{F}{n-1}(Q_{n-1}) = I_{n-1}$.
    Note that $I_n \subseteq F(I_{n-1}) = \itr{F}{n}(Q_{n-1})$.
    Proposition~\ref{prop:liyorke1} applied to $\itr{F}{n}$ on $Q_{n-1}$ proves the existence of a compact interval $Q_n \subseteq Q_{n-1}$ such that $\itr{F}{n}(Q_n) = I_n$.
  \end{proof}
\end{proposition}
%%%
\begin{proposition}
  Let $I$ be an interval, and $F: I\to \R$ be a continuous map.
  Suppose $J \subseteq I$ is a compact interval and $J \subseteq F(J)$. 
  Then, there exists a point $p \in I$ such that $G(p) = p$.
  \begin{proof}
    This is a special case of the contraction fixed point theorem proved in Preliminaries.
  \end{proof}
  \label{prop:liyorke3}
\end{proposition}
Proposition~\ref{prop:liyorke2} will be especially useful in the proofs.
For brevity, we call $[a,b]$ and $[b,c]$ as $K$ and $L$, respectively.
The key observation to make is that we have $K,L \subseteq F(L)$ by $F(b) = c$, $F(c) = d$, and continuity of $F$.
Then, we can design an orbit that stays in $L$ as long as we wish.
%%%
\begin{proof}[Proof of Theorem~\ref{thm:liyorke1}.]
  We will prove the theorem for $d \leq a < b < c$.
  Define two intervals $K,L$ as follows
  \begin{equation*}
    K = [a,b] \mbox{ and } L = [b,c].
  \end{equation*}
  Let $k$ be a positive integer.
  For each $k > 1$, let $\set{I_n}$ be the sequence of intervals $I_n = L$ for $n = 0, \ldots, k-2$, and $I_{k-1} = K$.
  Furthermore, define $I_n$ for $n \geq k$ by $I_{n+k} = I_n$.
  We associate this sequence with an orbit of some interval $J$ such that
  \begin{equation*}
    \itr{F}{n}(J) \subseteq I_n.
  \end{equation*}
  Note that we have 
  \begin{equation*}
    K,L \subseteq F(L) \mbox{ and } L \subseteq F(K).
  \end{equation*}
  Then, by Proposition~\ref{prop:liyorke2}, there exists a sequence of compact intervals $\set{Q_n}$ such that $Q_{n+1} \subseteq Q_{n}$; in particular $Q_k \subseteq Q_0$.
  Also by Proposition~\ref{prop:liyorke2}, we have $\itr{F}{k}(Q_k) = I_0 = Q_0$.
  Since $\itr{F}{k}$ is continuous and $Q_k \subseteq \itr{F}{k}(Q_k)$, we may apply Proposition~\ref{prop:liyorke3} to conclude that $\itr{F}{k}$ has a fixed point $p_k \in Q_k$.
  We claim that $p_k$ does not have period less than $k$.
  To see this, assume the contrary and let the period of $p_k$ be $t < k$.
  Since $p_k \in Q_n \subseteq Q_{n-1}$, we have 
  \begin{equation*}
    \itr{F}{k-1}(p_k) \in I_{k-1} = K.
  \end{equation*}
  Then, since $p_k$ is $t$-periodic, we need 
  \begin{equation*}
    \itr{F}{k-1-t}(p_k) \in K.
  \end{equation*}
  In addition, we have
  \begin{equation*}
    \itr{F}{k-1-t}(p_k) \in I_{k-1-t} = L.
  \end{equation*}
  Hence, $\itr{F}{k-1-t}(p_k) \in L \cap K = \set{b}$.
  It follows that $\itr{F}{k-1}(p_k) = b$, and therefore, $\itr{F}{k+1}(p_k) = a$.
  This is a contradiction, because by construction, $\itr{F}{k+1}(p_k)$ must be in $L$.
  Thus, $p_k$ is a $k$-periodic point for $F$.
  The other case, $d \leq a < b < c$, can be proved in the same manner.
\end{proof}
%%%
We already know that points in $S$ are not periodic.
However, we may draw an even stronger conclusion, that is, the points in $S$ are not even asymptotically periodic.
\begin{theorem}
  (Points in $S$ are not asymptotically periodic \citep{li-yorke})
  For every $p \in S$ and periodic point $q \in I$,
  \begin{equation*}
    \limsup\limits_{n \to \infty}\; \metric{\itr{F}{n}(p) , \itr{F}{n}(q)} > 0.
  \end{equation*}
  \label{thm:liyorke3}
  \begin{proof}
    The proof is similar to the 'limsup' part of Theorem~\ref{thm:liyorke2}.
    See \citet{li-yorke}.
  \end{proof}
\end{theorem}
%%%
Next, we check that chaos in Li-Yorke's sense is preserved under conjugacy.
\begin{theorem}
  (Li-Yorke's chaos is a topological invariant)
  Let $X$ be a metric space, and $F: X\to X$ be chaotic in Li-Yorke's sense.
  If $G: Y\to Y$ is conjugate to $F$ i.e. $\phi \circ F = G \circ \phi$, where $\phi: X \to Y$ is a homeomorphism, then $G$ is chaotic.
  \begin{proof}
    Let $S \subseteq X$ be an uncountable set given by definition.
    $S$ contains no periodic points, and each pair $p,q \in S$ satisfies
    \begin{equation*}
      \limsup\limits_{n \to \infty}\; \metric{\itr{F}{n}(p) , \itr{F}{n}(q)} > 0
    \end{equation*}
    and
    \begin{equation*}
      \liminf\limits_{n \to \infty}\; \metric{\itr{F}{n}(p) , \itr{F}{n}(q)} = 0.
    \end{equation*}
    Let $T \ceq \phi(S) \subseteq Y$.
    Let $b \ceq \limsup\limits_{n \to \infty}\; \metric{\itr{F}{n}(p) , \itr{F}{n}(q)} > 0$.
    Since $\phi$ is injective, $T$ is uncountable, and contains no periodic points (see Theorem~(uncomment!)%\ref{thm:conj-per}).
    We first prove
    \begin{equation*}
      \liminf\limits_{n \to \infty}\; \metric{\itr{G}{n}(\phi(p)) , \itr{G}{n}(\phi(q))} = 0.
    \end{equation*}
    For convenience, let us use the following notation:
    \begin{equation*}
      d_{G_n} \ceq \metric{\itr{G}{n}(\phi(p)) , \itr{G}{n}(\phi(q))}.
    \end{equation*}

    Fix $p,q \in S$.
    Since we have $\itr{F}{n} = \phi^{-1}\circ\itr{G}{n}\circ\phi$ by conjugacy,
    \begin{align*}
      &\liminf\limits_{n \to \infty}\; \metric{\itr{F}{n}(p) , \itr{F}{n}(q)} \\
      &= \liminf\limits_{n \to \infty}\; \metric{(\phi^{-1}\circ\itr{G}{n})(\phi(p)) , (\phi^{-1}\circ\itr{G}{n})(\phi(q))} \\
      &= 0.
    \end{align*}
  By continuity of $\phi$, for each $\delta > 0$ there exists $\epsilon>0$ such that 
  \begin{equation*}
    y \in \oball{\epsilon}{(\phi^{-1}\circ G^n)(\phi(p))} \mbox{ implies } \phi(y) \in \oball{\delta}{G^n(\phi(p))}.
  \end{equation*}
  Take $(\phi^{-1}\circ G^n)(\phi(q))$ as our $y$.
  For each $\delta$, by definition of limit inferior, as $n$ increases from $0$ to $\infty$, $(\phi^{-1}\circ G^n)(\phi(q))$ comes inside the neighborhood $\oball{\epsilon}{(\phi^{-1}\circ G^n)(\phi(p))}$ infinitely many times.
  Therefore, for each $\delta$, $\phi(y) \in \oball{\delta}{G^n(\phi(p))}$ holds for infinitely many $n$.
  We have $d_{G_n} \geq 0$, so we conclude that
  \begin{equation*}
    \liminf\limits_{n \to \infty}\; d_{G_n} = 0.
  \end{equation*}
  Next, we prove
  \begin{equation*}
    \limsup\limits_{n \to \infty}\; d_{G_n} > 0
  \end{equation*}
  by contradiction.
  So suppose
  \begin{equation*}
    \limsup\limits_{n \to \infty}\; d_{G_n} = 0.
  \end{equation*}
  Since we already have
  \begin{equation*}
    \liminf\limits_{n \to \infty}\; d_{G_n} = 0,
  \end{equation*}
  it follows that 
  \begin{equation*}
    \lim\limits_{n \to \infty}\; d_{G_n} = 0.
  \end{equation*}
  By continuity of $\phi^{-1}$, for each $\delta > 0$, there exists $\epsilon > 0$ such that
  \begin{equation*}
    y \in \oball{\epsilon}{G^n(\phi(p))} \mbox{ implies } \phi^{-1}(y) \in \oball{\delta}{(\phi^{-1} \circ G^n)(\phi(p))}.
  \end{equation*}
  Take $G^n(\phi(q))$ as our $y$.
  Note that, for each $\epsilon$, there are infinitely many $n$ such that $G^n(\phi(q)) \in \oball{\epsilon}{G^n(\phi(p))}$, since 
  \begin{equation*}
    \lim\limits_{n \to \infty}\; d_{G_n} = 0.
  \end{equation*}
  Then, for each $\delta$, there exists $N$ such that for each $n \geq N$,
  \begin{equation*}
    (\phi^{-1} \circ G^n)(\phi(p)) \in \oball{\delta}{(\phi^{-1} \circ G^n)(\phi(p))}.
  \end{equation*}
  In particular, we can take $0 < \delta_0 < b$ so that
  \begin{equation*}
    \limsup\limits_{n \to \infty}\; \metric{(\phi^{-1}\circ\itr{G}{n})(\phi(p)) , (\phi^{-1}\circ\itr{G}{n})(\phi(q))}
  \end{equation*}
  is less than $b$.
  This contradicts with the assumption that
  \begin{equation*}
    \limsup\limits_{n \to \infty}\; \metric{(\phi^{-1}\circ\itr{G}{n})(\phi(p)) , (\phi^{-1}\circ\itr{G}{n})(\phi(q))}
    = \limsup\limits_{n \to \infty}\; \metric{\itr{F}{n}(p) , \itr{F}{n}(q)} 
    = b.
    \end{equation*}
  This completes the proof.
  \end{proof}
  \label{thm:liyorke-conj}
\end{theorem}
%%%
Although the definition by Li and Yorke captures our intuitive notion of chaos well, it is by no means flawless.
Most notably, Li-Yorke's theorem (Theorem~\ref{thm:liyorke1}) applies only to one-dimensional maps.
Also, there are mappings that are chaotic on a negligible set \citep{martelli}.
\begin{example}
  (Chaos on a negligible set)
  Let $I = [0,1]$ and $F: I \to I$ defined as
  \begin{equation*}
    f(x) =
    \begin{cases}
      0 \quad & 0\leq x < \frac{1}{4} \\
      4x - 1 \quad  & \frac{1}{4} \leq x < \frac{1}{2} \\
      -4x + 3 \quad & \frac{1}{2} \leq x < \frac{3}{4} \\
      0 \quad & \frac{3}{4} \leq x \leq 1. \\
    \end{cases}
  \end{equation*}
  Clearly, $F$ is continuous.
  Also, $x = 23/65$ is a 3-periodic point:
  \begin{equation*}
    \itr{F}{3}\paren{\frac{23}{65}} = \itr{F}{2}\paren{\frac{27}{65}} = F\paren{\frac{43}{65}} = \frac{23}{65}.
  \end{equation*}
Hence, by Theorem~\ref{thm:liyorke2}, $F$ is chaotic in Li-Yorke's sense.
%
Next, we will show that, for almost every $x_0 \in I$ (with respect to the Lebesgue measure), the orbit $\set{\itr{F}{n}(x_0)}$ converges to 0.
First, note that 
\begin{equation*}
  F^{-1}\paren{\left(\frac{2}{3},1\right]} 
  = \paren{\frac{5}{12},\frac{7}{12}}.
\end{equation*}
Let $I_0 = (5/12, 7/12)$.
Then, inductively define $I_n (n\geq 1)$ as 
\begin{equation*}
  I_n = F^{-1}(I_{n-1}).
\end{equation*}
%
For convenience, we denote the length of an interval as $m(I)$.
We claim that
\begin{enumerate}[(1)]
  \item For $q\neq m$, $I_q \cap I_m = \emptyset$.
  \item $m(I_n) = 1/(6 \cdot 2^n)$
\end{enumerate}
%
We prove claim (1) by contradiction.
For any $q,m$, assume there exists $x \in I_q \cap I_m$. 
Without loss of generality, we may assume that $q > m$.
By the definition of $I_m$, we have $F^m(x) \in I_0$.
It follows that $F^{m+1}(x) \in F(I_0) = (2/3, 1]$, and $F^{m+t}(x) \in [0, 1/3)$ for $t \geq 2$.
However, this is a contradiction, because we must also have $F^q(x) \in I_0$.
%
To prove (2), first note that $F$ restricted to $(1/3, 2/3)$ is a piecewise affine transformation.
Consider $g_1(x) = 4x - 1$ and $g_2(x) = -4x + 3$ restricted to $(1/3,1/2]$ and $[1/2,2/3)$, respectively.
The inverse functions of $g_1$ and $g_2$ exist, and they are
\begin{equation*}
  g_1^{-1}(x) = \frac{1+x}{4}; \quad g_2^{-1} = \frac{3-x}{4}.
\end{equation*}
The images of $g_1^{-1}$ and $g_2^{-1}$ are $(1/3,2/3)$, and the inverse functions are continuous.
It follows that, if we suppose $J$ is an interval in $(1/3,2/3)$, then $J$ is mapped to $(1/3,1/2]$ by $g_1^{-1}$ and to $[1/2,2/3)$ by $g_2^{-1}$, respectively.
Then, $m(g_1^{-1}(J))$ and $m(g_2^{-1}(J))$ are disjoint (except for the point $x = 1/2$), and also
\begin{equation*}
  4 \cdot m(g_1^{-1}(J)) = 4 \cdot m(g_2^{-1}(J)) = m(J).
\end{equation*}
Since $m(I_0) = 1/6$, we have
\begin{equation*}
  m(I_1) 
  = m(g_1^{-1}(I_0)) + m(g_2^{-1}(I_0))
  = 2 * \frac{1}{6} * \frac{1}{4}
  = \frac{1}{6\cdot 2^1},
\end{equation*}
and inductively, we obtain
\begin{equation*}
  m(I_n) = \frac{1}{6 \cdot 2^n}.
\end{equation*}
%
Finally, by (1) and (2) we have
\begin{equation*}
  \sum\limits_{n\geq 0} m(I_n)
  = \frac{1}{3} \cdot \sum\limits_{n\geq 0} \frac{1}{2^{n+1}}
  = \frac{1}{3}.
\end{equation*}
Thus, almost every point in $I$ converges to 0 under iteration by $F$.
\end{example}


\section{Chaos in Marotto's sense}
\citet{marotto1} has a definition of chaos, which implies chaos in Li-Yorke's sense, using the notion of \textit{snap-back repeller}, which requires the mapping to be differentiable.
The original definition had a minor technical flaw \citep{shi}, and Marotto later corrects his definition \citep{marotto2}.
We present Marotto's definition in $\R^n$.
%%%
\begin{definition}
  (Repelling fixed point)
  Let $F: \R^n \to \R^n$ be a continuous mapping. 
  A fixed point $z$ is called a \textit{repelling fixed point} if the magnitudes of all eigenvalues of the Jacobian matrix of $F$ at $z$ are strictly greater than $1$.
  \index{repelling fixed point}
\end{definition}
(A repelling fixed point is sometimes referred to as a source.)

\begin{definition}
  (Expanding fixed point)
  Let $F: \R^n \to \R^n$ be a continuous mapping, and $d$ be a metric induced by some norm.
  A fixed point $z$ is said to be \textit{expanding with respect to the norm} if there exists a neighborhood $N$ of $z$ such that, for each $y \in N$, there exists $s > 1$ such that 
  \begin{equation*}
    \metric{F(z), F(y)} > s \cdot \metric{z,y}.
  \end{equation*}
  \index{expanding fixed point}
\end{definition}
%%%
Next, we show that a fixed point is repelling if and only if it is expanding with respect to some norm.
This equivalence is not true in general.
In particular, \citet{shi} point out that, when using the Euclidean norm, a repelling fixed point is not necessarily expanding.

\begin{proposition}
  Suppose $z$ is a repelling fixed point of $F$.
  Then, there exists a norm defined on $\R^n$ and a neighborhood $N$ of $z$ such that $z$ is an expanding fixed point in $N$ with respect to this norm.
  \begin{proof}
    See \citet[p. 278-281]{hirsch}.
  \end{proof}
  \label{prop:repelling-expanding}
\end{proposition}

\begin{definition}
  (Snapback repeller)
  Let $F: X\to X$ be a $\mathcal{C}^1$-differentiable function, where $X$ is a compact subset of $\R^n$.
  Suppose $z \in X$ is a repelling fixed point of $F$.
  By Proposition~\ref{prop:repelling-expanding}, $z$ is an expanding fixed point in some neighborhood $N$ with respect to some norm.
  $z$ is called a \textit{snap-back repeller} of $F$, if there exist a point $y \in N$ $(y \neq z)$ and a positive integer $n$ such that 
  \begin{equation*}
   \itr{F}{n}(y) = z \quad\mbox{ and }\quad \det(J\itr{F}{n}(y)) \neq 0,
  \end{equation*}
  where $J\itr{F}{n}$ is the Jacobian matrix of $\itr{F}{k}$.
  \index{chaos!definition of!Marotto}
\end{definition}
%%%
The terminology comes from the fact that $z$ is a fixed point such that a point in its neighborhood gets repelled away, yet the point eventually ``snaps back'' to $z$.

\begin{theorem}
  \citep{marotto1,marotto2}
  Let $F: \R^n \to \R^n$ be differentiable.
  If there exists a snap-back repeller of $F$, then $F$ is chaotic in Li-Yorke's sense.
  \label{thm:sbrepeller}
  \begin{proof}
    See \citet[Theorem 3.1]{marotto1}.
    \citet{martellibook} discusses concepts employed in the proof.
  \end{proof}
\end{theorem}
The ease of application is perhaps the primary advantage of using Marotto's definition.
This definition has been applied to the Henon map \citep{marotto1} a certain wave equation \citep{marottoapp1}, a model in population biology \citep{marottoapp2}, and neuroscience \citep{marottoapp3}.
A longer list of examples of applications of Marotto's definition is in \citet{marotto2}.

%%% Proof of the theorem.
%\begin{proposition}
%  (Martelli p115, 128)
%  Let $M$ be a square matrix whose eigenvalues have moduli larger than 1.
%  Then 
%  \begin{enumerate}[(i)]
%    \item $M$ is invertible.
%    \item $\rho(M^{-1}) < 1$, where $\rho(M^{-1})$ denotes the spectral radius of $M^{-1}$.
%    \item There exists a norm $\norm{.}_a$ such that $\norm{M}_a < 1$.
%  \end{enumerate}
%\end{proposition}
%\begin{proof}
%  (i) Since $0$ is not an eigenvalue of $M$, we have $\det M \neq 0$.
%  Hence $M$ is invertible.
%
%  (ii) The result follows naturally from the fact that
%  $\lambda$ is an eigenvalue of $M$ if and only if $\lambda^{-1}$ is an eigenvalue of $M^{1-}$.
%
%  (iii) 
%\end{proof}
%\begin{proposition}
%  (Repeller implies expanding fixed point \citep[p. 188]{martellibook})
%  Let $X \subset \R^n$ be open, and let $x_s$ be a repelling fixed point
%  of a continuous function $F: X \to X$. Suppose that $F$ is
%  differentiable in some neighborhood of $x_s$, and the derivative of $F$ at $x_s$ is continuous.
%  Then $x_s$ is an expanding fixed point.
%\end{proposition}
%\begin{proof}
%  $J$ is invertible because it satisfies the conditions of the preceding proposition.
%  Also, we can find a norm $\norm{.}_a$ such that $\norm{M^{-1}}_a < 1$.
%  For each $x, y \in \R^n$, we have 
%  \begin{equation*}
%    \norm{x-y}_a = \norm{M^{-1} M x - M^{-1} M y}_a \leq \norm{M^{-1}}_a \norm{M(x - y)}_a.
%  \end{equation*}
%  It follows that 
%  \begin{equation*}
%    \frac{\norm{M(x-y)}_a}{\norm{x-y}_a} \geq \frac{1}{\norm{M^{-1}}_a}.
%  \end{equation*}
%  For convenience, let $K = 1/\norm{M^{-1}}_a$.
%  Note that $K > 1$.
%  (fill in the blank)
%  \begin{equation*}
%    \frac{\norm{F(x) - F(x_s)}_a}{\norm{x - x_s}_a} \geq \frac{K+1}{2}.
%  \end{equation*}
%  Let $k \equiv \frac{K+1}{2}$.
%  (Proposition begin) Now suppose
%  \begin{equation*}
%    0 < \norm{x_0 - x_s}_a \leq r_1.
%  \end{equation*}
%  ($r_1$ just needs to be small enough that an open ball
%  of radius $r_1$ around $x_s$ does not distend over the
%  neighborhood where $F$ is differentiable.) Then
%  \begin{align*}
%    \norm{F^n(x_0) - x_s}_a &= \norm{F(F^{n-1}) - x_s}_a \geq k \norm{F^{n-1} - x_s}_a     \\
%    &= \norm{F(F^{n-2}) - x_s}_a \geq \cdots \\
%    &\geq k^n \norm{x_0 - x_s}_a.
%  \end{align*}
%  (Proposition end)
%  Since $k > 1$, we can find $n$ that satisfies the following inequality
%  \begin{equation*}
%    \norm{F^n(x_0) - x_s}_a > r_1.
%  \end{equation*}
%  By the equivalence of norms defined in $\R^n$, there exist $0 < b < c$ such that 
%  \begin{equation*}
%    b \norm{x_0 - x_s} \leq \norm{x_0 - x_s}_a \leq c \norm{x_0 - x_s}_a.
%  \end{equation*}
%  Finally, set $r \equiv r_1/c$. If we suppose that 
%  \begin{equation*}
%    \norm{x_0 - x_s} \leq r,
%  \end{equation*}
%  then
%  \begin{equation*}
%    \norm{x_0 - x_s}_a \leq c \norm{x_0 - x_s} \leq cr = r_1.
%  \end{equation*}
%  It follows by the proposition that there exists $m > 0$ such that 
%  \begin{equation*}
%    c \norm{F^m(x_0) - x_s} \geq \norm{F^m(x_0) - x_s}_a > r_1,
%  \end{equation*}
%  which implies
%  \begin{equation*}
%    \norm{F^{m}(x_0) - x_s} > \frac{r_1}{c} = r.
%  \end{equation*}
%\end{proof}

% Let $\lambda$ be an eigenvalue of $M$. Note that
%   \begin{equation*}
%     (\lambda^{-1} I - M^{-1}) M = \lambda^{1-} M - I = \lambda^{-1} (M - \lambda I)
%   \end{equation*}
%   implies $\det ( (\lambda^{-1}I - M^{-1}) M) = 0 $ since $\det (M - \lambda I) = 0$.
%   Also, it follows from the following equation
%   \begin{equation*}
%     \det( (\lambda^{-1} I - M^{-1} ) M) = \det( (\lambda^{-1} I - M^{-1} )) \det M
%   \end{equation*}
%   that $det(\lambda^{-1} I - M^{-1}) = 0$.
%   This establishes the fact that $\lambda^{-1}$ is an eigenvalue of $M^{-1}$ if and only if $\lambda$ is an eigenvalue of $M$.
% 

\bibliographystyle{../../bibliography/pjgsm}
\bibliography{../../bibliography/thesis}
\printindex
\end{document}
