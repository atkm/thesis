\documentclass[12pt,draft,twoside]{book}
\usepackage{../../thesis}

\makeindex
\begin{document}

\chapter{Chaos in Li-Yorke sense}

\section{Period Three Implied Chaos}
This section follows the treatment in ``Period Three Implies Chaos'' \citep{li-yorke}, a seminal paper in the early era of chaos theory.
Li and Yorke were the first to use the term ``chaos'' to describe the complex behaviors seen in maps such as those of Lorentz, Henon, and Ueda.
Robert May, a population biologist who noticed chaotic behavior of the logistic map, popularized the term in his articles \citeyearpar{may1,may2}.
\footnote{Although ``chaos'' is commonly used in academic papers nowadays, the title of Li and Yorke's paper was considered to be playful and inappropriate.
Admittedly, a liberal choice of word has to some extent contributed to the popularity that chaos theory currently enjoys \citep[``Exploring Chaos on an Interval'']{ueda-abraham}.}
This definition inspired many other definitions of chaos.
dense chaos~\citep{densechaos}, generic chaos~\citep{genericchaos}, extreme chaos~\citep{extremechaos}, distributional chaos, and so on. 


\begin{definition}
  (Chaos in Li-Yorke's sense)
  Let $X$ be a compact metric space and $F: X\to X$.
  $F$ is said to be \textit{Li-Yorke chaotic}, or 
  \textit{chaotic in Li-Yorke's sense} if there is an uncountable set $S \subseteq X$ containing no
  periodic points such that each $p,q \in S$ satisfy
  \begin{equation}
    \limsup\limits_{n \to \infty}\; \abs{\itr{F}{n}(p) - \itr{F}{n}(q)} > 0
    \label{eqn:liyorke1}
  \end{equation}
  and
  \begin{equation}
    \liminf\limits_{n \to \infty}\; \abs{\itr{F}{n}(p) - \itr{F}{n}(q)} = 0.
    \label{eqn:liyorke2}
  \end{equation}
  \index{chaos!definition of!Li-Yorke}
\end{definition}

The equation~\eqref{eqn:liyorke2} says that the orbits of two points come arbitrarily close to each other infinitely many times.
Nevertheless, as equation~\eqref{eqn:liyorke1} states, the same two orbits get separated by some positive constant infinitely often.
Thus, even if two initial points $p$ and $q$ has come very close to each other at some time $t$, after some time, say $t + t_0$, the two points can be far apart.
Li and Yorke defined chaos as the existence of \textit{uncountably many} pairs of such points.
Intuitively, we can expect that the uncountable set would behave in a complex manner under iteration.

\citet{li-yorke} proved that any continuous map with a 3-periodic orbit is chaotic in their sense.

\begin{theorem}
  (Period three implies all periods)
  Let $I$ be an interval and $F: I\to I$ be continuous. Suppose there is a point $a \in I$ for which
  the points
  \begin{equation*}
  b = F(a), \quad c = F(b), \quad d = F(c)
  \end{equation*}
  satisfy
  \begin{equation*}
    d \leq a < b < c \quad(\mbox{or } d \geq a > b > c).
  \end{equation*}
  Then $F$ has a periodic orbit of every period.
  (Note that the hypothesis is satisfied is $F$ has a periodic orbit of period 3.)
  \label{thm:liyorke1}
\end{theorem}

\begin{theorem}
  (Period three implies Li-Yorke chaos)
  Suppose the same hypothesis as theorem~\ref{thm:liyorke1}.
  Then $F$ is chaotic in Li-Yorke's sense.
  \label{thm:liyorke2}
\end{theorem}

\noindent We will prove later that both logistic map and baker's map are Li-Yorke chaotic by showing that the mappings have 3-periodic orbits.

One may be surprised that the innocent condition gives rise to a complex behaviour.
However, Alexander Sarkovskii, a Soviet mathematician, would tell us to develop a better appreciation of a 3-periodic orbit.
\citet{sarkovskii} proved a theorem similar to Theorem~\ref{thm:liyorke1} ten years before the publication of ``Period Three Implies Chaos.''
Sarkovskii's result extends Li-Yorke's Theorem~\ref{thm:liyorke1} by proving analogous results for period 5, 7, 9, \ldots, and so on.
To be more specific, Sarkovskii's theorem states that there exists an ordering of $\mathbb{N}$---called \textit{Sarkovskii's ordering}---such that if a mapping has a $k$-periodic orbit, then it has periodic orbits of \textit{all} periods that come after $k$ in the ordering.
Three comes the first in Sarkovskii's ordering, and in this sense, the number three is at the top of the Sarkovskii's hierarchy of the natural numbers.
Appendix contains the Sarkovskii's ordering and its proof.

Now we proceed to prove Theorem~\ref{thm:liyorke1} and Theorem~\ref{thm:liyorke2}.
We shall start with proving three lemmas that we require for proving the theorems.

\begin{lemma}
  Let $I$ be an interval, $F: I \to \R$ a continuous map.
  For each compact interval $I_1 \subseteq F(I)$, there exists a compact interval $Q \subseteq I$ such that $G(Q) = I_1$.
  \label{lem:liyorke1}
  \begin{proof}
    Suppose $I_1 = [F(p),F(q)]$ for some $p,q \in I$.
    Without loss of generality, we may assume $p < q$, for otherwise, we merely switch the roles of $p$ and $q$ in the proof.
    Let $Q = [r,s]$, where $r$ is the greatest point in $[p,q]$ such that $F(r) = F(p)$, and $s$ is the least point in $(r,q]$ such that $F(s) = F(q)$.
    Since $F$ is continuous, $F(Q) = I_1$.
  \end{proof}
\end{lemma}

\begin{lemma}
  Let $I$ be an interval, and $F: I\to I$ be a continuous map.
  Suppose that for each $n$, $\seq{I_n}{\infty}{n=0}$ is a sequence of compact intervals with $I_n \subseteq I$ and $I_{n+1} \subseteq F(I_n)$.
  Then, there exists a sequence $\seq{Q_n}{\infty}{n=0}$ of compact intervals such that 
  

\begin{equation*}
    Q_{n+1} \subseteq Q_{n} \subseteq I_0,
  \end{equation*}
and, for each $n$,
  \begin{equation*}
    \itr{F}{n}(Q_n) = I_n.
  \end{equation*}
  Furthermore, for any $x \in Q \equiv \bigcap\limits_{n=0}^{\infty}Q_n$, we have for each $n$
  \begin{equation*}
    \itr{F}{n}(x) \in I_n.
  \end{equation*}
  \label{lem:liyorke2}
  %%%
  \begin{proof}
    Proof by induction.
    Let $Q_0 = I_0$.
    Clearly, we have $\itr{F}{0}(Q_0) = I_0$.
    For each $n$, define $Q_{n-1}$ as a compact interval such that $\itr{F}{n-1}(Q_{n-1}) = I_{n-1}$.
    Note that $I_n \subseteq F(I_{n-1}) = \itr{F}{n}(Q_{n-1})$.
    Lemma~\ref{lem:liyorke1} applied to $\itr{F}{n}$ on $Q_{n-1}$ proves the existence of a compact interval $Q_n \subseteq Q_{n-1}$ such that $\itr{F}{n}(Q_n) = I_n$.
  \end{proof}
\end{lemma}

\begin{lemma}
  Let $I$ be an interval, and $F: I\to \R$ be a continuous map.
  Suppose $J \subseteq I$ is a compact interval and $J \subseteq F(J)$. 
  Then, there exists a point $p \in I$ such that $G(p) = p$.
  \begin{proof}
    This is a special case of the contraction fixed point theorem proved in Preliminaries.
  \end{proof}
  \label{lem:liyorke3}
\end{lemma}

Lemma~\ref{lem:liyorke2} will be especially useful in the proofs.
The proof of Theorem~\ref{thm:liyorke1} introduces ideas that are used in proving the other theorem.
For brevity, we call $[a,b]$ and $[b,c]$ as $K$ and $L$, respectively.
The key observation to make is that we have $K,L \subseteq F(L)$ by $F(b) = c$, $F(c) = d$, and continuity of $F$.
Then, we can design an orbit that stays in $L$ as long as we wish.

\begin{proof}[Proof of Theorem~\ref{thm:liyorke1}.]
  We will prove the theorem for $d \leq a < b < c$.
  Define two intervals $K,L$ as follows
  \begin{equation*}
    K = [a,b] \mbox{ and } L = [b,c].
  \end{equation*}
  Let $k$ be a positive integer.
  For each $k > 1$, let $\set{I_n}$ be the sequence of intervals $I_n = L$ for $n = 0, \ldots, k-2$, and $I_{k-1} = K$.
  Furthermore, define $I_n$ for $n \geq k$ by $I_{n+k} = I_n$.
  We associate this sequence with an orbit of some interval $J$ such that
  \begin{equation*}
    \itr{F}{n}(J) \subseteq I_n.
  \end{equation*}
  Note that we have 
  \begin{equation*}
    K,L \subseteq F(L) \mbox{ and } L \subseteq F(K).
  \end{equation*}
  Then, by Lemma~\ref{lem:liyorke2}, there exists a sequence of compact intervals $\set{Q_n}$ such that $Q_{n+1} \subseteq Q_{n}$; in particular $Q_k \subseteq Q_0$.
  Also by Lemma~\ref{lem:liyorke2}, we have $\itr{F}{k}(Q_k) = I_0 = Q_0$.
  Since $\itr{F}{k}$ is continuous and $Q_k \subseteq \itr{F}{k}(Q_k)$, we may apply Lemma~\ref{lem:liyorke3} to conclude that $\itr{F}{k}$ has a fixed point $p_k \in Q_k$.
  We claim that $p_k$ does not have period less than $k$.
  To see this, assume the contrary and let the period of $p_k$ be $t < k$.
  Since $p_k \in Q_n \subseteq Q_{n-1}$, we have 
  \begin{equation*}
    \itr{F}{k-1}(p_k) \in I_{k-1} = K.
  \end{equation*}
  Then, since $p_k$ is $t$-periodic, we need 
  \begin{equation*}
    \itr{F}{k-1-t}(p_k) \in K.
  \end{equation*}
  In addition, we have
  \begin{equation*}
    \itr{F}{k-1-t}(p_k) \in I_{k-1-t} = L.
  \end{equation*}
  Hence, $\itr{F}{k-1-t}(p_k) \in L \cap K = \set{b}$.
  It follows that $\itr{F}{k-1}(p_k) = b$, and therefore, $\itr{F}{k+1}(p_k) = a$.
  This is a contradiction, because by construction, $\itr{F}{k+1}(p_k)$ must be in $L$.
  Thus, $p_k$ is a $k$-periodic point for $F$.
  The other case, $d \leq a < b < c$, can be proved in the same manner.
\end{proof}

Next, we prove the main result of this chapter.
\begin{proof}[Proof of Theorem~\ref{thm:liyorke2}.]
  Let $\mathcal{M}$ be the set of sequences $M = \seq{M_n}{n=1}{\infty}$.
\end{proof}

We already know that points in $S$ are not periodic.
However, we may draw an even stronger conclusion, that is, the points in $S$ are not even asymptotically periodic.
\begin{theorem}
  (Points in $S$ are not asymptotically periodic)
  For every $p \in S$ and periodic point $q \in I$,
  \begin{equation*}
    \limsup\limits_{n \to \infty}\; \abs{\itr{F}{n}(p) - \itr{F}{n}(q)} > 0.
  \end{equation*}
  \label{thm:liyorke3}
\end{theorem}

%We will see that Marotto's definition is equivalent to Li-Yorke's.
%Moreover, Marotto's definition is generalizable to $n$-space.

The theorem does not apply to transformations that are not continuous, such as the sawtooth transformation.
However, as we have shown in the chapter on conjugacy, the sawtooth transformation is conjugate to the tent map by a surjecive, continuous transformation.
The tent map is chaotic in Li-Yorke's sense, since it is continuous, and has a 3-periodic orbit.
This observation leads to the following conjecture.

\begin{conjecture}
  Let $X$ be a compact metric space, and $F: X\to X$ be chaotic in Li-Yorke's sense.
  If $G$ is conjugate to $F$ (by a continuous surjective transformation?), then $G$ is chaotic.
\end{conjecture}

Although the definition by Li and Yorke captures our intuitive notion of chaos well, it is by no means free of flaws.
There are mappings that are chaotic on a negligible set \citep{martelli98}.
\begin{example}
  (Chaos on a negligible set)
  Let $I = [0,1]$ and $F: I \to I$ defined as
  \begin{equation*}
    f(x) =
    \begin{cases}
      0 \quad & 0\leq x < \frac{1}{4} \\
      4x - 1 \quad  & \frac{1}{4} \leq x < \frac{1}{2} \\
      -4x + 3 \quad & \frac{1}{2} \leq x < \frac{3}{4} \\
      0 \quad & \frac{3}{4} \leq x \leq 1. \\
    \end{cases}
  \end{equation*}
  Clearly, $F$ is continuous.
  Also, $x = 23/65$ is a 3-periodic point:
  \begin{equation*}
    \itr{F}{3}\paren{\frac{23}{65}} = \itr{F}{2}\paren{\frac{27}{65}} = F\paren{\frac{43}{65}} = \frac{23}{65}.
  \end{equation*}
Hence, by Theorem~\ref{thm:liyorke2}, $F$ is chaotic in Li-Yorke's sense.
%
Next, we will show that, for almost every $x_0 \in I$ (with respect to the Lebesgue measure), the orbit $\set{\itr{F}{n}(x_0)}$ converges to 0.
First, note that 
\begin{equation*}
  F^{-1}\paren{\left(\frac{2}{3},1\right]} 
  = \paren{\frac{5}{12},\frac{7}{12}}.
\end{equation*}
Let $I_0 = (5/12, 7/12)$.
Then, inductively define $I_n (n\geq 1)$ as 
\begin{equation*}
  I_n = F^{-1}(I_{n-1}).
\end{equation*}
%
For convenience, we denote the length of an interval as $m(I)$.
We claim that
\begin{enumerate}[(1)]
  \item For $q\neq m$, $I_q \cap I_m = \emptyset$.
  \item $m(I_n) = 1/(6 \cdot 2^n)$
\end{enumerate}
%
We prove claim (1) by contradiction.
For any $q,m$, assume there exists $x \in I_q \cap I_m$. 
Without loss of generality, we may assume that $q > m$.
By the definition of $I_m$, we have $F^m(x) \in I_0$.
It follows that $F^{m+1}(x) \in F(I_0) = (2/3, 1]$, and $F^{m+t}(x) \in [0, 1/3)$ for $t \geq 2$.
However, this is a contradiction, because we must also have $F^q(x) \in I_0$.
%
To prove (2), first note that $F$ restricted to $(1/3, 2/3)$ is a piecewise affine transformation.
Consider $g_1(x) = 4x - 1$ and $g_2(x) = -4x + 3$ restricted to $(1/3,1/2]$ and $[1/2,2/3)$, respectively.
The inverse functions of $g_1$ and $g_2$ exist, and they are
\begin{equation*}
  g_1^{-1}(x) = \frac{1+x}{4}; \quad g_2^{-1} = \frac{3-x}{4}.
\end{equation*}
The images of $g_1^{-1}$ and $g_2^{-1}$ are $(1/3,2/3)$, and the inverse functions are continuous.
It follows that, if we suppose $J$ is an interval in $(1/3,2/3)$, then $J$ is mapped to $(1/3,1/2]$ by $g_1^{-1}$ and to $[1/2,2/3)$ by $g_2^{-1}$, respectively.
Then, $m(g_1^{-1}(J))$ and $m(g_2^{-1}(J))$ are disjoint (except for the point $x = 1/2$), and also
\begin{equation*}
  4 \cdot m(g_1^{-1}(J)) = 4 \cdot m(g_2^{-1}(J)) = m(J).
\end{equation*}
Since $m(I_0) = 1/6$, we have
\begin{equation*}
  m(I_1) 
  = m(g_1^{-1}(I_0)) + m(g_2^{-1}(I_0))
  = 2 * \frac{1}{6} * \frac{1}{4}
  = \frac{1}{6\cdot 2^1},
\end{equation*}
and inductively, we obtain
\begin{equation*}
  m(I_n) = \frac{1}{6 \cdot 2^n}.
\end{equation*}
%
Finally, by (1) and (2) we have
\begin{equation*}
  \sum\limits_{n\geq 0} m(I_n)
  = \frac{1}{3} \cdot \sum\limits_{n\geq 0} \frac{1}{2^{n+1}}
  = \frac{1}{3}.
\end{equation*}
Thus, almost every point in $I$ converges to 0 under iteration by $F$.
\end{example}

% Quasi continuous function f: X\to Y. If V\subseteq Y is open, then f^{-1}(V) \subseteq X is 'semi-open'.
% U \subseteq X is semi-open if U \subseteq clo(int(U))

There are other limitations to this definition of chaos.
Most notably, Li-Yorke's definition is limited to one-dimensional maps.
Also, there are maps that are chaotic in Li-Yorke's sense, but their topological entropies are zero.
We will discuss the latter case in more depth in the chapter on topological entropy.

\bibliographystyle{../../bibliography/pjgsm}
\bibliography{../../bibliography/thesis}

\printindex
\end{document}
