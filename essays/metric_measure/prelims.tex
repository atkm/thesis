\documentclass[11pt]{article}
\usepackage{thesis}

\begin{document}
\section{Dynamical Systems}
\begin{definition}
  (Orbit)
  Let $F: \R^n \to \R^n$. 
  The \textbf{orbit of $F$} for (initial value) $x_0$ is defined as
  \begin{equation*}
    O_F(x_0) = \set{x \; | \; x = F^n(x_0) \mbox{ for some } n\in \N}
  \end{equation*}
  \label{def:orbit}
\end{definition}
$O_F$ will simply be denoted as $O$ when there is no ambiguity in the choice of $F$ in the context.

\begin{definition}
  (Periodic Orbits)
  \label{def:porbit}
\end{definition}

The usual notion of limit points are extended to limit set of an orbit.
\begin{definition}
  (Limit Set)
  Limit set, $L_F(x_0)$ of $O_F(x_0)$ is the set of limit points of $x \in O_F(x_0)$.
  \label{def:limset}
\end{definition}

With the notion of limit sets, we may speak of aympototical periodicity.
\begin{definition}
  (Asymptotically periodic orbit)
  An orbit $O(x_0)$ is said to be asymptotically periodic if its limit set is a periodic orbit.
  \label{def:asymporb}
\end{definition}

\begin{definition}
  (Aperiodic orbit)
  The orbit $O_F(x_0)$ is said to be aperiodic if its limit set $L_F(x_0)$ is not finite.
  \label{def:aporbit}
\end{definition}


\section{Metric Spaces}
\begin{definition}
  (Metric)
\end{definition}
\begin{definition}
  (Open balls)
\end{definition}
\begin{definition}
  (Distance between a point and an orbit)
\end{definition}

\section{Measure Spaces}
\begin{definition}
  (Null Set)
\end{definition}
\begin{definition}
  (Measure)
\end{definition}
\end{document}
